\documentclass[11pt]{report}

\usepackage{packages}

\begin{document}

\begin{titlepage}
\vbox{ }
\vbox{ }
\begin{center}

%------------------------------------------
% Logo
%------------------------------------------
\includegraphics[width=0.40\textwidth]{figures/NTNU_logo.png}\\[1cm]

%------------------------------------------
% Department and Course
%------------------------------------------
\textsc{\LARGE Department of Physics}\\[1.5cm]
\textsc{\Large TFY4510 -- Physics, Specialization Project}\\[0.5cm]

\vspace{0.5cm}

%------------------------------------------
% Title
%------------------------------------------
\HRule \\[0.4cm]
{\huge \bfseries Modelling Neutron Stars}\\[0.4cm]
{\Large A Journey from Simple Fermion Models to Interacting Nuclear Matter}
\HRule \\[1.5cm]

%------------------------------------------
% Author and Supervisor
%------------------------------------------
\large
\emph{Author:}\\
Sondre Klyve\\[0.8cm]

\emph{Supervisor:}\\
Jens Oluf Andersen

\vfill

%------------------------------------------
% Date
%------------------------------------------
{\large \today}

\end{center}
\end{titlepage}


\frontmatter

\begin{abstract}
Neutron stars contain matter at extremely high densities, where both general relativity and
quantum many-body physics are required to describe their structure. In this project, we study
how different models of dense matter affect the relation between the mass and radius of a
neutron star. The analysis is based on the Tolman–Oppenheimer–Volkoff equations, which are solved
numerically for various equations of state.

We first consider a simple model, namely an ideal, non-interacting Fermi gas. This illustrates how
degeneracy pressure can counteract gravitational collapse, while also demonstrating the limitations
of such idealised descriptions when confronted with observed neutron-star properties.
Building on this, the main part of the project develops an interacting model for nucleons and
leptons in equilibrium. The interactions are described through self-consistent meson fields,
leading to a more realistic equation of state, which is combined with a standard crust
model before being used in the numerical stellar calculations.

The resulting mass–radius relations yield radii typical of neutron stars,
but they do not support the heaviest observed stars and show clear differences from current
astronomical constraints. This demonstrates that while the interacting model improves upon the
ideal, non-interacting Fermi gas description, it remains incomplete at the highest densities.

Overall, the project presents a coherent path from ideal models to a description of interacting
dense matter and shows the need for more advanced equations of state in order to reproduce
the full range of observed neutron-star properties.
\end{abstract}

\tableofcontents

\mainmatter

\chapter{Introduction}
\label{chap:introduction}

\begin{quote}\itshape
When I heard the learn’d astronomer,\\
When the proofs, the figures, were arranged in columns before me,\\
When I was shown the charts and the diagrams, to add, divide, and measure them,\\
When I, sitting, heard the astronomer where he lectured with much applause in the lecture-room,\\
How soon unaccountable I became tired and sick,\\
Till rising and gliding out, I wandered off by myself,\\
In the mystical moist night-air, and from time to time,\\
Look’d up in perfect silence at the stars.
\par\raggedleft
Walt Whitman, \emph{When I Heard the Learn’d Astronomer}
\end{quote}

On a clear night, far from streets and houses and the glow of city lights, the
sky opens up into a dark dome scattered with starlight. The stars do not just
appear as distant lamps; they give the night a quiet structure. Constellations
trace out shapes, familiar or imagined, and slow, patient patterns in their
motion mark the passing of the seasons. For most of human history, these
lights have been woven into stories and beliefs. They have served as a calendar
for planting and harvest, as a map for travellers and sailors, and as a canvas
for myths about origins and destiny. Long before anyone knew what a star
really is, people understood that the sky above was something both constant and
mysterious.

Among all these lights there is one that dominates our everyday experience,
although we rarely think of it as a star at all: the Sun. Its rising and
setting define the rhythm of our days. Its warmth makes liquid water and a
mild climate possible on Earth. The light it provides is taken up by plants and
turned into food, and through countless steps in the chain of life it becomes
our own energy. When clouds cover the sky for days, we feel its absence; when
it breaks through after a long winter, the world changes character. In a very
direct sense, everything familiar on the surface of our planet depends on this
one ordinary star.

From a physical point of view, stars like the Sun are enormous, hot spheres of
ionized gas that shine because of nuclear reactions in their interiors. Their
size and shape are maintained by a balance between two competing effects:
gravity, which pulls all the matter inward, and pressure from hot, fast-moving
particles and radiation, which pushes outward. For most of their active lives
these effects balance each other in a relatively stable way. This is the stage
we usually picture when we think about stars: long-lived objects steadily
emitting light and heat over billions of years.

However, this is only one part of a star's history. More broadly, a star passes
through three qualitative stages: an initial phase of formation, a long middle
period of more or less stable burning, and a final stage in which it exhausts
its fuel and dies. Stars are born when cold, diffuse clouds of gas in a galaxy
slowly contract under their own gravity. As a cloud fragment collapses, its
central regions become denser and hotter. When the conditions are extreme
enough, nuclear fusion reactions start in the core: light nuclei combine into
heavier ones, releasing energy. Once the energy production settles into a
stable pattern that balances gravity, the newly formed star enters its
long-lived main sequence phase.

During this main sequence phase the star shines by steadily fusing hydrogen
into helium in its core. Gravity continues to draw matter inward, but the
pressure from the hot plasma, together with the energy released by fusion,
pushes outward. As long as the star can sustain this balance, it remains
stable. Over time, however, the hydrogen in the core is gradually used up. When
the easy nuclear fuel is exhausted, the fusion reactions slow down and can no
longer fully support the star against its own gravity. The core contracts and
heats up, and the outer layers adjust, sometimes expanding into a red giant. In
more massive stars, further nuclear reactions involving heavier nuclei can
occur, but this chain cannot continue indefinitely. At some point, fusing
heavier elements no longer produces energy. The star then reaches the end of
its active life.

What happens next depends mainly on how massive the star is. If the remaining
core is relatively light, it can settle into a compact object called a white
dwarf: a stellar remnant with roughly the mass of the Sun compressed into a
volume similar to that of the Earth. White dwarfs are held up by quantum
mechanical effects rather than by ordinary thermal pressure. More massive stars
do not stop there. Their cores continue to collapse, and the outer layers can
be expelled in a violent explosion known as a supernova. The collapsing core is
subjected to extreme densities, and very different end states become possible.

For stars with sufficiently massive cores, the collapse can be so extreme that
even the quantum resistance of electrons is no longer enough to halt it. In
this case the core is compressed to densities comparable to those inside atomic
nuclei. Under such conditions, protons and electrons are driven together to
form neutrons and neutrinos. The result is a neutron star: an object containing
about one to two times the mass of the Sun squeezed into a sphere with a radius
of only about ten to fifteen kilometres. To put this into perspective, a
tablespoon of neutron-star material would weigh about the same as Mount Everest. 
If the mass of the collapsing core is even larger, gravity can overcome
all known forms of pressure and the end product is a black hole.

Neutron stars occupy an intermediate position between white dwarfs and black
holes. They are compact enough that the effects of Einstein's theory of general
relativity are essential for describing their structure and gravitational
field, but they are not so extreme as to be hidden behind an event horizon.
Their densities are so high that matter is pushed far beyond the conditions
that can be created in laboratories on Earth. A neutron star therefore
provides a unique natural laboratory for physics at the intersection of
gravity, nuclear physics, and quantum theory. On the one hand, our theoretical
description of these objects is based on extrapolating physical laws that have
been tested in less extreme regimes. On the other hand, observations of neutron
stars allow us to test and refine those laws under conditions that would
otherwise be inaccessible.

The basic idea that such ultra-dense stellar remnants might exist dates back to
the early 1930s, when it was first realized that extremely dense matter could
consist mainly of neutrons. Shortly afterwards, neutron stars were proposed as
possible remnants of supernova explosions: stellar cores so dense that
electrons and protons are forced to combine into neutrons. For several decades
this remained a theoretical speculation. The situation changed in the late
1960s with the discovery of pulsars: rapidly rotating, highly regular sources
of radio pulses. These were soon interpreted as spinning neutron stars whose
strong magnetic fields sweep beams of radiation across our line of sight. Since
then, neutron stars have been observed in many different settings, including
binary systems where matter is accreted from a companion star, and in systems
where two neutron stars orbit around each other and eventually merge.

As observational techniques have improved, the range and precision of neutron
star measurements have increased significantly. High-precision radio timing
allows very accurate mass measurements for pulsars in binary systems. X-ray
observations of the hot surfaces of neutron stars give information that can be
used, with some modelling, to infer their radii. Gravitational-wave detections
from merging neutron star binaries add further constraints, because the details
of the signal depend on how easily the stars deform under tidal forces. All of
these observations can be translated into bounds on the relationship between
the mass and radius of neutron stars.

On the theoretical side, the structure of a non-rotating neutron star is
governed by two main ingredients. The first is general relativity, which
provides the equations that describe how gravity behaves in a static,
spherically symmetric star. These equations express the balance between
gravity, pressure, and mass at each radius in such a way that the star can
remain in hydrostatic equilibrium. The second ingredient is an input that tells
us how the pressure inside the star depends on the density: the so-called
equation of state. Physically, the equation of state summarises our knowledge
of what the star is made of and how its constituents interact. Once it is
specified, the relativistic structure equations can be integrated from the
centre outwards to give a complete model of the star and, in particular, a
prediction of how mass and radius are related.

The equation of state of neutron-star matter at very high densities is not
known with certainty. In the outer regions of the star, matter is thought to
be relatively similar to very dense nuclear matter, arranged in a crust of
nuclei and electrons. Deeper inside, the density increases far beyond values
that can be reached in laboratory experiments, and our knowledge becomes much
more uncertain. Different theoretical models make different assumptions about
how matter behaves in this regime, and each choice leads to a different
equation of state and therefore to a different prediction for the relationship
between mass and radius. Comparing these predictions with astrophysical
observations is one of the main ways to learn about the behaviour of matter
under such extreme conditions.

The aim of this thesis is to explore, in a controlled and stepwise way, how
different theoretical descriptions of dense matter translate into different
neutron-star structures. The focus is on a small number of models that are
simple enough to be treated in detail, but still realistic enough to show how
assumptions about the matter inside the star shape observable quantities such
as mass and radius. 

In the first main chapter, we
introduce the framework of general relativity for static, spherically symmetric
stars and derive the Tolman-Oppenheimer-Volkov (TOV) equations. 
These equations express the balance between
gravity and pressure in a compact star and form the backbone of all later
calculations. In the second chapter, we study ideal Fermi gases at zero
temperature. This provides a simple and transparent description of degenerate
matter and serves as a reference point for more realistic models. In the third
chapter, we combine these ingredients to build ideal neutron-star models based
on non-interacting neutrons, solve the TOV equations numerically, and analyse
the resulting mass--radius curves and their radial stability.

Having established this baseline, the fourth chapter turns to a more realistic
description of neutron-star matter in terms of the so-called \(npe\mu\) model,
in which neutrons and protons interact through a relativistic mean-field
theory and are accompanied by electrons and muons to ensure charge neutrality
and equilibrium. Within this framework, we construct the equation of state
self-consistently and extend it towards lower densities to include a simple
crust description. The resulting \(npe\mu\) equation of state is then used as
input for the TOV equations to compute mass--radius relations. Finally, these
theoretical curves are compared with observationally inferred bands for the
equation of state and for the mass--radius relation, illustrating how the
underlying model assumptions are reflected in the observable properties of
neutron stars.

\chapter{Tolman–Oppenheimer–Volkoff Equations}
\label{chap:tov}

\section{Introduction}

The structure of compact stars cannot always be captured within the framework of Newtonian gravity. 
For white dwarfs, where electrons are relativistic but the gravitational field is relatively weak, 
Newtonian gravity combined with special relativity provides an adequate description. 
In contrast, neutron stars and hypothetical hybrid stars reach densities comparable to the nuclear 
saturation density $\rho_{0}$, which is the characteristic density inside atomic nuclei. 
At such densities, the curvature of spacetime becomes so strong that general relativity is indispensable. 
It provides the natural framework to describe how matter curves spacetime and, conversely, 
how this curved geometry governs the equilibrium of stellar matter.

In this chapter we derive the equations governing static, spherically symmetric configurations of 
perfect fluids in general relativity, known as the Tolman–Oppenheimer–Volkoff (TOV) equations \cite{oppenheimer1939}.
They generalize the Newtonian hydrostatic equilibrium equation to curved spacetime. 
Hence, the TOV equations form the central tool for studying compact stars such as neutron stars 
and hybrid stars with exotic cores, while also providing a relativistic extension of the classic treatment 
that suffices for white dwarfs. They link microscopic physics, encoded in the equation of state of dense matter, 
to macroscopic observables such as stellar masses and radii.

Our starting point will be Einstein's field equations, together with the assumption of spherical 
symmetry and a perfect-fluid energy–momentum tensor. From these ingredients we obtain a set of 
coupled ordinary differential equations relating the pressure, energy density, and enclosed 
gravitational mass as functions of the radial coordinate. To close the system, one must specify 
an equation of state, thereby providing the crucial connection between microphysics and 
astrophysical observables.

Historically, these equations were first formulated independently by Tolman \cite{tolman1939} 
and by Oppenheimer and Volkoff. Their pioneering work laid the foundation 
for modern neutron star astrophysics, showing that general relativity predicts a maximum mass for 
compact stars beyond which no stable configuration exists. This insight still guides research into 
the ultimate fate of dense matter and the physics of gravitational collapse.

The derivation presented here follows the standard approach used in the literature,
and closely follows the steps from Carroll’s \emph{Spacetime and Geometry}.
\cite{oppenheimer1939, tolman1939, carroll}. 

\section{Metric ansatz and coordinates}
We now restrict outselves to static, spherically symmetric stellar configurations. 
By Birkhoff’s theorem, the most general spherically symmetric vacuum solution of 
Einstein’s equations is the Schwarzschild metric \cite{carroll}. Inside the star, where matter is present, 
the metric must reduce continuously to the Schwarzschild form at the stellar surface. 
It is therefore natural to adopt Schwarzschild–like (curvature) coordinates and write the 
line element as
\begin{equation}
ds^{2} = A(r)\, dt^{2} - B(r)\, dr^{2} - r^{2}\,(d\theta^{2}+\sin^{2}\theta\, d\phi^{2}),
\label{eq:metric_ansatz}
\end{equation}
where $A(r)$ and $B(r)$ are functions of the radial coordinate only. 
The assumption of staticity ensures that no cross terms such as $dr\,dt$ appear, 
while spherical symmetry fixes the angular part to the standard 
$r^{2} d\Omega^{2}$ form.

\section{Explicit computation of Einstein tensor components}

It is convenient to rewrite the metric functions by
\begin{equation}
A(r)=e^{2\alpha(r)},\qquad B(r)=e^{2\beta(r)},
\end{equation}
so that the line element becomes
\begin{equation}
ds^{2}=e^{2\alpha(r)}\,dt^{2}-e^{2\beta(r)}\,dr^{2}-r^{2}\,\bigl(d\theta^{2}+\sin^{2}\theta\,d\phi^{2}\bigr).
\label{eq:sss_alpha_beta}
\end{equation}
Using the definition of the Christoffel symbols from Eq.~\eqref{eq:christoffel_def}, one finds the nonvanishing components
\begin{align}
\Gamma^{t}{}_{tr} &= \alpha', &
\Gamma^{r}{}_{tt} &= \alpha' e^{2(\alpha-\beta)}, &
\Gamma^{r}{}_{rr} &= \beta', \nonumber\\[4pt]
\Gamma^{r}{}_{\theta\theta} &= -r\,e^{-2\beta}, &
\Gamma^{r}{}_{\phi\phi} &= -r\,e^{-2\beta}\sin^{2}\theta, &
\Gamma^{\theta}{}_{r\theta} &= \Gamma^{\phi}{}_{r\phi}=\tfrac{1}{r}, \nonumber\\[2pt]
\Gamma^{\theta}{}_{\phi\phi} &= -\sin\theta\cos\theta, &
\Gamma^{\phi}{}_{\theta\phi} &= \cot\theta,
\label{eq:Gamma_subset}
\end{align}
where primes denote derivatives with respect to $r$.
From the definition of the Ricci tensor in Eq.~\eqref{eq:ricci_def},
stationarity ($\partial_{t}=0$) and spherical symmetry simplify the components. 
As an example, for $R_{tt}$ one finds
\begin{align}
R_{tt}
&=\partial_{r}\Gamma^{r}{}_{tt}
+\Gamma^{r}{}_{tt}\!\left(\Gamma^{t}{}_{rt}+\Gamma^{r}{}_{rr}+\Gamma^{\theta}{}_{r\theta}+\Gamma^{\phi}{}_{r\phi}\right)
-\Gamma^{t}{}_{tr}\Gamma^{r}{}_{tt}\nonumber\\[2pt]
&=e^{2(\alpha-\beta)}\!\left[\alpha''+(\alpha')^{2}-\alpha'\beta'+\frac{2\alpha'}{r}\right].
\label{eq:Rtt_result}
\end{align}
In addition to $R_{tt}$, one must also evaluate the radial and angular components 
($R_{rr}$, $R_{\theta\theta}$ and $R_{\phi\phi}$). 
The Ricci scalar is then obtained by contraction $R = g^{\mu\nu}R_{\mu\nu}$, which yields
\begin{equation}
R
=2e^{-2\beta}\!\left[\alpha''+(\alpha')^{2}-\alpha'\beta'+\frac{2}{r}\,(\alpha'-\beta')\right]
-\frac{2}{r^{2}}\!\left(1-e^{-2\beta}\right).
\label{eq:R_scalar}
\end{equation}
Finally, using the definition of the Einstein tensor in
Eq.~\eqref{eq:einstein_tensor_def}, the nonvanishing components are found to be
\begin{align}
G_{tt} &= \frac{e^{2(\alpha-\beta)}}{r^{2}}\Big(2r\,\beta'-1+e^{2\beta}\Big), \label{eq:Gtt_final}\\[6pt]
G_{rr} &= \frac{1}{r^{2}}\Big(2r\,\alpha'+1-e^{2\beta}\Big), \nonumber\\[6pt]
G_{\theta\theta} &= e^{-2\beta}\!\left[\alpha''+(\alpha')^{2}-\alpha'\beta'
+\frac{\alpha'-\beta'}{r}\right]r^{2}, \label{eq:GrGth_cov}\\[6pt]
G_{\phi\phi} &= \sin^{2}\theta\,G_{\theta\theta}. \nonumber
\end{align}


\paragraph{Mixed components.}
It is often convenient to work with $G^{\mu}{}_{\nu}=g^{\mu\lambda}G_{\lambda\nu}$. 
For the metric \eqref{eq:sss_alpha_beta} these read
\begin{align}
G^{t}{}_{t} &= \frac{e^{-2\beta}}{r^{2}}\Bigl(2r\,\beta'-1+e^{2\beta}\Bigr),\label{eq:Gtt_mixed}\\
G^{r}{}_{r} &= \frac{e^{-2\beta}}{r^{2}}\Bigl(-2r\,\alpha'-1+e^{2\beta}\Bigr),\label{eq:Grr_mixed}\\
G^{\theta}{}_{\theta} &= G^{\phi}{}_{\phi}
= -\,e^{-2\beta}\!\left[\alpha''+(\alpha')^{2}-\alpha'\beta'+\frac{\alpha'-\beta'}{r}\right].
\label{eq:Gang_mixed}
\end{align}

\section{Einstein equations with a perfect fluid source}

In relativistic astrophysics, stellar matter is commonly modeled as a 
perfect fluid. A perfect fluid is an idealized medium that has 
no viscosity, no heat conduction, and is completely characterized by 
its local rest--frame energy density $\epsilon$ and isotropic pressure $P$. 
This means that, in the rest frame of the fluid, the stress is the same 
in all spatial directions and there are no dissipative processes such 
as friction or heat flow. While real nuclear matter may exhibit more 
complicated transport properties, the perfect fluid approximation 
captures the essential macroscopic features of dense stellar matter 
and provides a tractable starting point for deriving equilibrium equations.

Accordingly, we assume the perfect-fluid form \eqref{eq:perfect_fluid_Tmn}
with energy density $\epsilon(r)$, pressure $P(r)$, and four–velocity
$u^\mu=(e^{-\alpha},0,0,0)$ in the fluid rest frame.
Normalization $u^\mu u_\mu=1$ then implies $u_\mu=(e^{\alpha},0,0,0)$ 
for the metric \eqref{eq:sss_alpha_beta}.  
In mixed form the components take the simple diagonal structure
\begin{equation}
T^{\mu}{}_{\nu}=\mathrm{diag}\!\big(\epsilon,\,-P,\,-P,\,-P\big).
\end{equation}

\subsection*{Independent Einstein equations}

Equating \eqref{eq:Gtt_mixed}--\eqref{eq:Gang_mixed} to $8\pi T^{\mu}{}_{\nu}$ 
via \eqref{eq:einstein_equation} yields three independent equations,
\begin{align}
\frac{e^{-2\beta}}{r^{2}}\Bigl(2r\,\beta'-1+e^{2\beta}\Bigr) &= 8\pi\,\epsilon, 
\label{eq:Ein_tt}\\[6pt]
\frac{e^{-2\beta}}{r^{2}}\Bigl(-2r\,\alpha'-1+e^{2\beta}\Bigr) &= -\,8\pi\,P,
\label{eq:Ein_rr}\\[6pt]
e^{-2\beta}\!\left[\alpha''+(\alpha')^{2}-\alpha'\beta'
   +\tfrac{1}{r}(\alpha'-\beta')\right] &= -\,8\pi\,P.
\label{eq:Ein_thth}
\end{align}
By spherical symmetry $G^{\phi}{}_{\phi}=G^{\theta}{}_{\theta}$, 
so the $\phi\phi$ component provides no new equation beyond \eqref{eq:Ein_thth}.  

Each of these equations has a clear physical interpretation.  
Equation \eqref{eq:Ein_tt} links the $tt$ component of spacetime curvature 
to the local energy density $\epsilon(r)$; it plays the role of a relativistic 
generalization of Poisson’s equation in Newtonian gravity.  
Equation \eqref{eq:Ein_rr} involves the $rr$ component and connects the radial 
metric function $\alpha(r)$ to the local pressure $P(r)$. 
This equation expresses how the pressure gradient balances the inward pull of gravity 
to maintain hydrostatic equilibrium.  
Finally, Eq.~\eqref{eq:Ein_thth} comes from the angular components of the Einstein tensor. 
It requires that the pressure entering the field equations is the same in the radial 
and tangential directions, reflecting the assumption of an isotropic perfect fluid.  

Taken together, these equations encode the interplay between energy density, pressure, 
and spacetime curvature that governs the internal structure of relativistic stars.

\section{Mass function}

It is convenient to introduce the Misner--Sharp mass function, defined as
\begin{equation}
m(r)\equiv \frac{r}{2}\Bigl(1-e^{-2\beta(r)}\Bigr).
\label{eq:ms_mass}
\end{equation}
This quantity measures the deviation of $g_{rr}$ from flat space and reduces to the 
Schwarzschild mass in the exterior vacuum region. Inverting the definition gives
\begin{equation}
e^{2\beta(r)}=\Bigl(1-\frac{2m(r)}{r}\Bigr)^{-1}.
\label{eq:ms_mass_inv}
\end{equation}
Differentiating $m(r)$ with respect to $r$ yields
\begin{equation}
\frac{dm}{dr}=\frac{1}{2}\Bigl[1+e^{-2\beta}\,(\,2r\beta'-1\,)\Bigr].
\label{eq:dm_dr_from_def}
\end{equation}
Comparing \eqref{eq:dm_dr_from_def} with the $tt$--equation \eqref{eq:Ein_tt} immediately leads to
\begin{equation}
\frac{dm}{dr}=4\pi r^{2}\,\epsilon(r), \qquad m(0)=0.
\label{eq:mass_equation}
\end{equation}
For $r>R$, outside the stellar surface, $m(r)$ is constant and equal to the total gravitational 
mass of the star.  

At first sight, Eq.~\eqref{eq:mass_equation} resembles the Newtonian result 
$m(r)=\int_0^r 4\pi r'^2 \rho(r')\,dr'$, i.e.\ the rest--mass integrated over the coordinate volume. 
In general relativity, however, the proper spatial volume element contains an extra factor from the 
metric, $dV = 4\pi r'^2 e^{\beta(r')}dr'$, so the expression above is not a direct proper--volume 
integral of the energy density. Instead, $m(r)$ is defined geometrically: it is the Misner--Sharp 
mass, introduced in Eq.~\eqref{eq:ms_mass}, which enters the metric function $B(r)$ and governs the 
exterior Schwarzschild limit. The difference between the naive proper integral and $m(r)$ accounts 
for the star’s gravitational binding energy, so $m(r)$ should be interpreted as the total enclosed 
gravitational mass rather than a simple sum of local rest masses.


\section{Hydrostatic equilibrium: the TOV pressure equation}

In this section we derive the differential equation governing the pressure profile $P(r)$ 
in a static, spherically symmetric star. The result expresses local force balance between the 
inward pull of gravity and the outward pressure gradient. The derivation rests on two ingredients:
\begin{enumerate}
\item An expression for the radial metric potential $\alpha'(r)$ obtained from the 
$rr$ component of Einstein’s equations Eq.~\eqref{eq:Ein_rr}. 
This connects the gradient of the redshift factor $e^{\alpha}$ to local matter variables.
\item Local energy--momentum conservation, $\nabla_{\mu}T^{\mu}{}_{\nu}=0$, applied to 
the perfect fluid energy--momentum tensor. 
This provides a relation between the pressure gradient and the metric function $\alpha'(r)$.
\end{enumerate}
Combining these two relations yields the hydrostatic equilibrium equation, usually referred 
to as the Tolman--Oppenheimer--Volkoff equation. We will keep all intermediate steps 
explicit in order to make clear how general relativity modifies the familiar Newtonian result.

\subsection*{Solving the rr equation for the metric potential derivative}

From Eq.~\eqref{eq:Ein_rr} we obtain the $rr$ component in terms of the radial pressure profile,
\begin{equation}
\frac{e^{-2\beta}}{r^{2}}\Bigl(-2r\,\alpha'-1+e^{2\beta}\Bigr)=-\,8\pi\,P(r).
\label{eq:Ein_rr_repeat}
\end{equation}
Multiplying by $r^{2}e^{2\beta}$ eliminates the prefactor and, rearranging, we get 
\begin{equation}
2r\,\alpha' = e^{2\beta}\!\left(1+8\pi P\,r^{2}\right) - 1.
\end{equation}
Inserting the inverted definition of the Misner--Sharp mass, given in Eq.~\eqref{eq:ms_mass_inv}, we obtain
\begin{align}
2r\,\alpha' &= \frac{1+8\pi P\,r^{2}}{1-2m/r}-1 \nonumber\\
            &= \frac{1+8\pi P\,r^{2}-(1-2m/r)}{1-2m/r}\nonumber\\
            &= \frac{2m/r+8\pi P\,r^{2}}{1-2m/r}.
\end{align}
Hence
\begin{equation}
\boxed{\;
\alpha'(r)=\frac{m(r)+4\pi r^{3}P(r)}{r\,[\,r-2m(r)\,]}\,.
\;}
\label{eq:alpha_prime}
\end{equation}
This equation determines how the metric function $\alpha(r)$ varies with radius. 
Since $g_{tt}=e^{2\alpha(r)}$, $\alpha(r)$ controls the $tt$--component of the metric and 
therefore the redshift of clocks in the gravitational field. 
In the weak–field limit $r \gg 2m(r)$ and for small pressures, the denominator reduces to $r^{2}$ 
and the equation approaches $\alpha'(r)\simeq m(r)/r^{2}$, 
which is the familiar Newtonian form of the gravitational potential gradient.

\subsection*{Energy–momentum conservation in the radial direction}

For a perfect fluid with $T^{\mu}{}_{\nu}=\mathrm{diag}(\epsilon,-P,-P,-P)$ 
in the static frame and with metric \eqref{eq:sss_alpha_beta}, 
the only nontrivial conservation law is the radial one. 
Evaluating the covariant derivative gives
\begin{equation}
\nabla_{\mu}T^{\mu}{}_{r}
=\partial_{r}T^{r}{}_{r}
+\Gamma^{\mu}{}_{\mu r}\,T^{r}{}_{r}
-\Gamma^{\lambda}{}_{\mu r}\,T^{\mu}{}_{\lambda}
= -\,p' - (\epsilon+p)\,\alpha' = 0,
\label{eq:fluid_cons_component}
\end{equation}
where the relevant Christoffel symbols 
are those listed in \eqref{eq:Gamma_subset}.

This relation expresses local energy–momentum conservation in the radial direction: 
the change of pressure with radius is tied to the gradient of the metric function $\alpha(r)$. 
It can also be written in the standard form of the relativistic Euler equation,
\begin{equation}
(\epsilon+P)\,a_{\nu}=-\nabla_{\nu}p,\qquad 
a_{\nu}=u^{\mu}\nabla_{\mu}u_{\nu},
\end{equation}
where $a_{\nu}$ is the four–acceleration of the fluid. 
For our static configuration the only nonzero component is $a_{r}=\alpha'$, 
so the Euler equation reduces exactly to Eq.~\eqref{eq:fluid_cons_component}.

Equation \eqref{eq:fluid_cons_component} expresses the condition of hydrostatic 
equilibrium in general relativity: the pressure decreases outward in order to 
balance the effect of the gravitational field. 
Rearranging gives
\begin{equation}
\frac{dP}{dr}=-\,(\epsilon+P)\,\alpha'(r).
\label{eq:preTOV}
\end{equation}

\subsection*{The TOV pressure equation}

Substituting \eqref{eq:alpha_prime} into \eqref{eq:preTOV} gives the Tolman--Oppenheimer--Volkoff equation:
\begin{equation}
\boxed{\;
\frac{dP}{dr}
= -\,(\epsilon+P)\,\frac{m(r)+4\pi r^{3}P(r)}{r\,[\,r-2m(r)\,]}\,.
\;}
\label{eq:TOV_pressure}
\end{equation}
Together with the mass equation \eqref{eq:mass_equation} 
and an equation of state $P=P(\epsilon)$, these equations form a closed system of ODEs
for $(P(r),m(r))$.

\paragraph{Physical interpretation.}
Equation \eqref{eq:TOV_pressure} is the relativistic form of hydrostatic equilibrium.  
The factor $(\epsilon+P)$ shows that both energy density $\epsilon$ and pressure $P$ enter the balance 
between the pressure gradient and the gravitational field.  
The quantity $m(r)$ is the mass function introduced in Eq.~\eqref{eq:mass_equation}, and the term 
$4\pi r^{3}P$ appears directly from the Einstein field equations.  
The denominator $r(r-2m)$ reflects the effects of spacetime curvature and reduces to $r^{2}$ in the 
weak–field limit.  
For $2m/r \ll 1$ and $P \ll \epsilon$, the TOV equation reduces to the Newtonian hydrostatic balance,
\begin{equation}
\frac{dP}{dr}\approx -\,\rho(r)\,\frac{G\,m(r)}{r^{2}},
\end{equation}
where $\rho$ is the rest–mass density.  
Since $\epsilon\simeq \rho c^{2}$ in this limit, the relativistic and Newtonian forms are consistent.  

\paragraph{Boundary conditions and matching.}
At the center of the star, regularity requires that $\epsilon(0)$ and $P(0)$ are finite, and that $\alpha'(0)=0$.  
From the mass equation one finds $m(r)\sim \tfrac{4\pi}{3}\,\epsilon(0)\,r^{3}$ as $r\to 0$, 
and Eq.~\eqref{eq:TOV_pressure} then implies $P'(0)=0$.  
The stellar surface $R$ is defined by $P(R)=0$.  
Outside this radius the solution must match smoothly to the vacuum Schwarzschild metric, 
so $m(R)=M$ is the total mass and $e^{2\alpha(R)}=1-2M/R$.  
The integration constant for $\alpha$ is fixed by asymptotic flatness, 
so that $e^{2\alpha}\to 1$ as $r\to\infty$.  
The factor $(1-2M/R)$ is the Schwarzschild redshift factor, and the condition $R>2M$ 
ensures that the star is not a black hole.

\section*{Example: Uniform--density (``incompressible'') star}
\addcontentsline{toc}{section}{Example: Uniform--density (``incompressible'') star}

As an analytically solvable illustration of the TOV system, consider a star with
\begin{equation}
\epsilon(r)\equiv \epsilon_{0}=\text{const}.
\label{eq:const_density_assumption}
\end{equation}
This is \emph{not} a realistic equation of state: because $\epsilon$ does not change when $P$ changes, the adiabatic sound speed
\[
c_{s}^{2} \equiv \frac{dP}{d\epsilon}
\]
would be infinite (since $d\epsilon/dP=0$), i.e.\ acausal. We use the model only as a clean analytic foil and a code check when we later implement a numerical TOV solver.

\subsection*{Mass function and radial metric}
Integrating the mass equation \eqref{eq:mass_equation} gives
\begin{equation}
m(r)=\frac{4\pi}{3}\,\epsilon_{0}\,r^{3},\qquad
M\equiv m(R)=\frac{4\pi}{3}\,\epsilon_{0}\,R^{3}.
\label{eq:const_density_mass}
\end{equation}
From the definition \eqref{eq:ms_mass_inv} it follows that
\begin{equation}
e^{2\beta(r)}=\Bigl(1-\frac{2m(r)}{r}\Bigr)^{-1}
=\left(1-\frac{2M}{R^{3}}\,r^{2}\right)^{-1}.
\label{eq:beta_uniform}
\end{equation}
It is convenient to define
\begin{equation}
\Phi(r)\equiv \sqrt{1-\frac{2m(r)}{r}}=\sqrt{1-\frac{2M}{R^{3}}\,r^{2}},
\qquad
\Phi_{R}\equiv \Phi(R)=\sqrt{1-\frac{2M}{R}}.
\label{eq:Phi_defs}
\end{equation}

\subsection*{Pressure profile}
Combining hydrostatic balance \eqref{eq:preTOV} with \eqref{eq:alpha_prime} and the uniform density \eqref{eq:const_density_mass}, one separates variables and integrates from $r$ to $R$ using $P(R)=0$. The result is
\begin{equation}
P(r)
=\epsilon_{0}\,
\frac{\Phi_{r}-\Phi(R)}{3\,\Phi(R)-\Phi_{r}},
\label{eq:const_density_pressure}
\end{equation}
and, at the center ($\Phi(0)=1$),
\begin{equation}
P_{c}
=\epsilon_{0}\,
\frac{1-\Phi_{R}}{3\,\Phi_{R}-1}.
\label{eq:Pc_const_density}
\end{equation}

\subsection*{Compactness and the Buchdahl bound}
Define the compactness
\begin{equation}
\mathcal{C}\equiv \frac{2M}{R}.
\end{equation}
From \eqref{eq:Pc_const_density}, $P_c$ is finite for $\Phi_R\neq \frac{1}{3}$ and positive only if $3\,\Phi_R-1>0$, i.e.\ $\mathcal{C}<\frac{8}{9}$.
\begin{equation}
\Phi_{R}>\frac{1}{3}
\quad\Longleftrightarrow\quad
\mathcal{C}<\frac{8}{9}
\quad\Longleftrightarrow\quad
R>\frac{9}{4}\,M.
\label{eq:Buchdahl_bound}
\end{equation}
This inequality is the \emph{Buchdahl bound}: for static, spherically symmetric, isotropic fluids with nonincreasing $\epsilon(r)$ one must have $R\ge \frac{9}{4}M$ \cite{buchdahl1959}. The uniform--density solution makes the divergence as the bound is approached explicit: $P_{c}\to\infty$ as $R$ approaches $\frac{9}{4}M$ from above.

\subsection*{Interior time component and surface matching}
With $m(r)$ and $P(r)$ known, the remaining metric function is the redshift potential $e^{2\alpha(r)}$. 
Equation~\eqref{eq:alpha_prime} relates $\alpha'(r)$ to the matter variables; substituting the uniform–density mass profile \eqref{eq:const_density_mass} and the pressure \eqref{eq:const_density_pressure} yields a first–order equation that integrates in closed form. 
Up to an overall constant, the result is
\begin{equation}
e^{\alpha(r)}=C\,\bigl(3\,\Phi_{R}-\Phi(r)\bigr),
\label{eq:alpha_shape}
\end{equation}
where $\Phi(r)$ and $\Phi_{R}$ are defined in \eqref{eq:Phi_defs}. 
Continuity of the $tt$–component at the surface fixes $C$ via $e^{2\alpha(R)}=1-2M/R=\Phi_{R}^{2}$, hence $C=\tfrac{1}{2}$. 
Therefore
\begin{equation}
e^{2\alpha(r)}=\frac{1}{4}\,\bigl(3\,\Phi_{R}-\Phi(r)\bigr)^{2},
\label{eq:e2alpha_uniform}
\end{equation}
and the interior line element is
\begin{equation}
ds^{2}
=\frac{1}{4}\,\bigl(3\,\Phi_{R}-\Phi(r)\bigr)^{2}\,dt^{2}
-\left(1-\frac{2M}{R^{3}}\,r^{2}\right)^{-1}\!dr^{2}
-r^{2}\,d\Omega^{2},\qquad (r\le R).
\label{eq:interior_schwarzschild}
\end{equation}
At the center $r=0$ we have $\Phi(0)=1$, so $e^{\alpha(0)}=\tfrac{1}{2}(3\Phi_{R}-1)$ and $e^{2\beta(0)}=1$ are finite, and from \eqref{eq:alpha_prime} with $m(r)\sim \tfrac{4\pi}{3}\epsilon_{0}r^{3}$ it follows that $\alpha'(0)=0$. 
Thus the solution is regular at $r=0$. 
For $R>\frac{9}{4}M$ the metric functions remain smooth and finite throughout the interior; only the central pressure $P_{c}$ diverges as the Buchdahl limit $R\to \frac{9}{4}M$ is approached.

\subsection*{Summary and use}
The uniform--density model is analytically solvable but acausal (infinite $c_{s}$). It is useful as a pedagogical foil and as a numerical check: a TOV integrator supplied with \eqref{eq:const_density_assumption} should reproduce \eqref{eq:const_density_pressure} and \eqref{eq:interior_schwarzschild}, and should approach the compactness limit \eqref{eq:Buchdahl_bound} from below in $\mathcal{C}$ as the central pressure is increased.

\chapter{Ideal Fermi Gases}
\label{chap:fermi}

\section{Introduction}

The central microscopic mechanism that allows compact stars to resist gravitational collapse
is the Pauli exclusion principle, which forbids fermions from occupying the same quantum state.
Even in the absence of thermal motion, a collection of fermions must successively fill momentum
states, starting from the lowest energy and extending up to a sharp cutoff. This cutoff is
characterized by the \emph{Fermi momentum} $p_F$, the largest momentum occupied in the ground
state of the system, and the associated \emph{Fermi energy} $E_F$, the energy of the highest
occupied single--particle state. The resulting degeneracy pressure, which arises entirely from the
quantum mechanical exclusion principle rather than from thermal motion, provides the dominant
contribution to the internal support of dense stellar matter. 

The simplest theoretical description of such systems is the \emph{ideal, zero--temperature Fermi
gas}. In this model, fermions are treated as non--interacting particles that occupy all quantum
states inside a sphere of radius $p_F$ in momentum space, known as the \emph{Fermi surface}.
Despite its simplicity, the ideal Fermi gas captures the essential qualitative physics of compact
stars~\cite{pathria2011}. For instance, the equilibrium of white dwarfs can be described by the
degeneracy pressure of electrons. For lighter white dwarfs the electrons are mostly
non--relativistic, while in more massive white dwarfs the electrons at the Fermi surface become
relativistic. In this high--density regime the model predicts the existence of a maximum stable
mass, the Chandrasekhar limit~\cite{chandrasekhar1931,chandrasekhar1935,shapiro1983}. Likewise,
an \emph{ideal neutron gas} provided the basis of the pioneering Oppenheimer--Volkoff study of
neutron stars~\cite{oppenheimer1939}.

From the perspective of general relativity, the ideal Fermi gas plays a crucial role as it provides
an explicit equation of state, $P = P(\epsilon)$, which closes the Tolman--Oppenheimer--Volkoff
equations derived in the previous chapter. While real neutron star matter is far more complicated,
involving strong nuclear interactions, finite temperatures, and possibly exotic components, the ideal Fermi gas remains a natural starting point. It offers both
analytic insight in certain limits and a benchmark against which more sophisticated models can be
compared.

The purpose of this chapter is therefore twofold: first, to review the theoretical foundations of
an ideal Fermi gas at zero temperature, deriving expressions for number density, energy density,
pressure, and chemical potential; and second, to connect these results to the astrophysical
context of compact stars. We emphasize the non--relativistic and ultra--relativistic limits, where
the expressions simplify considerably and provide physical intuition. Finally, we will discuss how
the ideal Fermi gas equation of state can be coupled to the TOV equations to describe equilibrium
configurations of idealized neutron stars.

\section{Quantum statistics at zero temperature}

Fermions are particles with half--integer spin that obey the Pauli exclusion principle: 
no two fermions can occupy the same quantum state simultaneously. 
The statistical distribution that incorporates this principle is the Fermi--Dirac distribution,
\begin{equation}
f(E) = \frac{1}{e^{(E-\mu)/T}+1},
\label{eq:fermi_dirac_distribution}
\end{equation}
where $E$ is the single--particle energy, $\mu$ the chemical potential, 
and $T$ the temperature (in natural units $k_B=1$).  

In the zero--temperature limit $T\to 0$, the distribution becomes a sharp step function,
\begin{equation}
f(E) = \Theta(\mu - E),
\label{eq:zeroT_distribution}
\end{equation}
so that all states with $E<\mu$ are occupied and all states with $E>\mu$ are empty.  
At $T=0$, the chemical potential takes the value
\begin{equation}
E_F = \mu(T=0),
\end{equation}
which is called the \emph{Fermi energy}. It is the energy of the highest occupied 
single--particle state in the system.  

\begin{figure}[ht]
\centering
\includegraphics[width=0.65\textwidth]{figures/fermi_dirac_distribution.pdf}
\caption[Fermi--Dirac distribution at different temperatures]{
The Fermi--Dirac distribution $f(E)$ for several finite temperatures $T$ 
with chemical potential $\mu=1$. 
At $T=0$ the distribution reduces to a step function (dashed line), 
so that all states with $E<\mu$ are occupied and all higher states are empty. 
Increasing temperature smooths the step around the Fermi energy.}
\label{fig:fermi_dirac_distribution}
\end{figure}

From the special--relativistic dispersion relation
\begin{equation}
E(p) = \sqrt{p^2+m^2},
\label{eq:SR_dispersion}
\end{equation}
with $m$ the particle rest mass and $p=|\vec{p}|$ the momentum magnitude, one obtains at $T=0$ 
\begin{equation}
p_F = \sqrt{\mu^2 - m^2},
\end{equation}
since $\mu = E_F$ in this limit.  

Thus a degenerate Fermi gas is fully specified once the particle mass $m$ and the chemical potential $\mu$ are given. 
Equivalently, one can use the pair $(m, p_F)$, since $p_F$ and $\mu$ are related by the dispersion relation above. 
These quantities set the natural scales of the system and will serve as the basis for the thermodynamic description in the following sections.

\subsection*{Physical meaning of $E_F$ and $p_F$}
The Fermi energy $E_F$ measures the energy cost of adding one more particle to the system at zero
temperature, since all lower--energy states are already occupied. The Fermi momentum $p_F$ is
therefore the largest momentum carried by any fermion in the ground state. Together they define
the ``surface'' of occupied states in momentum space --- the Fermi surface --- which plays a central
role in both condensed matter and astrophysical systems. Even at $T=0$, fermions cannot all sit in
the lowest momentum state; instead, the exclusion principle forces them to occupy an entire sphere
of states up to $p_F$. This filling of states gives rise to degeneracy pressure, independent of
temperature.


\section{Thermodynamics of an ideal Fermi gas}
\label{sec:thermo_ideal_fermi}

Having reviewed the statistical foundations and the concepts of Fermi energy and Fermi
momentum, we now derive the thermodynamic quantities of a degenerate Fermi gas at zero
temperature by integrating over occupied momentum states up to $p_F$. The momentum--space
volume element is $d^3p=4\pi p^2\,dp$, and the density of states per unit volume is
$\tfrac{g}{(2\pi)^3}\,d^3p$, where $g=2s+1$ accounts for spin degeneracy. The following expressions are
standard and can be found in e.g. Shapiro \& Teukolsky or Carroll~\cite{shapiro1983,carroll}.

\subsection*{Number density}
The \emph{number density} $n$ denotes the number of particles per unit volume in the fluid
rest frame ($n \equiv N/V$). At zero temperature it is determined by how many
single--particle momentum states are filled up to the Fermi momentum $p_F$.
Because of the Pauli principle, each momentum state can be occupied by several fermions
that differ only in internal quantum numbers. For a particle of spin $s$, the available spin
projections $m_s=-s,-s+1,\dots,s$ give a total \emph{degeneracy}
\begin{equation}
g \;=\; 2s+1.
\end{equation}
Thus, each momentum state admits up to $g$ fermions. For spin-$\tfrac12$ particles
such as electrons, neutrons, or protons one has $g=2$.

Counting the total number of occupied states inside the Fermi sphere then gives
\begin{equation}
n \;=\; g \int \frac{d^3p}{(2\pi)^3}\,\Theta(p_F-|\vec p|)
\;=\; \frac{g}{2\pi^2}\int_0^{p_F}p^2\,dp
\;=\; \frac{g}{6\pi^2}\,p_F^{3}.
\label{eq:fermi_number_density}
\end{equation}
This fundamental relation fixes the Fermi momentum in terms of $n$,
\begin{equation}
p_F \;=\; \Big(\tfrac{6\pi^2}{g}\,n\Big)^{1/3},
\label{eq:pF_of_n}
\end{equation}
and connects the microscopic filling of momentum states to the macroscopic particle density.

\subsection*{Energy density}
The total energy density follows from summing the single--particle energies $E(p)=\sqrt{p^2+m^2}$ 
over all occupied momentum states. In terms of the general phase--space integral,
\begin{equation}
\epsilon \;=\; g \int \frac{d^3p}{(2\pi)^3}\,E(p)\,\Theta(p_F-|\vec p|)
\;=\; \frac{g}{2\pi^2}\int_0^{p_F}\sqrt{p^2+m^2}\,p^2\,dp.
\label{eq:energy_density_integral}
\end{equation}
To simplify the evaluation it is convenient to introduce the dimensionless variable $x=p/m$, 
with upper limit $x_F=p_F/m$, so that
\begin{equation}
\epsilon \;=\; \frac{g\,m^4}{2\pi^2}\int_0^{x_F}\!\sqrt{1+x^2}\,x^2\,dx.
\end{equation}
Carrying out the integral yields the compact analytic form
\begin{equation}
\epsilon \;=\; \frac{g\,m^4}{16\pi^2}
\Big[\,x_F\sqrt{1+x_F^2}\,(2x_F^2+1)\;-\;\sinh^{-1}x_F\,\Big].
\label{eq:energy_density}
\end{equation}
This expression makes explicit the crossover between the non--relativistic regime 
($x_F\ll 1$, where $\epsilon \approx m n$ with small kinetic corrections) 
and the ultra--relativistic regime ($x_F\gg 1$, where $\epsilon \propto p_F^4$).

\subsection*{Chemical potential}
The chemical potential measures the change in energy when adding one particle to the system, 
and at zero temperature it coincides with the Fermi energy. 
Intuitively, this is because the highest occupied single--particle state at $T=0$ 
sits exactly at the Fermi surface, so adding one more fermion forces it into that state. 
Formally, one finds
\begin{equation}
\mu \;=\; E_F \;=\; \sqrt{p_F^2+m^2}
\;=\; m\,\sqrt{1+x_F^2}
\;=\; \sqrt{\,m^2 + \Big(\tfrac{6\pi^2}{g}\,n\Big)^{2/3}\,},
\label{eq:mu_zeroT}
\end{equation}
where the last equality follows from the relation between Fermi momentum and number density, 
Eq.~\eqref{eq:pF_of_n}.

Two useful thermodynamic identities hold in the degenerate limit $T=0$: 
\begin{equation}
d\epsilon \;=\; \mu\,dn,
\qquad
\epsilon + P \;=\; \mu\,n.
\label{eq:zeroT_identities}
\end{equation}
The first expresses that the chemical potential is the derivative of the energy density 
with respect to particle number, while the second is the Gibbs--Duhem relation
with the entropy term $Ts$ vanishing at zero temperature. 
These relations are easily checked by differentiating the explicit integral expressions 
for $\epsilon$ and $n$: from \eqref{eq:energy_density_integral} and \eqref{eq:fermi_number_density} one finds
\[
\mu \;=\; \frac{d\epsilon}{dn} \;=\; \sqrt{p_F^2+m^2},
\]
which confirms Eq.~\eqref{eq:mu_zeroT}.

\subsection*{Pressure}
Microscopically, pressure can be defined as the flux of momentum carried by the particles across a unit area. 
In relativistic kinetic theory this notion is captured by the spatial diagonal components of the energy--momentum tensor,
\begin{equation}
T^{\mu\nu} \;=\; g \int \frac{d^3p}{(2\pi)^3}\,\frac{p^\mu p^\nu}{E(p)}\,f(p),
\end{equation}
with $E(p)=\sqrt{p^2+m^2}$ and $f(p)$ the occupation number. 
At zero temperature, the distribution is simply a step function $f(p)=\Theta(p_F-p)$, filling all states up to the Fermi momentum. 
The pressure is then given by the diagonal spatial component, for instance $T^{xx}$,
\begin{equation}
P \;=\; T^{xx} \;=\; g \int \frac{d^3p}{(2\pi)^3}\,\frac{p_x^2}{E(p)}\,f(p).
\end{equation}
Because the Fermi sphere is isotropic, the average over directions yields $\langle p_x^2 \rangle = \frac{1}{3}p^2$, so that
\begin{equation}
P \;=\; \frac{g}{3}\int \frac{d^3p}{(2\pi)^3}\,\frac{p^2}{E(p)}\,f(p).
\end{equation}
Switching to spherical momentum coordinates, $d^3p=4\pi p^2 dp$, gives the momentum--flux integral
\begin{equation}
P \;=\; \frac{g}{6\pi^2}\int_0^{p_F}\frac{p^4}{\sqrt{p^2+m^2}}\,dp
\;=\; \frac{g\,m^4}{6\pi^2}\int_0^{x_F}\frac{x^4}{\sqrt{1+x^2}}\,dx,
\label{eq:pressure_integral}
\end{equation}
which evaluates to
\begin{equation}
P \;=\; \frac{g\,m^4}{48\pi^2}
\Big[\,x_F\sqrt{1+x_F^2}\,(2x_F^2-3)\;+\;3\,\sinh^{-1}x_F\,\Big].
\label{eq:pressure}
\end{equation}

\noindent
Alternatively, the pressure can also be obtained from the thermodynamic relation
\begin{equation}
P \;=\; -\,\epsilon \;+\; \mu n,
\end{equation}
which follows from the Gibbs--Duhem relation introduced in the previous section. 
This provides a useful check on the explicit momentum--flux calculation above.

\section{Equation of State and Limiting Cases}

The expressions derived above provide $n$, $\epsilon$ and $P$ as explicit functions of the
Fermi momentum $p_F$ (or equivalently the dimensionless ratio $x_F=p_F/m$). In practice,
this means that the ideal Fermi gas equation of state is most conveniently written in
\emph{parametric form}:
\begin{align}
\epsilon &= \epsilon(x_F), \\
P &= P(x_F),
\end{align}
with $\epsilon(x_F)$ and $P(x_F)$ given by
Eqs.~\eqref{eq:energy_density} and \eqref{eq:pressure}. The chemical potential
is simultaneously determined by Eq.~\eqref{eq:mu_zeroT}. Eliminating $x_F$ gives a
closed functional relation $P(\epsilon)$, but the analytic form is not especially
transparent; in astrophysical applications one typically evaluates the parametric
expressions numerically.

Nevertheless, in the two limiting regimes $p_F \ll m$ (non--relativistic) and
$p_F \gg m$ (ultra--relativistic), the integrals admit simple expansions that are
physically illuminating.

\subsection*{Non--relativistic limit}
In the non--relativistic regime $x_F=p_F/m\ll 1$, single--particle energies are
$E(p)\simeq m+\frac{p^2}{2m}$. The pressure and number density at $T=0$ are
\begin{equation}
P \;\simeq\; \frac{g}{30\pi^2\,m}\,p_F^5,
\qquad
n \;=\; \frac{g}{6\pi^2}\,p_F^3.
\label{eq:NR_p_and_n}
\end{equation}
In this limit the \emph{total} energy density is dominated by rest mass,
\begin{equation}
\epsilon \;\simeq\; m\,n,
\label{eq:NR_eps_total}
\end{equation}
with a subleading kinetic piece (of order $p_F^5/m$) that we neglect when relating
$\epsilon$ to $P$. Eliminating $p_F$ between \eqref{eq:NR_p_and_n} gives
\[
p_F^5 \;=\; \frac{30\pi^2\,m}{g}\,P,
\qquad
p_F^3 \;=\; \biggl(\frac{30\pi^2\,m}{g}\,P\biggr)^{\!3/5}.
\]
Hence
\begin{align}
n \;&=\; \frac{g}{6\pi^2}\,p_F^3
\;=\; \frac{g}{6\pi^2}\,\biggl(\frac{30\pi^2\,m}{g}\,P\biggr)^{\!3/5},
\\[2pt]
\epsilon(P) \;&=\; m\,n
\;=\; \Bigl(\tfrac{30^{3/5}}{6}\,\pi^{-4/5}\Bigr)\,
g^{2/5}\,m^{8/5}\,P^{3/5}.
\end{align}
Thus, in the non--relativistic limit the equation of state exhibits the polytropic scaling
\begin{equation}
\epsilon \;\propto\; P^{3/5}.
\end{equation}
The term ``polytropic'' refers to an equation of state of the form 
$P = K \rho^\gamma$, where $\rho$ is the mass density. 
At zero temperature in the non--relativistic regime the energy density is dominated 
by rest mass, $\epsilon \simeq m n \simeq \rho$, so that the relation above is 
equivalent to $P \propto \rho^{5/3}$. The effective polytropic index is therefore 
$\gamma = 5/3$ for a non--relativistic degenerate Fermi gas.

\subsection*{Ultra--relativistic limit}
For large Fermi momenta, $x_F \gg 1$, the rest mass becomes negligible and
$E(p) \approx p$. The integrals then yield
\begin{align}
\epsilon &\;\approx\; \frac{g}{8\pi^2}\,p_F^4, \\
P &\;\approx\; \frac{g}{24\pi^2}\,p_F^4,
\end{align}
so that the equation of state approaches
\begin{equation}
\epsilon \;=\; 3\, P.
\label{eq:ultrarel_eos}
\end{equation}
This is identical to the relation for a gas of massless particles such as
photons or neutrinos, consistent with the fact that for $P \gg m$ the particle
dispersion relation becomes effectively massless.

\subsection*{Physical interpretation}
The two limiting behaviors highlight the changing \emph{stiffness} of the Fermi gas
equation of state, i.e.\ how strongly the pressure responds to increasing energy density. 
In the non--relativistic regime, the pressure scales as $n^{5/3}$, which rises faster 
than linearly with density but remains subdominant to the rest--mass energy in $\epsilon$. 
In contrast, in the ultra--relativistic regime the pressure is a fixed fraction ($1/3$) 
of the energy density, producing a much softer relation between $P$ and $\epsilon$. 
The transition between these two regimes occurs when $p_F \sim m$, i.e.\ when
the typical fermion momentum becomes comparable to the rest mass. 
In compact stars, electrons often approach or enter the relativistic regime at 
sufficiently high densities, while baryons such as neutrons may remain largely 
non--relativistic in lighter stars but become increasingly relativistic as central 
densities grow. In all cases, the degeneracy pressure encapsulated by the Fermi gas 
equation of state provides the fundamental microscopic mechanism preventing 
gravitational collapse.

\begin{figure}[ht]
\centering
\includegraphics[width=0.7\textwidth]{figures/fermi_eos.pdf}
\caption[Stiffness of the Fermi gas equation of state]{
Ratio of pressure to energy density $p/\epsilon$ as a function of the
dimensionless Fermi momentum $x_F = p_F/m$. 
In the non--relativistic regime $x_F \ll 1$, rest--mass energy dominates
and $P/\epsilon \to 0$. 
In the ultra--relativistic regime $x_F \gg 1$, the relation approaches
$P = \epsilon/3$, indicated by the dashed line.
}
\label{fig:fermi_eos}
\end{figure}

\section{Astrophysical Relevance}

The ideal Fermi gas model, while highly simplified, captures the essential
microscopic mechanism that allows compact stars to resist gravitational
collapse: degeneracy pressure. Its relevance can be illustrated in two
canonical astrophysical contexts.

\paragraph{White dwarfs.}
In white dwarfs, the pressure support is provided almost entirely by
degenerate electrons. As the central density increases, the electron
Fermi momentum can approach or exceed the rest mass, $p_F \gtrsim m_e$,
driving the system toward the relativistic regime. In this limit the
equation of state approaches $P = \epsilon/3$, and the stellar mass
becomes essentially independent of radius. Balancing this relativistic
degeneracy pressure against gravity yields the Chandrasekhar mass limit,
$M_{\rm Ch} \simeq 1.4\,M_\odot$. This result follows under the
idealized assumptions of a cold, non--interacting electron gas providing
all of the pressure support, with ions contributing only to the rest--mass
density, and with the electrons treated in the ultra--relativistic
limit. Despite these simplifications, the Chandrasekhar limit gives a
value close to the maximum masses of observed white dwarfs, making the
ideal Fermi gas a useful model for this calculation.

\paragraph{Neutron stars.}
In their pioneering work, Oppenheimer and Volkoff modeled neutron stars
as a cold, ideal Fermi gas of neutrons with no nuclear interactions,
described by the full relativistic dispersion relation
$E(p)=\sqrt{p^2+m_n^2}$. Solving the TOV equations with this equation of
state, they found a maximum stable mass of order $0.7\,M_\odot$. Although
this value is far below the $\sim 2\,M_\odot$ neutron stars observed
today~\cite{haensel2007}, the calculation was historically crucial: it
demonstrated that combining general relativity with a microscopic
equation of state leads to an upper mass limit for neutron stars,
analogous to Chandrasekhar’s result for white dwarfs.

\paragraph{Limitations.}
The ideal Fermi gas neglects several important physical ingredients. Real
neutron star matter is strongly interacting, with short--range nuclear
forces significantly modifying the pressure at supranuclear densities.
Charge neutrality and beta equilibrium imply a mixture of neutrons,
protons, electrons, and muons, rather than a pure single--species Fermi
gas. At sufficiently high densities, more exotic components may appear.
Moreover, finite temperature effects can be relevant in newly born
protoneutron stars. All of these factors alter the equation of state,
and hence the predicted stellar structure.

\paragraph{Summary.}
Despite its limitations, the ideal Fermi gas remains a valuable baseline
model. It provides analytic control in limiting regimes, a clear physical
picture of degeneracy pressure, and a simple parametric equation of state
that can be directly coupled to the Tolman--Oppenheimer--Volkoff equations.
As such, it serves as a pedagogical introduction to compact star physics,
and its predictions---such as the Chandrasekhar mass for white dwarfs and
the Oppenheimer--Volkoff mass limit for neutron stars---show that even this
simplified treatment can reproduce the correct order of magnitude for the
maximum masses of compact stars.

\chapter{Ideal Neutron Stars}
\label{chap:ideal_neutron_stars}

\section{Introduction}

The concept of an \emph{ideal neutron star} provides a minimal
theoretical model for a relativistic compact star.
Here the term ``ideal'' refers to the use of the simplest, non--interacting
Fermi gas equation of state, representing matter at zero temperature.
The composition is taken to be pure neutrons for simplicity,
in analogy with the classic white dwarf model where support against
gravity arises from the electron degeneracy pressure.
The key distinction is that neutron stars reach densities so high
that the gravitational field itself becomes extremely strong,
requiring a treatment within general relativity.
The appropriate framework is therefore given by the TOV equations
introduced in Chapter~\ref{chap:tov}.

The aim of this chapter is to construct and analyze such ideal neutron
star models, with the focus on the numerical implementation and results.

\section{Numerical Setup}

The TOV equations derived in Chapter~\ref{chap:tov} describe the
radial dependence of the pressure $P(r)$ and enclosed mass $m(r)$ in a
static, spherically symmetric star. To integrate these equations we
must specify the boundary conditions. They are
\[
m(0)=0, \qquad P(0)=P_c, \qquad P(R)=0,
\]
where $P_c$ is the central pressure and $R$ is the stellar radius at
which the pressure vanishes. The corresponding gravitational mass is
then $M=m(R)$.

In Chapter~\ref{chap:fermi} we obtained the EOS for a degenerate,
zero--temperature Fermi gas. Specializing to neutrons of mass $m_N$,
the EOS can be expressed in parametric form
\[
P = P(x_F), \qquad \epsilon = \epsilon(x_F),
\]
with $x_F=p_F/m_N$ the dimensionless Fermi momentum.
Although this parametric representation is convenient analytically,
the numerical solver expects $\epsilon(P)$. We therefore invert
$P(x_F)$ numerically to obtain $x_F(P)$, and then evaluate
$\epsilon(x_F)$ accordingly. This guarantees a consistent mapping
between $P$ and $\epsilon$ across all density regimes.

For numerical stability it is advantageous to work with dimensionless
variables. We introduce reference scales for length, mass, and energy
density,
\[
r_0 = \frac{1}{\sqrt{G \epsilon_0}}, \qquad
m_0 = \frac{1}{\sqrt{G^3 \epsilon_0}}, \qquad
\epsilon_0 = m_N^4.
\]
All quantities are then expressed in units of $(r_0, m_0, \epsilon_0)$,
so that
\[
\hat r = r/r_0, \qquad
\hat m = m/m_0, \qquad
\hat P = P/\epsilon_0, \qquad
\hat \epsilon = \epsilon/\epsilon_0.
\]
In terms of these variables the TOV system takes a dimensionless form
suitable for numerical integration. This rescaling avoids very large
or small floating--point numbers and improves the stability of the
Runge--Kutta solver.

In summary, the numerical setup consists of:
\begin{enumerate}
  \item The TOV equations, integrated outward from the center with 
  initial conditions $m(0)=0$ and $P(0)=P_c$, and terminated when 
  the surface condition $P(R)=0$ is reached. The corresponding 
  radius $R$ and gravitational mass $M=m(R)$ then characterize 
  the macroscopic quantities of the star.
  \item The ideal neutron gas EOS $\epsilon(P)$, obtained from the
  parametric Fermi gas expressions of Chapter~\ref{chap:fermi}.
  \item Dimensionless variables to reduce numerical instabilities.
\end{enumerate}
Together these ingredients define a well--posed system suitable for
numerical solution.

\section{Numerical Method}

The numerical integration of the TOV equations with an ideal neutron
gas EOS is carried out using a Python program originally developed by
Sletmoen as part of his Master's thesis~\cite{sletmoen2022}.
In this work the code is used without modification to generate stellar
models for the case of ideal neutron matter.

The implementation proceeds as follows:
\begin{itemize}
  \item The coupled ODEs for $P(r)$ and $m(r)$ are integrated outward
  from $r=0$ with initial values $P(0)=P_c$ and $m(0)=0$.
  \item A fifth--order Runge--Kutta method with adaptive step size is used to
  advance the solution.
  \item The integration terminates once $P(r)$ vanishes, which defines
  the stellar radius $R$. The corresponding $m(R)$ gives the
  gravitational mass $M$.
\end{itemize}
Repeating the calculation for a range of central pressures $P_c$
generates a sequence of stellar configurations. From these solutions
one can extract the macroscopic properties of the star, namely its mass
$M=m(R)$ and radius $R$ where $P(R)=0$. These results can then be
combined into the mass--radius relation. In addition, the solver also
provides access to the radial structure, meaning the profiles
of pressure $P(r)$ and energy density $\epsilon(r)$ throughout the
stellar interior. Such results reveal how matter is distributed
and how pressure balances gravity at each radius, complementing the
macroscopic description. In the following section we present both
macroscopic and structural results for ideal neutron stars.
These results can be directly compared with the analytic behavior found
for the uniform--density star in Chapter~\ref{chap:tov}.

\section{Results}
\subsection*{Mass--radius relation}

Figure~\ref{fig:mr_neutron} shows the resulting mass--radius curves for
ideal neutron stars.  The four models represent combinations of
Newtonian and relativistic hydrostatic equilibrium with either the
non--relativistic or the full relativistic form of the Fermi gas EOS.
Each point corresponds to a distinct central pressure $P_c$.

At low central pressures the curves coincide, reflecting the Newtonian
limit where gravitational fields are weak and the EOS is nearly
non--relativistic. As $P_c$ increases, however, the relativistic models
deviate markedly: the curves flatten and eventually exhibit a turnover,
indicating that beyond a certain maximum mass, no further stable
solutions exist. This behavior is analogous to the instability already 
encountered in the incompressible model, where the
pressure diverges at the Buchdahl bound $2M/R=8/9$.  In an ideal, non--interacting
neutron gas, the same effect manifests itself as the \emph{Oppenheimer--Volkoff
limit}~\cite{oppenheimer1939}: a maximum gravitational mass of order
$M_{\mathrm{max}}\!\approx\!0.7\,M_{\odot}$ for an ideal, non--interacting
neutron fluid.  Beyond this point, increasing the central pressure only
reduces the radius without adding mass, and the configuration becomes
unstable to collapse.

The Newtonian solutions, in contrast, show no turnover.  They predict
monotonically increasing mass with increasing central pressure, an
artifact of neglecting relativistic corrections to both gravity and the
EOS.  The fully relativistic treatment thus captures a key qualitative
feature: the existence of a maximum stable mass set by the balance
between degeneracy pressure and spacetime curvature.

\begin{figure}[ht]
  \centering
  \includegraphics[width=0.8\textwidth]{figures/mass_radius_ideal_neutron_star.pdf}
  \caption[Mass--radius relation for ideal neutron stars]{
  Mass--radius relation for ideal neutron stars. Four models are compared:
  Newtonian and relativistic hydrostatic equilibrium, each with a
  non--relativistic or fully relativistic Fermi gas EOS.  The color scale
  indicates the central pressure $P_c$.  The relativistic models exhibit a
  maximum mass, the Oppenheimer--Volkoff limit, beyond which no static
  equilibrium is possible.}
  \label{fig:mr_neutron}
\end{figure}

\subsection*{Pressure profiles}

Figure~\ref{fig:profiles_neutron} displays the normalized pressure
profiles $P(r)/P_c$ for a range of central pressures.
For small $P_c$, the pressure decreases almost linearly with radius,
resembling the Newtonian incompressible case.
As $P_c$ increases, the curves become increasingly concentrated toward
the center, revealing stronger gravitational compression and greater
compactness.  The outer layers contribute less to the total mass, while
the interior supports an ever larger fraction of the gravitational
weight.  This mirrors the analytic results for the constant--density
star, where high compactness led to steep gradients and eventual
divergence at the Buchdahl bound.

At the highest central pressures, the pressure gradient near the center
becomes extremely steep, signaling the proximity of the
Oppenheimer--Volkoff limit.  Beyond this limit, no equilibrium solution
exists: the star must collapse into a black hole.  Thus, the numerical
pressure profiles confirm the same qualitative physics inferred from the
analytic model—namely, that gravity strengthens faster than the pressure
support as compactness increases.

\begin{figure}[ht]
  \centering
  \includegraphics[width=0.8\textwidth]{figures/pressure_profile_ideal_neutron_star.pdf}
  \caption[Pressure profiles of ideal neutron stars]{
  Radial pressure profiles $P(r)/P_c$ for ideal neutron stars with
  different central pressures $P_c$.  The color scale represents
  $\log_{10}(P_c)$.  Increasing compactness leads to steeper gradients and
  a stronger concentration of pressure toward the center, consistent with
  the approach to the Oppenheimer--Volkoff limit.}
  \label{fig:profiles_neutron}
\end{figure}

\section{Radial stability analysis}
\label{sec:ideal_ns_stability}

\subsection{Physical motivation}

The mass--radius curves in Fig.~\ref{fig:mr_neutron} show that the
relativistic models reach a maximum mass and then turn over.
This turning point marks a change in stability. 
To understand why this happens, we study how small radial
perturbations behave in a star that is initially in hydrostatic
equilibrium.

If the star is slightly compressed, gravity and pressure respond in
opposite directions: gravity pulls matter inward while pressure pushes
it outward. 
If the pressure increase is strong enough to balance the added gravity,
the star will oscillate around its equilibrium shape and remain stable.
If gravity grows faster than the pressure response, the perturbation
will instead amplify with time and lead to collapse.

\subsection{Linearized equations}

A convenient way to study the stability of a relativistic star is to
examine how it responds to small, time--dependent, radial perturbations.
The stability problem was first solved in detail by
Chandrasekhar~\cite{chandrasekhar1964}, who derived the full set of
relativistic perturbation equations for a fluid sphere.
In this work we follow the later formulation of
Tooper~\cite{tooper1965}, in which the equations are rewritten in a form that is
more convenient for numerical integration and directly leads to the
Sturm--Liouville equation for the radial modes.

In this approach, one introduces a small radial displacement
$\xi(r,t)$ of the fluid layers and linearizes the Einstein and fluid
equations to first order in the perturbation amplitude.
The resulting system can be combined into a single
second--order differential equation for a variable $U(r,t)$ that is
proportional to $\xi(r,t)$.
Assuming harmonic time dependence $U(r,t)=U(r)\,e^{i\omega t}$ gives
\begin{equation}
\frac{d}{dr}\!\left[\Pi(r)\frac{dU}{dr}\right]
+ Q(r)\,U(r) + \omega^2 W(r)\,U(r) = 0.
\label{eq:SL_ideal_NS}
\end{equation}
The quantity $\omega^2$ is the squared oscillation frequency.
The coefficients $\Pi(r)$, $Q(r)$, and $W(r)$ depend only on the
background stellar structure, that is, on $P(r)$, $\epsilon(r)$,
$m(r)$, and the metric potential $\alpha(r)$ obtained from the TOV
equations.
They also include the adiabatic index
\[
\Gamma_1(r) = \frac{\epsilon(r)+P(r)}{P(r)}
\frac{dP}{d\epsilon}\bigg|_{r},
\]
which describes the local stiffness of the equation of state.

Equation \eqref{eq:SL_ideal_NS} has the standard form of a \emph{Sturm--Liouville eigenvalue
problem}. This is a general class of second--order linear
differential equations with specific boundary conditions that admit a
discrete set of eigenvalues and eigenfunctions.
Each allowed solution $U_n(r)$ corresponds to an \emph{eigenmode}, and
the associated constant $\omega_n^2$ is its \emph{eigenvalue}.
In the context of stellar oscillations, $\omega_n^2$ represents the
squared frequency of the $n$th radial mode.
If all eigenvalues are positive, the star performs stable oscillations.
When the lowest eigenvalue $\omega_0^2$ becomes zero or negative, the
oscillation changes into a growing mode, and the equilibrium becomes
dynamically unstable~\cite{arfken2013}.

Regularity at the center requires that the eigenfunction vanish at least
as $U(r)\!\sim\!r^{3}$ for $r\!\to\!0$, ensuring that the 
displacement $\xi = U\,e^{-\alpha}/r^{2}$ remains finite.  The cubic
form is the lowest--order regular behavior and is therefore used as the
boundary condition at $r = 0$.  At the surface $r = R$, the pressure
must vanish smoothly, giving the boundary condition $\Delta P(R) = 0$
for the Lagrangian pressure perturbation, which represents the total
change in pressure experienced by a moving fluid element,
$\Delta P = \delta P + \xi\,dP/dr$.  With these two boundary conditions,
the problem is well defined and can be solved for the eigenvalues
$\omega^{2}$.

\subsection{Numerical method}

Equation~\eqref{eq:SL_ideal_NS} forms an eigenvalue problem for the
frequency $\omega^2$. For each stellar model obtained from the TOV
equations, we can calculate the background functions
$P(r)$, $\epsilon(r)$, $m(r)$, and $\alpha(r)$, and from them construct
the coefficients $\Pi(r)$, $Q(r)$, and $W(r)$.
These three functions contain all the information about the background
structure and determine how the star responds to small radial perturbations.
They are given by~\cite{tooper1965}
\[
\Pi = e^{\beta+3\alpha}\frac{\Gamma_1 P}{r^2}, \qquad
Q = e^{\beta+3\alpha}\!\left(
  \frac{dP}{dr}\frac{d\alpha}{dr}
  - \frac{4}{r}\frac{dP}{dr}
  - 8\pi G e^{2\beta} P(\epsilon + P)
\right), \qquad
W = e^{3\beta+\alpha}\frac{\epsilon+P}{r^2}.
\]
where $\alpha(r)$ and $\beta(r)$ are the metric functions,
$\Gamma_1$ is the adiabatic index defined in the previous section, and
$P(r)$ and $\epsilon(r)$ are determined by the equation of state.
The equation is then fully specified and can be solved for the allowed
values of $\omega^2$.

In practice, this is done using a \emph{shooting method}.
We start by choosing a trial value of $\omega^2$ and integrate the
differential equation outward from the center using the regularity
condition $U(r)\propto r^3$ near $r=0$.
The integration proceeds until the surface $r=R$ is reached, where the
Lagrangian pressure perturbation must vanish, $\Delta P(R)=0$.
For most trial values, this boundary condition will not be satisfied:
the solution may diverge or exhibit an incorrect number of nodes,
indicating that the guessed eigenvalue $\omega^2$ is not an allowed one.

To find the correct eigenvalue, $\omega^2$ is adjusted and the
integration is repeated.
The number of nodes (zero crossings) of $U(r)$ provides a useful guide:
the fundamental mode has no internal nodes, the first overtone has one
node, and so on.
By searching for the value of $\omega^2$ that produces the correct
number of nodes and satisfies the surface condition, we obtain the
eigenvalue of that mode.
This process is repeated for several stellar models with different
central pressures $P_c$.

In the implementation used here, the coefficients
$\Pi(r)$, $Q(r)$, and $W(r)$ are evaluated numerically from the TOV
profiles, and the equation is integrated using a finite--difference
scheme with adaptive step size.
A bisection search is used to locate the eigenvalue $\omega^2$ to the
desired accuracy.
The full algorithm is based on the code developed by
Sletmoen~\cite{sletmoen2022}, used to analyze the stability of the ideal neutron star
sequence.

\subsection{Results and stability criterion}

The numerical integration of Eq.~\eqref{eq:SL_ideal_NS} yields the
discrete eigenvalues $\omega_n^2$ and eigenfunctions $U_n(r)$ for each
stellar model in the sequence.
The sign of the lowest eigenvalue determines the dynamical stability:
\[
\omega_0^2 > 0 \;\Rightarrow\; \text{stable}, \qquad
\omega_0^2 = 0 \;\Rightarrow\; \text{marginally stable}, \qquad
\omega_0^2 < 0 \;\Rightarrow\; \text{unstable}.
\]

Figure~\ref{fig:shooting_convergence} shows typical normalized mode
functions $U_n(r)$ obtained with the shooting method.
Each function satisfies the regularity condition $U_n(r)\propto r^3$
near the center and the vanishing Lagrangian pressure perturbation at
the surface.
The number of interior nodes increases with mode index $n$, as expected
from Sturm--Liouville theory: the fundamental mode $U_0(r)$ has no
nodes, $U_1(r)$ one node, and so on.
This confirms that the numerical integration behaves as required and
that the node theorem is fulfilled.
The smooth convergence of the solutions toward the correct boundary
condition at $r=R$ demonstrates the reliability of the shooting
algorithm.

\begin{figure}[ht]
  \centering
  \includegraphics[width=0.8\textwidth]{figures/shoot_shoot_relative.pdf}
  \caption[Shooting method convergence]{
  Shooting method convergence for the first few radial modes. Each line
  corresponds to a trial eigenfunction $U_n(r)$ normalized by
  $\max|U_0|$. The smooth approach to the boundary condition
  $\Delta P(R)=0$ illustrates successful eigenvalue convergence.}
  \label{fig:shooting_convergence}
\end{figure}

The corresponding normalized eigenfunctions are displayed in
Fig.~\ref{fig:nmodes_norm}.
They form a complete set of orthogonal modes describing the radial
oscillations of the star.
The fundamental mode is positive everywhere and represents a coherent
breathing motion of the entire star, while higher overtones exhibit
additional internal nodes associated with local compressions and
rarefactions.

\begin{figure}[ht]
  \centering
  \includegraphics[width=0.8\textwidth]{figures/nmodes_norm.pdf}
  \caption[Normalized eigenfunctions]{
  Normalized eigenfunctions $U_n(r)$ for the lowest radial modes. The
  number of interior nodes increases with $n$, as expected for a
  Sturm--Liouville spectrum.}
  \label{fig:nmodes_norm}
\end{figure}

The eigenfrequencies $\omega_n^2$ vary systematically along the TOV
sequence.
As the central pressure $P_c$ increases, the stellar mass grows until it
reaches a maximum at the turning point of the mass--radius curve.
At this point, the fundamental frequency satisfies $\omega_0^2 = 0$,
indicating marginal stability.
Configurations on the ascending branch ($dM/dP_c > 0$) have
$\omega_0^2 > 0$ and are dynamically stable; beyond the maximum
($dM/dP_c < 0$), $\omega_0^2$ becomes negative, signalling instability
against radial collapse.
This behaviour constitutes the general relativistic stability criterion
for spherical stars.
The equivalence between the turning point condition $dM/dP_c = 0$
and the vanishing of the fundamental frequency $\omega_0^2 = 0$
was first demonstrated by Chandrasekhar~\cite{chandrasekhar1964}
using a variational principle for radial oscillations.
In this formalism, the total energy of the configuration is stationary
at equilibrium, and the sign of the second variation determines
stability. The change of sign in $\omega_0^2$ at the mass maximum
therefore coincides with the point where the equilibrium sequence
passes from stable to unstable configurations.


Physically, the transition arises because general relativity
strengthens the effective gravitational attraction compared to
Newtonian theory.
At low compactness, the pressure response of the Fermi gas remains
sufficient to counteract gravity, and perturbations lead to stable
oscillations.
As the central density increases, relativistic corrections make gravity
grow more rapidly than pressure, reducing the effective restoring force.
Once the mass maximum is reached, any further compression increases the
gravitational binding faster than the pressure can respond, causing
small perturbations to amplify.
The star thus becomes dynamically unstable and collapses toward a more
compact configuration—typically a black hole in realistic equations of
state.

In summary, the numerical results verify the expected behaviour:
\begin{itemize}
  \item The fundamental mode frequency $\omega_0^2$ changes sign exactly
  at the maximum of the mass--radius curve.
  \item The node structure of $U_n(r)$ confirms the correct mode
  ordering.
  \item The shooting solutions converge smoothly and respect all boundary
  conditions.
\end{itemize}
This establishes that, for the ideal neutron star, the
Oppenheimer--Volkoff limit corresponds precisely to the onset of
dynamical instability against radial perturbations.

\subsection*{Discussion}

These results illustrate how the inclusion of relativistic gravity and a
degenerate Fermi gas EOS naturally produces a maximum stable mass for
neutron stars.  The simple ideal neutron model predicts a limiting mass
of only $M_{\mathrm{max}}\!\sim\!0.7\,M_{\odot}$, known as the
Oppenheimer--Volkoff limit~\cite{oppenheimer1939}.  In reality,
observations show several neutron stars with masses above
$2\,M_{\odot}$~\cite{linares2018,strader2019,linares2020,romani2022}.
The difference arises because the ideal model assumes non--interacting
neutrons, while real neutron matter is affected by strong nuclear forces
that make the material stiffer and able to support more mass.  Still,
the ideal neutron star captures the essential qualitative behavior:
general relativity imposes an upper bound on the stable mass, just as
the incompressible model indicated through the Buchdahl limit.  The
Oppenheimer--Volkoff limit therefore remains the simplest physical
expression of this relativistic instability.

\chapter{The \texorpdfstring{$npe\mu$}{npemu} model}
\label{chap:npe_mu}

\section{Introduction}

In the previous chapters we developed and tested the structure equations for
neutron stars using simple models of matter. These models were useful for
building intuition about how gravity and pressure balance to form stable
configurations, and for exploring how different equations of state affect the
stellar mass and radius. However, at densities near and above the nuclear
saturation point, the assumptions of non–interacting particles become too
simple. The internal structure of nucleons and their mutual interactions start
to play a major role in determining the properties of matter.

To describe these effects in a practical way, we now turn to the relativistic
mean–field (RMF) approximation. This framework models dense nuclear matter through
average fields that represent the main attractive and repulsive parts of the
nuclear force \cite{glendenning2000,pogliano2017}. 
It provides a way to include essential interaction physics while
keeping the formulation simple enough for use in stellar calculations.

In this chapter we use the RMF approximation to describe matter composed of neutrons,
protons, electrons, and muons ($npe\mu$). By imposing charge neutrality and
$\beta$–equilibrium, this composition forms a minimal realistic model of a cold
neutron–star core. By studying existing RMF equations of state for such matter,
we can examine how nuclear interactions influence the pressure and density
profiles and how this, in turn, affects the mass–radius relation obtained from
the structure equations.

\section{Bulk Properties of Nuclear Matter}
\label{sec:bulk_properties}

A useful way to test any microscopic model of nuclear interactions is to see
whether it can reproduce the known bulk properties of symmetric nuclear matter.
These are macroscopic quantities that characterize infinite, uniform matter
composed of equal numbers of neutrons and protons, in its most stable (saturated)
state \cite{glendenning2000}. They summarize how the nuclear force behaves on average and provide
empirical reference points for any equation of state.

The most important of these bulk properties are:
\begin{itemize}
  \item the \textbf{binding energy per nucleon} $B/A$, which measures how tightly the
        nucleons are bound together at equilibrium;
  \item the \textbf{saturation density} $\rho_{0}$, i.e.\ the baryon number density at which
        the energy per particle reaches its minimum;
  \item the \textbf{effective nucleon mass} $m^{*}$, which reflects how the scalar field
        modifies the nucleon rest mass in the medium;
  \item the \textbf{compression modulus} $K$, which quantifies how stiff or soft the
        matter is against uniform compression;
  \item the \textbf{symmetry energy coefficient} $a_{\mathrm{sym}}$, which describes how
        the energy increases when the number of neutrons and protons becomes unequal.
\end{itemize}

The simplest RMF description of such matter is the
\emph{$\sigma$–$\omega$ model} \cite{glendenning2000,pogliano2017}.
In this approach, the strong nuclear interaction is
represented by the exchange of two meson fields: an attractive scalar field
($\sigma$) and a repulsive vector field ($\boldsymbol{\omega}$). The scalar attraction binds the
nucleons, while the vector repulsion prevents the system from collapsing at high
density. The balance between these two effects produces saturation naturally, in
contrast to non–relativistic models where it must be inserted by hand.

The Lagrangian density of the linear $\sigma$–$\omega$ model is
\begin{align}
\mathcal{L}
&= \bar{\psi}\bigl[\gamma_{\mu}(i\partial^{\mu} - g_{\omega}\omega^{\mu})
      - (m - g_{\sigma}\sigma)\bigr]\psi
   + \tfrac{1}{2}(\partial_{\mu}\sigma\,\partial^{\mu}\sigma - m_{\sigma}^{2}\sigma^{2})
   - \tfrac{1}{4}\omega_{\mu\nu}\omega^{\mu\nu}
   + \tfrac{1}{2}m_{\omega}^{2}\omega_{\mu}\omega^{\mu},
\label{eq:lagrangian_sigmaomega}
\end{align}
where $\psi$ is the nucleon field, $\omega_{\mu\nu}=\partial_{\mu}\omega_{\nu}-\partial_{\nu}\omega_{\mu}$
is the field tensor of the vector meson, and $m$, $m_{\sigma}$, $m_{\omega}$ are the
nucleon and meson masses, respectively.

From Eq.~\eqref{eq:lagrangian_sigmaomega} it is evident that the interaction
between nucleons and the meson fields is governed solely by the coupling
constants $g_{\sigma}$ and $g_{\omega}$. Since there is one scalar and one vector
interaction channel, the theory contains only two independent coupling
strengths, which appear in the dimensionless ratios $g_{\sigma}/m_{\sigma}$ and
$g_{\omega}/m_{\omega}$. Consequently, only two bulk nuclear observables—typically the
binding energy per nucleon and the saturation density—can be fitted
independently. The other quantities then follow automatically as predictions
from the model, but the resulting values are not realistic. In particular, the
predicted compression modulus $K$ is much too large, typically around
$K \simeq 550\,\mathrm{MeV}$, whereas empirical analyses of nuclear resonances give
$K \approx 230\,\mathrm{MeV}$ \cite{glendenning2000}. The effective mass and
symmetry energy are also found to deviate significantly from experiment.

To correct these deficiencies, the model is extended in two main ways.
First, one introduces non–linear self–interactions of the scalar field
$\sigma$, adding cubic and quartic terms to the Lagrangian.
These modify the density dependence of the attractive potential and thereby
soften the equation of state, bringing the compression modulus into agreement
with empirical data.
Second, one includes an additional isovector meson field ($\boldsymbol{\rho}$), which
couples to the difference between neutron and proton densities.
The $\rho$ meson restores isospin symmetry and governs how the energy changes
with neutron–proton asymmetry, allowing the model to reproduce the correct
symmetry energy.

With these improvements, the model can be tuned to
reproduce all five key bulk properties listed above.
Once calibrated in this way, it provides a realistic and internally consistent
equation of state that can be used to describe the dense matter inside neutron–star
cores.

At this point the model can be generalized from symmetric nuclear matter to
neutron–rich matter, as found in the interiors of neutron stars. In such
environments, weak interactions continuously convert neutrons into protons and
leptons until the system reaches $\beta$–equilibrium, where
\[
\mu_{n} = \mu_{p} + \mu_{e} = \mu_{p} + \mu_{\mu}.
\]
To maintain overall charge neutrality, the total positive charge of protons must
be balanced by the negative charge of the leptons,
\[
n_{p} = n_{e} + n_{\mu}.
\]
Electrons and muons are therefore included as free, relativistic Fermi gases,
while the baryons (neutrons and protons) continue to interact through the
exchange of the $\sigma$, $\omega$, and $\rho$ mesons introduced above.

The resulting composition—neutrons ($n$), protons ($p$), electrons ($e$), and
muons ($\mu$)—defines the so–called \emph{$npe\mu$ model}. It represents the
minimal, physically consistent description of cold matter in the
stellar core. Within this framework, the baryonic sector is governed by the
RMF approximation tuned to reproduce nuclear bulk properties,
while the leptonic sector enforces charge neutrality and $\beta$–stability.
Together, these ingredients yield a barotropic equation of state $P(\varepsilon)$
that can be directly used in the Tolman–Oppenheimer–Volkoff equations to compute
the mass–radius relation of neutron stars \cite{glendenning2000,pogliano2017}.


\section{RMF Lagrangian Formulation}
\label{sec:rmf_lagrangian}

To translate the qualitative picture of the previous section into a quantitative
framework, we formulate the RMF approximation in terms of a
Lagrangian density that describes the interactions among nucleons and meson
fields. From this Lagrangian, the field equations and thermodynamic quantities
follow in a systematic way through the Euler–Lagrange formalism. In the present
context, the aim is not to explore field theory in detail, but rather to obtain
a practical and self–consistent expression for the energy density and pressure
of uniform matter in $\beta$–equilibrium.

In RMF theory, the total Lagrangian is written as the sum of contributions from
each sector,
\begin{equation}
\mathcal{L}
= \mathcal{L}_{N}
+ \mathcal{L}_{\sigma}
+ \mathcal{L}_{\omega}
+ \mathcal{L}_{\rho}
+ \mathcal{L}_{\ell},
\label{eq:L_total}
\end{equation}
where $\mathcal{L}_{N}$ denotes the nucleonic Dirac term, 
$\mathcal{L}_{\sigma}$, $\mathcal{L}_{\omega}$, and $\mathcal{L}_{\rho}$ are the
scalar, isoscalar–vector, and isovector–vector meson sectors, respectively, and
$\mathcal{L}_{\ell}$ accounts for the free leptons (electrons and muons).  The
explicit expressions are
\begin{align}
\mathcal{L}_{N}
&= 
\bar{\psi}\!\left[
i\gamma^{\mu}\!\left(\partial_{\mu}
+ i g_{\omega}\,\omega_{\mu}
+ \tfrac{1}{2} g_{\rho}\,\boldsymbol{\tau}\!\cdot\!\boldsymbol{\rho}_{\mu}\right)
- \bigl(m - g_{\sigma}\sigma\bigr)
\right]\!\psi,
\label{eq:L_N}\\[0.5em]
\mathcal{L}_{\sigma}
&=
\frac{1}{2}\bigl(\partial_{\mu}\sigma\,\partial^{\mu}\sigma - m_{\sigma}^{2}\sigma^{2}\bigr)
- U(\sigma),
\qquad
U(\sigma)
= \frac{1}{3}\,m_{n}b(g_{\sigma}\sigma)^{3}
+ \frac{1}{4}\,c(g_{\sigma}\sigma)^{4},
\label{eq:L_sigma}\\[0.5em]
\mathcal{L}_{\omega}
&=
-\frac{1}{4}\omega_{\mu\nu}\omega^{\mu\nu}
+\frac{1}{2}m_{\omega}^{2}\omega_{\mu}\omega^{\mu},
\qquad
\omega_{\mu\nu}=\partial_{\mu}\omega_{\nu}-\partial_{\nu}\omega_{\mu},
\label{eq:L_omega}\\[0.5em]
\mathcal{L}_{\rho}
&=
-\frac{1}{4}\boldsymbol{\rho}_{\mu\nu}\!\cdot\!\boldsymbol{\rho}^{\mu\nu}
+\frac{1}{2}m_{\rho}^{2}\boldsymbol{\rho}_{\mu}\!\cdot\!\boldsymbol{\rho}^{\mu},
\qquad
\boldsymbol{\rho}_{\mu\nu}
=\partial_{\mu}\boldsymbol{\rho}_{\nu}-\partial_{\nu}\boldsymbol{\rho}_{\mu},
\label{eq:L_rho}\\[0.5em]
\mathcal{L}_{\ell}
&=
\sum_{\lambda=e,\mu}
\bar{\psi}_{\lambda}\bigl(i\gamma^{\mu}\partial_{\mu}-m_{\lambda}\bigr)\psi_{\lambda}.
\label{eq:L_leptons}
\end{align}

The constants $m$, $m_{\sigma}$, $m_{\omega}$, and $m_{\rho}$ are the masses of
the nucleon and mesons, and $g_{\sigma}$, $g_{\omega}$, $g_{\rho}$ are the
respective coupling constants.  The parameters $b$ and $c$ control the
non–linear self–interactions of the scalar field and are fixed empirically to
reproduce the saturation properties of nuclear matter
\cite{glendenning2000}.

In the following sections, we examine each sector in turn. By using the
RMF approximation, we derive the field
equations, the single–particle spectrum, and the corresponding contributions to
the total energy density and pressure. 

\subsection{Nucleon term}
\label{subsec:nucleon_term}

The nucleonic part of the Lagrangian, Eq.~\eqref{eq:L_N}, describes a relativistic
fermion field interacting with meson fields through Yukawa-type couplings.
Varying the Lagrangian with respect to $\bar\psi$ gives the Dirac equation
\begin{equation}
\Bigl[\gamma^\mu\!\bigl(i\partial_\mu - g_\omega\,\omega_\mu - \tfrac{1}{2}g_\rho\,\boldsymbol{\tau}\!\cdot\!\boldsymbol{\rho}_\mu\bigr)
 - (m - g_\sigma\,\sigma)\Bigr]\psi = 0.
\end{equation}
In the RMF approximation appropriate for infinite, uniform matter, spatial
gradients vanish and only the temporal components of the vector fields remain
nonzero. The expectation values of the meson fields are denoted
$\langle\sigma\rangle$, $\langle\omega_0\rangle$, and
$\langle\rho_{03}\rangle$, while the corresponding spatial
components vanish. The nucleons then move independently in the presence of
constant background fields that represent the average effect of their
interactions. The expectation value of the scalar field shifts the nucleon
rest mass to an effective value
\begin{equation}
m^{*} = m - g_\sigma\,\langle\sigma\rangle,
\end{equation}
which reduces the single-particle energy relative to the vacuum and
corresponds to an attractive potential. The time components of the vector
fields, on the other hand, act as constant potentials that add repulsive
energy shifts proportional to the baryon and isospin densities. For a nucleon
species $B$ (proton or neutron) with isospin projection $I_3 = +\tfrac{1}{2}$
for protons and $I_3 = -\tfrac{1}{2}$ for neutrons, the single-particle energy
spectrum becomes
\begin{equation}
e_B(p) = g_\omega\,\langle\omega_0\rangle + I_B\,g_\rho\,\langle\rho_{03}\rangle
         + \sqrt{p^{2} + m^{*2}}.
\label{eq:single_particle_energy}
\end{equation}
At zero temperature, each species fills all momentum states up to its Fermi
momentum $p_{F,B}$, giving the number densities
\[
n_B = \frac{p_{F,B}^{3}}{3\pi^{2}}, \qquad
n_B^{\mathrm{tot}} = n_p + n_n.
\]
The corresponding kinetic contribution to the total energy density follows from
the sum of the single-particle energies over occupied states,
\begin{equation}
\epsilon_{N}^{\mathrm{kin}}
  = \frac{1}{\pi^{2}}\sum_{B=n,p}
    \int_{0}^{p_{F,B}}\!dp\,p^{2}\sqrt{p^{2} + m^{*2}},
\label{eq:epsilon_nucleon}
\end{equation}
while the kinetic (degeneracy) pressure, obtained from the diagonal component of the
energy–momentum tensor or equivalently from the thermodynamic relation
$P = n_B^{2}\,d(\epsilon/n_B)/dn_B$, takes the form
\begin{equation}
P_{N}^{\mathrm{kin}}
  = \frac{1}{3\pi^{2}}\sum_{B=n,p}
    \int_{0}^{p_{F,B}}\!dp\,\frac{p^{4}}{\sqrt{p^{2} + m^{*2}}}.
\label{eq:pressure_nucleon}
\end{equation}
These two integrals represent the free Fermi–gas contributions modified by the
effective mass $m^{*}$. The vector fields enter
the single-particle energies in Eq.~\eqref{eq:single_particle_energy} as constant
shifts. Their effect on the thermodynamics appears not through the integrals
above, but through separate classical field–energy terms associated with
$\langle\omega_0\rangle$ and $\langle\rho_{03}\rangle$. These will be added when the corresponding
meson sectors are discussed below.

The chemical potentials of the nucleons, which determine the conditions for
$\beta$-equilibrium, follow directly from the Fermi energies. For each species,
\begin{align}
\mu_p &= g_\omega\,\langle\omega_0\rangle + \tfrac{1}{2}g_\rho\,\langle\rho_{03}\rangle
         + \sqrt{p_{F,p}^{2} + m^{*2}},\\
\mu_n &= g_\omega\,\langle\omega_0\rangle - \tfrac{1}{2}g_\rho\,\langle\rho_{03}\rangle
         + \sqrt{p_{F,n}^{2} + m^{*2}}.
\end{align}
The difference between neutron and proton chemical potentials is thus governed
by the $\rho$ field, while their common shift arises from the $\omega$ field.
The effective mass $m^{*}$ encapsulates the attractive scalar interaction,
and the competition between these scalar and vector terms determines the net
binding and saturation of nuclear matter. Together, Eqs.~\eqref{eq:epsilon_nucleon}
and \eqref{eq:pressure_nucleon} provide the nucleonic (kinetic) parts of the
energy density and pressure, to which the meson field energies will now be
added to obtain the full equation of state.

\subsection{Scalar field term}
\label{subsec:sigma_term}

The scalar $\sigma$ field represents the attractive part of the strong
interaction between nucleons. Its Lagrangian, Eq.~\eqref{eq:L_sigma}, contains a
quadratic mass term and additional cubic and quartic self–interactions,
\[
\mathcal{L}_{\sigma}
= \frac{1}{2}\bigl(\partial_{\mu}\sigma\,\partial^{\mu}\sigma - m_{\sigma}^{2}\sigma^{2}\bigr)
- \frac{1}{3}\,m_{n}b\,(g_{\sigma}\sigma)^{3}
- \frac{1}{4}\,c\,(g_{\sigma}\sigma)^{4}.
\]
The self–interaction $U(\sigma)$ softens the scalar attraction at high
density and is essential to reproduce realistic nuclear compressibility.
Varying the Lagrangian with respect to $\sigma$ yields the field equation
\begin{equation}
\partial_{\mu}\partial^{\mu}\sigma
+ m_{\sigma}^{2}\sigma
+ m_{n}b\,(g_{\sigma}\sigma)^{2}
+ c\,(g_{\sigma}\sigma)^{3}
= g_{\sigma}\,\bar{\psi}\psi.
\end{equation}
In uniform matter, spatial derivatives vanish and the field is constant.
Within the mean–field approximation we assume factorization of powers of the
field, $\langle\sigma^{n}\rangle \approx \langle\sigma\rangle^{n}$, which leads to
\begin{equation}
m_{\sigma}^{2}\,\langle\sigma\rangle
+ m_{n}b\,(g_{\sigma}\langle\sigma\rangle)^{2}
+ c\,(g_{\sigma}\langle\sigma\rangle)^{3}
= g_{\sigma}\,\langle\bar{\psi}\psi\rangle.
\label{eq:sigma_mf_eq}
\end{equation}
where $\langle\bar{\psi}\psi\rangle$ is the scalar density of nucleons,
\begin{equation}
\langle\bar{\psi}\psi\rangle
= \sum_{B=n,p}\frac{1}{\pi^{2}}\!
  \int_{0}^{p_{F,B}}\!dp\,p^{2}\,
  \frac{m^{*}}{\sqrt{p^{2}+m^{*2}}}.
\label{eq:scalar_density}
\end{equation}
This self–consistency relation couples the scalar mean field $\langle\sigma\rangle$ to the
occupied Fermi seas and must be solved together with the vector–meson and
equilibrium conditions at each baryon density.

The scalar field contributes to the total energy density both directly, through
its potential energy, and indirectly, through the modification of the nucleon
masses in the baryon integral. The contribution to the energy density associated
with the $\sigma$ field is 
as
\begin{equation}
\epsilon_{\sigma}
= \frac{1}{2}m_{\sigma}^{2}\langle\sigma\rangle^{2}
  + \frac{1}{3}m_{n}b\,(g_{\sigma}\langle\sigma\rangle)^{3}
  + \frac{1}{4}c\,(g_{\sigma}\langle\sigma\rangle)^{4}
\label{eq:epsilon_sigma}
\end{equation}
The corresponding pressure follows from the
energy–momentum tensor or from the thermodynamic identity
$P = \sum_i \mu_i n_i - \epsilon$, yielding
\begin{equation}
P_{\sigma}
= -\,\frac{1}{2}m_{\sigma}^{2}\langle\sigma\rangle^{2}
  - \frac{1}{3}m_{n}b\,(g_{\sigma}\langle\sigma\rangle)^{3}
  - \frac{1}{4}c\,(g_{\sigma}\langle\sigma\rangle)^{4}
\label{eq:pressure_sigma}
\end{equation}
These expressions represent the pure field contributions of the scalar meson to
the total energy density and pressure. The $\sigma$ field lowers the total
energy by generating an attractive interaction between nucleons and thus plays
a central role in producing nuclear binding and saturation in the relativistic
framework.

\subsection{Vector field term}
\label{subsec:omega_term}

The vector $\omega$ field represents the short–range repulsive part of the
nuclear interaction. It couples to the conserved baryon current and provides a
repulsive potential that grows with density, counteracting the attraction from
the scalar field and ensuring that nuclear matter saturates at a finite
density. The corresponding part of the Lagrangian is
\[
\mathcal{L}_{\omega}
= -\frac{1}{4}\,\omega_{\mu\nu}\omega^{\mu\nu}
  + \frac{1}{2}\,m_{\omega}^{2}\,\omega_{\mu}\omega^{\mu},
\qquad
\omega_{\mu\nu} = \partial_{\mu}\omega_{\nu} - \partial_{\nu}\omega_{\mu}.
\]
Variation with respect to $\omega_{\mu}$ gives the field equation
\[
\partial_{\nu}\omega^{\mu\nu} + m_{\omega}^{2}\omega^{\mu}
= g_{\omega}\,\bar{\psi}\gamma^{\mu}\psi.
\]
For uniform, static matter, spatial derivatives vanish and only the time
component survives. Denoting its expectation value by
$\langle\omega^{0}\rangle$, one obtains the simple algebraic
relation
\begin{equation}
m_{\omega}^{2}\,\langle\omega_{0}\rangle = g_{\omega}\,n_{B},
\label{eq:omega_field_eq}
\end{equation}
where $n_{B} = \langle\psi^{\dagger}\psi\rangle = n_{p} + n_{n}$ is the total
baryon density. The $\omega$ field therefore grows linearly with density and
acts as a uniform repulsive potential felt equally by all nucleons.

Substituting this mean field into the energy–momentum tensor gives the
contribution of the $\omega$ field to the energy density and pressure,
\begin{equation}
\epsilon_{\omega} = \frac{1}{2}\,m_{\omega}^{2}\,\langle\omega_{0}\rangle^{2},
\qquad
P_{\omega} = \frac{1}{2}\,m_{\omega}^{2}\,\langle\omega_{0}\rangle^{2}.
\label{eq:omega_eps_P}
\end{equation}
Using Eq.~\eqref{eq:omega_field_eq}, these expressions can be written directly
in terms of the baryon density as
\[
\epsilon_{\omega} = P_{\omega}
= \frac{1}{2}\,\Bigl(\frac{g_{\omega}}{m_{\omega}}\Bigr)^{2} n_{B}^{2}.
\]
The $\omega$ field thus contributes an equal, positive amount to the energy
density and pressure, reflecting the purely repulsive character of the
interaction it mediates. This repulsion stiffens the equation of state at high
density and plays a key role in determining the maximum mass of neutron stars.
The competition between the attractive $\sigma$ field and the repulsive
$\omega$ field establishes the saturation point of nuclear matter and controls
the overall stiffness of the equation of state in the RMF approximation.

\subsection{Isovector field term}
\label{subsec:rho_term}

The $\rho$ meson introduces the dependence of the nuclear interaction on
isospin asymmetry, that is, on the difference between the neutron and proton
densities. It provides an additional repulsive contribution that grows with the
neutron–proton imbalance and determines the symmetry energy of nuclear matter.
The Lagrangian for the $\rho$ field is
\[
\mathcal{L}_{\rho}
= -\frac{1}{4}\,\boldsymbol{\rho}_{\mu\nu}\!\cdot\!\boldsymbol{\rho}^{\mu\nu}
  + \frac{1}{2}\,m_{\rho}^{2}\,
    \boldsymbol{\rho}_{\mu}\!\cdot\!\boldsymbol{\rho}^{\mu},
\qquad
\boldsymbol{\rho}_{\mu\nu}
  = \partial_{\mu}\boldsymbol{\rho}_{\nu} - \partial_{\nu}\boldsymbol{\rho}_{\mu}.
\]
Here, $\boldsymbol{\rho}_{\mu}$ is an isovector field with three components in
isospin space. It couples to the third component of the nucleon isospin
operator, and in uniform matter only this component contributes. The Euler–Lagrange equation for $\boldsymbol{\rho}_{\mu}$ reads
\[
\partial_{\nu}\boldsymbol{\rho}^{\mu\nu}
+ m_{\rho}^{2}\,\boldsymbol{\rho}^{\mu}
= g_{\rho}\,\bar{\psi}\gamma^{\mu}\,\boldsymbol{\tau}\psi.
\]
In the RMF approximation, only the time component of the third isospin direction
remains nonzero, $\langle\rho_{03}\rangle$, while all others
vanish. The equation of motion then reduces to
\begin{equation}
m_{\rho}^{2}\,\langle\rho_{03}\rangle
= \frac{1}{2}\,g_{\rho}\,(n_{p}-n_{n}),
\label{eq:rho_field_eq}
\end{equation}
where $n_{p}$ and $n_{n}$ are the proton and neutron densities. The $\rho$
field is therefore directly proportional to the neutron–proton imbalance and
vanishes in symmetric nuclear matter where $n_{p}=n_{n}$.

The single–particle energies of nucleons include an additional term
$\pm \tfrac{1}{2} g_{\rho}\,\langle\rho_{03}\rangle$, which increases the energy of
neutrons relative to protons when the matter becomes neutron-rich. This effect
raises the energy of asymmetric configurations and defines the symmetry energy
coefficient of nuclear matter.

The energy density and pressure associated with the uniform $\rho$ field follow
from its classical potential energy,
\begin{equation}
\epsilon_{\rho} = \frac{1}{2}\,m_{\rho}^{2}\,\langle\rho_{03}\rangle^{2},
\qquad
P_{\rho} = \frac{1}{2}\,m_{\rho}^{2}\,\langle\rho_{03}\rangle^{2}.
\label{eq:rho_eps_P}
\end{equation}
Using Eq.~\eqref{eq:rho_field_eq}, these can be written in terms of the baryon
densities as
\[
\epsilon_{\rho} = P_{\rho}
= \frac{1}{8}\,
  \Bigl(\frac{g_{\rho}}{m_{\rho}}\Bigr)^{2}
  (n_{p}-n_{n})^{2}.
\]
This term always increases the total energy and pressure, providing a repulsive
contribution that penalizes large isospin asymmetry. In neutron-star matter,
where $n_{n}>n_{p}$, the $\rho$ meson is essential to ensure that the chemical
potentials of neutrons, protons, and leptons can satisfy
$\beta$-equilibrium and charge neutrality simultaneously. Its inclusion allows
the model to reproduce the empirical symmetry energy coefficient of nuclear
matter and thus to describe correctly the composition and stiffness of
asymmetric matter in neutron-star cores.

\subsection{Leptonic sector}
\label{subsec:leptons}

The leptons (electrons and muons) are included to ensure charge neutrality and
$\beta$–equilibrium in stellar matter. They do not participate in the strong
interaction and are therefore treated as free relativistic Fermi gases. The
corresponding part of the Lagrangian is
\[
\mathcal{L}_{\ell}
= \sum_{\lambda=e,\mu}
  \bar{\psi}_{\lambda}\bigl(i\gamma^{\mu}\partial_{\mu}
  - m_{\lambda}\bigr)\psi_{\lambda}.
\]
Each lepton species $\lambda$ occupies all momentum states up to its Fermi
momentum $p_{F,\lambda}$, defined by the number density
\[
n_{\lambda} = \frac{p_{F,\lambda}^{3}}{3\pi^{2}}.
\]
In cold, degenerate matter the leptons obey the dispersion relation
$e_{\lambda}(p)=\sqrt{p^{2}+m_{\lambda}^{2}}$, with chemical potential
$\mu_{\lambda}=\sqrt{p_{F,\lambda}^{2}+m_{\lambda}^{2}}$. The muon appears only
when the electron chemical potential exceeds the muon rest mass,
$\mu_{e}\ge m_{\mu}$; below that threshold, muons are absent and all negative
charge is carried by electrons.

The energy density and pressure of the leptons are obtained by integrating the
relativistic Fermi–gas expressions,
\begin{align}
\epsilon_{\ell}
&= \sum_{\lambda=e,\mu}\frac{1}{\pi^{2}}
   \int_{0}^{p_{F,\lambda}}\!dp\,p^{2}\sqrt{p^{2}+m_{\lambda}^{2}},
\\[0.4em]
P_{\ell}
&= \sum_{\lambda=e,\mu}\frac{1}{3\pi^{2}}
   \int_{0}^{p_{F,\lambda}}\!dp\,\frac{p^{4}}{\sqrt{p^{2}+m_{\lambda}^{2}}}.
\end{align}
These integrals have analytic forms that can be evaluated numerically or
approximated in the ultrarelativistic limit ($p_{F,\lambda}\!\gg\!m_{\lambda}$)
as $\epsilon_{\ell}\simeq3P_{\ell}\simeq p_{F,\lambda}^{4}/(4\pi^{2})$.

The leptons interact only through the electromagnetic field, which in bulk
neutral matter averages to zero, so their only role in the RMF approximation is to
balance the positive charge of the protons and to enforce the conditions of
chemical equilibrium. Charge neutrality requires
\[
n_{p} = n_{e} + n_{\mu},
\]
while $\beta$–equilibrium, maintained by weak interactions, imposes the chemical
relations
\[
\mu_{n} = \mu_{p} + \mu_{e}, \qquad
\mu_{e} = \mu_{\mu},
\]
the latter holding whenever muons are present. Together with the baryonic field
equations, these constraints determine the composition of the system at each
density.

\subsection{Self–consistent mean–field system and equation of state}
\label{subsec:rmf_selfconsistent}

The RMF approximation used above leads to a closed set of
algebraic relations that determine the composition and thermodynamic properties
of matter at each baryon density.  For uniform, cold matter the meson fields are
replaced by their constant expectation values $\langle\sigma\rangle$, $\langle\omega_{0}\rangle$ and
$\langle\rho_{03}\rangle$, while the leptons are described by free Fermi gases.  The
quantities to be determined self–consistently are the mean fields, the Fermi
momenta of all particle species, and the associated chemical potentials.

The three meson fields obey the mean–field equations
\begin{align}
m_{\sigma}^{2}\,\langle\sigma\rangle
+ m_{n}b\,(g_{\sigma}\langle\sigma\rangle)^{2}
+ c\,(g_{\sigma}\langle\sigma\rangle)^{3}
&= g_{\sigma}\,
   \sum_{B=n,p}\frac{1}{\pi^{2}}
   \int_{0}^{p_{F,B}}\!dp\,p^{2}\,
   \frac{m^{*}}{\sqrt{p^{2}+m^{*2}}},
\\[0.5em]
m_{\omega}^{2}\,\langle\omega_{0}\rangle
&= g_{\omega}\,(n_{p}+n_{n}),
\\[0.5em]
m_{\rho}^{2}\,\langle\rho_{03}\rangle
&= \tfrac{1}{2}\,g_{\rho}\,(n_{p}-n_{n}).
\end{align}
with the effective nucleon mass $m^{*}=m-g_{\sigma}\langle\sigma\rangle$.  At the same
time, the conditions of charge neutrality and $\beta$–equilibrium link the
baryonic and leptonic sectors,
\begin{align}
n_{p} &= n_{e} + n_{\mu},\\
\mu_{n} &= \mu_{p} + \mu_{e},\\
\mu_{e} &= \mu_{\mu},
\end{align}
where the chemical potentials are
\begin{align}
\mu_{n} &= g_{\omega}\,\langle\omega_{0}\rangle
          - \tfrac{1}{2}\,g_{\rho}\,\langle\rho_{03}\rangle
          + \sqrt{p_{F,n}^{2}+m^{*2}},\\[0.4em]
\mu_{p} &= g_{\omega}\,\langle\omega_{0}\rangle
          + \tfrac{1}{2}\,g_{\rho}\,\langle\rho_{03}\rangle
          + \sqrt{p_{F,p}^{2}+m^{*2}},\\[0.4em]
\mu_{e} &= \sqrt{p_{F,e}^{2}+m_{e}^{2}},\\[0.4em]
\mu_{\mu} &= \sqrt{p_{F,\mu}^{2}+m_{\mu}^{2}}.
\end{align}
For a chosen total baryon density $n_{B}=n_{p}+n_{n}$, these equations form a
nonlinear system that is solved iteratively.  One starts with a trial value of
$\bar\sigma$, computes $m^{*}$, determines the Fermi momenta consistent with
charge neutrality and $\beta$–equilibrium, and then updates the fields until
convergence is achieved.  The procedure yields, for each density point, the
composition of the matter $(n_{p},n_{n},n_{e},n_{\mu})$, the mean fields
$(\langle\sigma\rangle,\langle\omega_{0}\rangle,\langle\rho_{03}\rangle)$, and the corresponding thermodynamic
quantities.

The total energy density and pressure follow by adding the contributions from
all sectors:
\begin{align}
\epsilon &=
  \frac{1}{2}\,m_{\sigma}^{2}\,\langle\sigma\rangle^{2}
  + \frac{1}{3}\,m_{n}b\,(g_{\sigma}\langle\sigma\rangle)^{3}
  + \frac{1}{4}\,c\,(g_{\sigma}\langle\sigma\rangle)^{4}
  + \frac{1}{2}\,m_{\omega}^{2}\,\langle\omega_{0}\rangle^{2}
  + \frac{1}{2}\,m_{\rho}^{2}\,\langle\rho_{03}\rangle^{2}
\\[0.3em]
&\quad
  + \sum_{B=n,p}\frac{1}{\pi^{2}}
    \int_{0}^{p_{F,B}}\!dp\,p^{2}\sqrt{p^{2}+m^{*2}}
\\[0.3em]
&\quad
  + \sum_{\lambda=e,\mu}\frac{1}{\pi^{2}}
    \int_{0}^{p_{F,\lambda}}\!dp\,p^{2}
    \sqrt{p^{2}+m_{\lambda}^{2}},
\\[1.0em]
P &=
 -\,\frac{1}{2}\,m_{\sigma}^{2}\,\langle\sigma\rangle^{2}
 - \frac{1}{3}\,m_{n}b\,(g_{\sigma}\langle\sigma\rangle)^{3}
 - \frac{1}{4}\,c\,(g_{\sigma}\langle\sigma\rangle)^{4}
 + \frac{1}{2}\,m_{\omega}^{2}\,\langle\omega_{0}\rangle^{2}
 + \frac{1}{2}\,m_{\rho}^{2}\,\langle\rho_{03}\rangle^{2}
\\[0.3em]
&\quad
  + \sum_{B=n,p}\frac{1}{3\pi^{2}}
    \int_{0}^{p_{F,B}}\!dp\,\frac{p^{4}}{\sqrt{p^{2}+m^{*2}}}
\\[0.3em]
&\quad
  + \sum_{\lambda=e,\mu}\frac{1}{3\pi^{2}}
    \int_{0}^{p_{F,\lambda}}\!dp\,\frac{p^{4}}{\sqrt{p^{2}+m_{\lambda}^{2}}}.
\end{align}
These relations define a barotropic equation of state $P(\epsilon)$ that can be used
directly in the TOV equations to compute the global
structure of neutron stars.  Once calibrated to reproduce the empirical bulk
properties of symmetric nuclear matter at saturation, the RMF approximation provides
a consistent microscopic description of dense, asymmetric matter and forms the
basis for many modern neutron-star equations of state.

\chapter{Conclusion and Outlook}
\label{chap:conclusion}

The aim of this thesis has been to develop a coherent and progressively more
realistic description of neutron--star structure, beginning with the
relativistic equations that govern hydrostatic equilibrium and culminating in a
self–consistent calculation of dense matter within the relativistic mean–field
$npe\mu$ model. Each step in this progression served to isolate how
microphysical assumptions about matter translate into macroscopic predictions
for the mass–radius relation of neutron stars.

The starting point was the derivation of the
Tolman--Oppenheimer--Volkoff equations, which form the backbone of any
non–rotating neutron--star model. Their structure makes clear that gravity,
pressure, and spacetime curvature are inseparably linked, and that no stellar
model is complete without a physically motivated equation of state. Using this
framework, ideal Fermi gases provided the first concrete examples. The
non–interacting, zero–temperature Fermi gas captures the essential role of
degeneracy pressure and illustrates how the stiffness of an equation of state
depends on whether the fermions are non–relativistic or ultra–relativistic.
Solving the TOV equations with this idealized microphysics yielded the
classical Oppenheimer--Volkoff limit: a maximum mass far below what is observed
in nature. Although quantitatively unrealistic, these models were valuable for
building numerical intuition and for demonstrating how relativistic corrections
enforce upper bounds on the mass of compact stars.

The next step was the construction of ideal neutron--star models based on a
pure neutron Fermi gas. These results largely confirmed the physical picture
anticipated from the analytic limits: relativistic gravitational effects
strengthen the tendency toward collapse, while degeneracy pressure becomes
progressively less effective at stabilizing high–density matter. The resulting
mass–radius curves exhibited the expected turnover at the maximum mass, and the
accompanying radial–mode analysis verified the standard stability criterion.
These calculations established the numerical infrastructure later used for more
realistic equations of state.

The principal part of this thesis was the implementation and study of the
relativistic mean–field $npe\mu$ model. Here, interactions between nucleons are
mediated by scalar, vector, and isovector meson fields, calibrated to reproduce
empirical bulk properties of saturated nuclear matter. Charge neutrality and
$\beta$–equilibrium were enforced through the inclusion of electrons and
muons. Solving the coupled nonlinear field equations at each density produced a
complete microscopic equation of state for uniform matter. This model captures
several qualitative features expected of neutron–star interiors, including the
growth of the proton fraction at higher densities, the appearance of muons once
the electron chemical potential becomes large enough, and a realistic
competition between attractive scalar and repulsive vector interactions.

To obtain a stellar model that spans the full density range, the microscopic
core description was matched to a crust equation of state based on the FPS
parametrization. This produced a complete $P(\varepsilon)$ relation suitable
for TOV integration. The resulting mass–radius curve showed reasonable radii
for typical neutron–star masses and behaved smoothly across the crust–core
interface.

However, when confronted with current observational constraints—including radio
timing of massive pulsars, NICER mass–radius measurements, and the
semiparametric EoS posteriors of Ng~et~al.—the limitations of the $npe\mu$
model became clear. Most notably, the maximum mass of the sequence lies below
that of the heaviest firmly measured pulsars. The predicted radii at the NICER
masses also deviate from the central values of the observational contours.
These mismatches indicate that the minimal nucleonic mean–field model, even
when calibrated to nuclear bulk properties, lacks enough stiffness at high
densities to support the observed population of heavy neutron stars. This is
consistent with the broader literature, where additional degrees of freedom,
more sophisticated treatments of nuclear correlations, or density–dependent
couplings are often required to achieve agreement with observations.

In summary, the results of this thesis show that the $npe\mu$ model provides a
clear and controlled improvement over idealized Fermi–gas descriptions. It
reproduces many qualitative aspects of dense matter and yields mass–radius
curves that fall within a plausible range for intermediate densities. At the
same time, the quantitative discrepancies with astrophysical data demonstrate
that this minimal approach is not sufficient as a complete description of
neutron–star interiors. A more realistic equation of state must include
additional physical mechanisms that soften or stiffen the matter in appropriate
density regimes.

\section*{Outlook}

Several natural extensions follow from the present work.

\textbf{Improved nuclear microphysics:}  
Density–dependent RMF models, Skyrme–type functionals, or
chiral–effective–field–theory–based constraints at low densities could be
incorporated to refine the stiffness of the EoS.

\textbf{Additional degrees of freedom:}  
At supranuclear densities, hyperons, $\Delta$–resonances, meson condensates, or
deconfined quark matter may appear. Their inclusion typically softens the EoS,
requiring repulsive interactions or phase–transition effects to remain
consistent with the observed $2\,M_{\odot}$ stars.

\textbf{Phase transitions and hybrid stars:}  
A natural next step is to explore hadron–quark transitions within the same TOV
framework. Matching the RMF model to a quark–matter EoS, such as quark–meson
or NJL–type models, would allow a systematic study of hybrid–star sequences.

\textbf{Confronting new observational data:}  
Future NICER measurements, improved gravitational–wave constraints on tidal
deformabilities, and radio timing of even heavier pulsars will further restrict
the allowed EoS space. Incorporating these results into the modelling pipeline
developed here is a direct continuation of this work.

The overall conclusion is that the methods and models developed in this thesis
establish a solid baseline for neutron–star calculations, while the
discrepancies with observations clearly motivate the exploration of more
sophisticated descriptions of dense matter.


\appendix
\chapter{General Relativity: Setup and Conventions}
\label{app:GR_basics}

This appendix fixes notation and defines the geometric objects used in the project. No derivations are given.

\section{Units, signature, indices}
We use natural units with $c=G=1$. The metric signature is $(+,-,-,-)$, so timelike vectors obey $u^\mu u_\mu=1$. Greek indices $\mu,\nu,\ldots$ run over spacetime coordinates $0,1,2,3$; Latin indices $i,j,\ldots$ denote spatial components $1,2,3$. Symmetrization and antisymmetrization use $A_{(\mu\nu)}=\tfrac12(A_{\mu\nu}+A_{\nu\mu})$ and $A_{[\mu\nu]}=\tfrac12(A_{\mu\nu}-A_{\nu\mu})$.

\section{Metric and volume element}
The metric tensor $g_{\mu\nu}$ defines the line element
\begin{equation}
ds^2 = g_{\mu\nu}\,dx^\mu dx^\nu,
\end{equation}
and raises/lowers indices with $g^{\mu\nu}$ and $g_{\mu\nu}$, where $g^{\mu\alpha}g_{\alpha\nu}=\delta^\mu{}_\nu$. The determinant is $g\equiv\det(g_{\mu\nu})<0$ for signature $(+,-,-,-)$. The invariant spacetime volume element is $\sqrt{-g}\,d^4x$.

\section{Covariant derivative and Christoffel symbols}
We use the Levi–Civita connection (metric compatible and torsion free). The covariant derivative of a vector is
\begin{equation}
\nabla_\mu V^\nu = \partial_\mu V^\nu + \Gamma^\nu{}_{\mu\lambda} V^\lambda,\qquad
\nabla_\mu V_\nu = \partial_\mu V_\nu - \Gamma^\lambda{}_{\mu\nu} V_\lambda,
\end{equation}
with Christoffel symbols
\begin{equation}
\Gamma^{\rho}{}_{\mu\nu}=\tfrac12 g^{\rho\sigma}(\partial_\mu g_{\nu\sigma}+\partial_\nu g_{\mu\sigma}-\partial_\sigma g_{\mu\nu}).
\label{eq:christoffel_def}
\end{equation}

\section{Curvature tensors}
The Riemann tensor acts on vectors transported around an infinitesimal loop and is defined by
\begin{equation}
R^{\rho}{}_{\sigma\mu\nu}
= \partial_{\mu}\Gamma^{\rho}{}_{\nu\sigma}
- \partial_{\nu}\Gamma^{\rho}{}_{\mu\sigma}
+ \Gamma^{\rho}{}_{\mu\lambda}\Gamma^{\lambda}{}_{\nu\sigma}
- \Gamma^{\rho}{}_{\nu\lambda}\Gamma^{\lambda}{}_{\mu\sigma}.
\label{eq:riemann_def}
\end{equation}
Its contractions give the Ricci tensor and Ricci scalar,
\begin{equation}
R_{\mu\nu}=R^{\rho}{}_{\mu\rho\nu},\qquad R=g^{\mu\nu}R_{\mu\nu}.
\label{eq:ricci_def}
\end{equation}

\section{Einstein tensor and Bianchi identity}
The Einstein tensor is
\begin{equation}
G_{\mu\nu}=R_{\mu\nu}-\tfrac12 R\,g_{\mu\nu}.
\label{eq:einstein_tensor_def}
\end{equation}
The twice–contracted Bianchi identity holds for the Levi–Civita connection:
\begin{equation}
\nabla^\mu G_{\mu\nu}=0.
\end{equation}

\section{Energy–momentum tensor}
For matter and fields the energy–momentum tensor $T_{\mu\nu}$ encodes energy density, momentum density, and stresses. Local conservation follows from diffeomorphism invariance and the Bianchi identity:
\begin{equation}
\nabla^\mu T_{\mu\nu}=0.
\end{equation}
For a perfect fluid,
\begin{equation}
T_{\mu\nu}=(\epsilon+p)u_\mu u_\nu - p\,g_{\mu\nu},
\label{eq:perfect_fluid_Tmn}
\end{equation}
with energy density $\epsilon$, pressure $p$, and four–velocity $u^\mu$ normalized by $u^\mu u_\mu=1$ (signature $(+,-,-,-)$).

\section{Einstein field equations}
Geometry and matter are related by
\begin{equation}
R_{\mu\nu}-\tfrac12 R\,g_{\mu\nu}=8\pi T_{\mu\nu}.
\label{eq:einstein_equation}
\end{equation}
With $c=G=1$, the coupling is $8\pi$. The sign conventions here are consistent with signature $(+,-,-,-)$. If comparing to sources using the mostly–plus signature $(-,+,+,+)$ (e.g.\ Carroll), note that several signs in intermediate formulas change accordingly, though invariant statements and final physics agree.
\chapter{Numerical Implementation and Code Availability}
\label{app:code}

All numerical results presented in the text were produced using
custom Python code developed for this project.
The implementation covers the construction of equations of state,
solutions of the Tolman--Oppenheimer--Volkoff equations, stability
analysis of radial oscillations, and post-processing for comparison
with observational constraints.

Due to the size and complexity of the codebase, the full source code
is not reproduced in this text. Instead, the complete implementation
is made available in a public version-controlled repository:
\url{https://github.com/sondreklyve/TFY4510}.

The numerical code is structured into modules corresponding to the
different physical models discussed in the text, including ideal
Fermi gas models, incompressible stars, idealized neutron stars, and
$npe\mu$ matter described within a relativistic mean-field framework.
All figures appearing in the text are produced using this code, either
from data generated by the implementation or from externally provided datasets.

Most of the numerical implementation is adapted from existing
academic work. In particular, the $npe\mu$ relativistic mean-field
solver is based on code developed by Pogliano \cite{pogliano2017},
and the ideal neutron star and radial stability analysis builds on work by Sletmoen \cite{sletmoen2022}.
These components have been refactored,
extended, and integrated into a unified framework for the present
project. Full attribution and links to the original sources are
provided in the repository documentation.


\printbibliography

\end{document}
