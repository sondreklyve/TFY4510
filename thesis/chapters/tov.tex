\chapter{Tolman–Oppenheimer–Volkoff Equations}

\section{Introduction}

The structure of relativistic stars cannot be captured within the framework of Newtonian gravity. 
In neutron stars, the densities reach values comparable to those of nuclear matter, and the 
gravitational field is so strong that a fully relativistic treatment becomes indispensable. 
General relativity provides the natural framework to describe how matter curves spacetime and, 
conversely, how this curved geometry governs the equilibrium of stellar matter.

In this chapter we derive the equations governing static, spherically symmetric configurations of 
perfect fluids in general relativity, known as the Tolman-Oppenheimer-Volkoff (TOV)
equations. They generalize the Newtonian hydrostatic equilibrium equation to curved spacetime. 
Hence, the TOV equations form the central tool for studying compact stars such as white dwarfs, neutron stars, and hypothetical 
hybrid stars with exotic cores. They link microscopic physics, encoded in the 
equation of state of dense matter, to macroscopic observables such as stellar masses and radii.

Our starting point will be Einstein's field equations, together with the assumption of spherical 
symmetry and a perfect-fluid energy–momentum tensor. From these ingredients we obtain a set of 
coupled ordinary differential equations relating the pressure, energy density, and enclosed 
gravitational mass as functions of the radial coordinate. To close the system, one must specify 
an equation of state, thereby providing the crucial connection between microphysics and 
astrophysical observables.

Historically, these equations were first formulated independently by Tolman \cite{tolman1939} 
and by Oppenheimer and Volkoff \cite{oppenheimer1939}. Their pioneering work laid the foundation 
for modern neutron star astrophysics, showing that general relativity predicts a maximum mass for 
compact stars beyond which no stable configuration exists. This insight still guides research into 
the ultimate fate of dense matter and the physics of gravitational collapse.

The derivation presented here follows the standard approach used in the literature 
\cite{oppenheimer1939, tolman1939, carroll}. 

\paragraph{Acknowledgment.} 
The organization and several intermediate steps are adapted from Carroll’s 
\emph{Spacetime and Geometry} \cite{carroll}, rewritten for the mostly minus signature.

\section{Setup and Conventions}

Before deriving the Tolman–Oppenheimer–Volkoff (TOV) equations we state the conventions used 
throughout this chapter. We employ natural units such that $c=G=1$. With this choice, the coupling 
in Einstein’s equations appears as $8\pi$. The metric signature is taken as $(+,-,-,-)$, so that 
timelike vectors satisfy $u^\mu u_\mu=1$. Greek indices $\mu,\nu,\dots$ run over spacetime 
coordinates $(0,1,2,3)$, while Latin indices $i,j,\dots$ are reserved for spatial components. 

\subsection*{Curvature and field equations}
We adopt the following definition of the Riemann tensor,
\begin{equation}
R^{\rho}{}_{\sigma\mu\nu}
 = \partial_{\mu}\Gamma^{\rho}{}_{\nu\sigma}
 - \partial_{\nu}\Gamma^{\rho}{}_{\mu\sigma}
 + \Gamma^{\rho}{}_{\mu\lambda}\Gamma^{\lambda}{}_{\nu\sigma}
 - \Gamma^{\rho}{}_{\nu\lambda}\Gamma^{\lambda}{}_{\mu\sigma},
\end{equation}
from which the Ricci tensor is $R_{\mu\nu}=R^{\rho}{}_{\mu\rho\nu}$ 
and the Ricci scalar $R=g^{\mu\nu}R_{\mu\nu}$. Other sign conventions exist in the literature, 
so it is important to note that all results here are consistent with this choice. 

The Einstein tensor is defined as $G_{\mu\nu}=R_{\mu\nu}-\tfrac{1}{2}R g_{\mu\nu}$, 
and Einstein’s field equations then take the form
\begin{equation}
R_{\mu\nu} - \tfrac{1}{2} R g_{\mu\nu} = 8\pi T_{\mu\nu}.
\label{eq:einstein_equation}
\end{equation}

\subsection*{Metric ansatz and coordinates}
We now restrict attention to static, spherically symmetric stellar configurations. 
By Birkhoff’s theorem, the most general spherically symmetric vacuum solution of 
Einstein’s equations is the Schwarzschild metric. Inside the star, where matter is present, 
the metric must reduce continuously to the Schwarzschild form at the stellar surface. 
It is therefore natural to adopt Schwarzschild–like (curvature) coordinates and write the 
line element as
\begin{equation}
ds^{2} = A(r)\, dt^{2} - B(r)\, dr^{2} - r^{2}\,(d\theta^{2}+\sin^{2}\theta\, d\phi^{2}),
\label{eq:metric_ansatz}
\end{equation}
where $A(r)$ and $B(r)$ are functions of the radial coordinate only. 
The assumption of staticity ensures that no cross terms such as $dr\,dt$ appear, 
while spherical symmetry fixes the angular part to the standard 
$r^{2} d\Omega^{2}$ form.

\section{Explicit computation of Einstein tensor components}

It is convenient to reparameterize the metric functions by
\begin{equation}
A(r)=e^{2\alpha(r)},\qquad B(r)=e^{2\beta(r)},
\end{equation}
so that the line element becomes
\begin{equation}
ds^{2}=e^{2\alpha(r)}\,dt^{2}-e^{2\beta(r)}\,dr^{2}-r^{2}\,\bigl(d\theta^{2}+\sin^{2}\theta\,d\phi^{2}\bigr).
\label{eq:sss_alpha_beta}
\end{equation}
Nonvanishing Christoffel symbols needed for the curvature are
\begin{align}
\Gamma^{t}{}_{tr} &= \alpha', &
\Gamma^{r}{}_{tt} &= \alpha' e^{2(\alpha-\beta)}, &
\Gamma^{r}{}_{rr} &= \beta', \nonumber\\[4pt]
\Gamma^{r}{}_{\theta\theta} &= -r\,e^{-2\beta}, &
\Gamma^{r}{}_{\phi\phi} &= -r\,e^{-2\beta}\sin^{2}\theta, &
\Gamma^{\theta}{}_{r\theta} &= \Gamma^{\phi}{}_{r\phi}=\tfrac{1}{r}, \nonumber\\[2pt]
\Gamma^{\theta}{}_{\phi\phi} &= -\sin\theta\cos\theta, &
\Gamma^{\phi}{}_{\theta\phi} &= \cot\theta,
\label{eq:Gamma_subset}
\end{align}
where primes denote $d/dr$.

The Ricci tensor is $R_{\mu\nu}=\partial_{\lambda}\Gamma^{\lambda}{}_{\mu\nu}
-\partial_{\nu}\Gamma^{\lambda}{}_{\mu\lambda}
+\Gamma^{\lambda}{}_{\mu\nu}\Gamma^{\sigma}{}_{\lambda\sigma}
-\Gamma^{\sigma}{}_{\mu\lambda}\Gamma^{\lambda}{}_{\nu\sigma}$. Stationarity ($\partial_{t}=0$) and spherical symmetry simplify the components. For $R_{tt}$ one finds
\begin{align}
R_{tt}
&=\partial_{r}\Gamma^{r}{}_{tt}
+\Gamma^{r}{}_{tt}\!\left(\Gamma^{t}{}_{rt}+\Gamma^{r}{}_{rr}+\Gamma^{\theta}{}_{r\theta}+\Gamma^{\phi}{}_{r\phi}\right)
-\Gamma^{t}{}_{tr}\Gamma^{r}{}_{tt}\nonumber\\[2pt]
&=e^{2(\alpha-\beta)}\!\left[\alpha''+(\alpha')^{2}-\alpha'\beta'+\frac{2\alpha'}{r}\right].
\label{eq:Rtt_result}
\end{align}
The Ricci scalar for \eqref{eq:sss_alpha_beta} is
\begin{equation}
R
=2e^{-2\beta}\!\left[\alpha''+(\alpha')^{2}-\alpha'\beta'+\frac{2}{r}\,(\alpha'-\beta')\right]
-\frac{2}{r^{2}}\!\left(1-e^{-2\beta}\right).
\label{eq:R_scalar}
\end{equation}

Hence the Einstein tensor $G_{\mu\nu}=R_{\mu\nu}-\tfrac12 g_{\mu\nu}R$ gives
\begin{equation}
\boxed{\,G_{tt}=\frac{e^{2(\alpha-\beta)}}{r^{2}}\Big(2r\,\beta'-1+e^{2\beta}\Big)\,}.
\label{eq:Gtt_final}
\end{equation}
For completeness, the other nonzero covariant components are
\begin{equation}
\boxed{\,G_{rr}=\frac{1}{r^{2}}\Big(2r\,\alpha'+1-e^{2\beta}\Big)\,},\qquad
\boxed{\,G_{\theta\theta}=e^{-2\beta}\!\left[\alpha''+(\alpha')^{2}-\alpha'\beta'+\frac{\alpha'-\beta'}{r}\right]r^{2}\,},
\label{eq:GrGth_cov}
\end{equation}
and by spherical symmetry
\begin{equation}
G_{\phi\phi}=\sin^{2}\theta\,G_{\theta\theta}.
\end{equation}

\paragraph{Mixed components.}
It is often convenient to work with $G^{\mu}{}_{\nu}=g^{\mu\lambda}G_{\lambda\nu}$. Using $g^{tt}=e^{-2\alpha}$, $g^{rr}=-e^{-2\beta}$, and $g^{\theta\theta}=-1/r^{2}$, one finds
\begin{align}
G^{t}{}_{t} &= \frac{e^{-2\beta}}{r^{2}}\Bigl(2r\,\beta'-1+e^{2\beta}\Bigr),\label{eq:Gtt_mixed}\\
G^{r}{}_{r} &= \frac{e^{-2\beta}}{r^{2}}\Bigl(-2r\,\alpha'-1+e^{2\beta}\Bigr),\label{eq:Grr_mixed}\\
G^{\theta}{}_{\theta} &= G^{\phi}{}_{\phi}
= -\,e^{-2\beta}\!\left[\alpha''+(\alpha')^{2}-\alpha'\beta'+\frac{\alpha'-\beta'}{r}\right].
\label{eq:Gang_mixed}
\end{align}

\section{Einstein equations with a perfect fluid source}

For stellar matter we assume a perfect fluid energy--momentum tensor,
\begin{equation}
T_{\mu\nu}=(\epsilon+p)u_\mu u_\nu - p\,g_{\mu\nu},
\end{equation}
with energy density $\epsilon(r)$, pressure $p(r)$, and four--velocity 
$u^\mu=(e^{-\alpha},0,0,0)$ in the fluid rest frame. 
Normalization $u^\mu u_\mu=1$ then implies $u_\mu=(e^{\alpha},0,0,0)$ 
for the metric \eqref{eq:sss_alpha_beta}.  
In mixed form the components take the simple diagonal structure
\begin{equation}
T^{\mu}{}_{\nu}=\mathrm{diag}\!\big(\epsilon,\,-p,\,-p,\,-p\big).
\end{equation}

\subsection*{Independent Einstein equations}

Equating \eqref{eq:Gtt_mixed}--\eqref{eq:Gang_mixed} to $8\pi T^{\mu}{}_{\nu}$ 
via \eqref{eq:einstein_equation} yields three independent equations:
\begin{align}
\frac{e^{-2\beta}}{r^{2}}\Bigl(2r\,\beta'-1+e^{2\beta}\Bigr) &= 8\pi\,\epsilon, 
\label{eq:Ein_tt}\\[6pt]
\frac{e^{-2\beta}}{r^{2}}\Bigl(-2r\,\alpha'-1+e^{2\beta}\Bigr) &= -\,8\pi\,p,
\label{eq:Ein_rr}\\[6pt]
e^{-2\beta}\!\left[\alpha''+(\alpha')^{2}-\alpha'\beta'
   +\tfrac{1}{r}(\alpha'-\beta')\right] &= -\,8\pi\,p.
\label{eq:Ein_thth}
\end{align}
By spherical symmetry $G^{\phi}{}_{\phi}=G^{\theta}{}_{\theta}$, 
so the $\phi\phi$ component provides no new equation beyond \eqref{eq:Ein_thth}.  
Equation \eqref{eq:Ein_tt} relates geometry to the energy density, 
\eqref{eq:Ein_rr} encodes the radial pressure balance, 
and \eqref{eq:Ein_thth} enforces equality of radial and tangential pressures.

\section{Mass function}

It is convenient to define the mass function as the areal--radius invariant
\begin{equation}
m(r)\equiv \frac{r}{2}\Bigl(1-e^{-2\beta(r)}\Bigr),
\end{equation}
which directly measures the deviation of $g_{rr}$ from flat space and reduces to the 
Schwarzschild mass in the exterior vacuum region. Physically, $m(r)$ represents the total mass--energy 
contained within radius $r$. Inverting the definition gives
\begin{equation}
e^{2\beta(r)}=\Bigl(1-\frac{2m(r)}{r}\Bigr)^{-1}.
\end{equation}

Differentiating $m(r)$ with respect to $r$ yields
\begin{equation}
\frac{dm}{dr}=\frac{1}{2}\Bigl(1+e^{-2\beta}\,[\,2r\beta'-1\,]\Bigr).
\label{eq:dm_dr_from_def}
\end{equation}
Comparing \eqref{eq:dm_dr_from_def} with the $tt$--equation \eqref{eq:Ein_tt} immediately leads to
\begin{equation}
\boxed{\;\frac{dm}{dr}=4\pi r^{2}\,\epsilon(r)\;,}
\label{eq:mass_equation}
\end{equation}
with the regularity condition $m(0)=0$ at the center. 
For $r>R$, outside the stellar surface, $m(r)$ is constant and equal to the total gravitational 
mass of the star.

\section{Hydrostatic equilibrium: the TOV pressure equation}

In this section we derive the differential equation governing the pressure profile $p(r)$ 
in a static, spherically symmetric star. The result expresses local force balance between the 
inward pull of gravity and the outward pressure gradient. The derivation rests on two ingredients:
\begin{enumerate}
\item An expression for the radial metric potential $\alpha'(r)$ obtained from the 
$rr$ component of Einstein’s equations [Eq.~\eqref{eq:Ein_rr}]. 
This connects the gradient of the redshift factor $e^{\alpha}$ to local matter variables.
\item Local energy--momentum conservation, $\nabla_{\mu}T^{\mu}{}_{\nu}=0$, applied to 
the perfect fluid energy--momentum tensor. 
This provides a relation between the pressure gradient and the metric function $\alpha'(r)$.
\end{enumerate}

Combining these two relations yields the hydrostatic equilibrium equation, usually referred 
to as the Tolman--Oppenheimer--Volkoff (TOV) equation. We will keep all intermediate steps 
explicit in order to make clear how general relativity modifies the familiar Newtonian result.

\subsection*{Solving the rr equation for the metric potential derivative}

From Eq.~\eqref{eq:Ein_rr}, the mixed $rr$ component of Einstein’s equations reads
\begin{equation}
\frac{e^{-2\beta}}{r^{2}}\Bigl(-2r\,\alpha'-1+e^{2\beta}\Bigr)=-\,8\pi\,p(r).
\label{eq:Ein_rr_repeat}
\end{equation}
Multiplying by $r^{2}e^{2\beta}$ eliminates the prefactor and gives
\begin{align}
-2r\,\alpha' - 1 + e^{2\beta} &= -8\pi p\, r^{2} e^{2\beta},\nonumber\\
2r\,\alpha' &= e^{2\beta}\!\left(1+8\pi p\,r^{2}\right) - 1.
\end{align}

Introducing the Misner--Sharp mass function 
$m(r)=\tfrac{r}{2}\bigl(1-e^{-2\beta}\bigr)$, one has
$e^{2\beta}=(1-2m/r)^{-1}$. Substitution yields
\begin{align}
2r\,\alpha' &= \frac{1+8\pi p\,r^{2}}{1-2m/r}-1 \nonumber\\
            &= \frac{1+8\pi p\,r^{2}-(1-2m/r)}{1-2m/r}\nonumber\\
            &= \frac{8\pi p\,r^{2}+2m/r}{1-2m/r}.
\end{align}
Hence
\begin{equation}
\boxed{\;
\alpha'(r)=\frac{m(r)+4\pi r^{3}p(r)}{r^{2}\!\bigl(1-2m(r)/r\bigr)}
=\frac{m(r)+4\pi r^{3}p(r)}{r\,[\,r-2m(r)\,]}\,.
\;}
\label{eq:alpha_prime}
\end{equation}

This relation shows that $\alpha(r)$ plays the role of a relativistic gravitational potential. 
The combination $m(r)+4\pi r^{3}p(r)$ acts as the effective gravitational mass, 
demonstrating that in general relativity pressure contributes to the active source of gravity 
along with energy density.

\subsection*{Energy–momentum conservation in the radial direction}

For a perfect fluid with $T^{\mu}{}_{\nu}=\mathrm{diag}(\epsilon,-p,-p,-p)$ 
in the static frame and with metric \eqref{eq:sss_alpha_beta}, 
the only nontrivial conservation law is the radial one. 
Evaluating the covariant derivative gives
\begin{equation}
\nabla_{\mu}T^{\mu}{}_{r}
=\partial_{r}T^{r}{}_{r}
+\Gamma^{\mu}{}_{\mu r}\,T^{r}{}_{r}
-\Gamma^{\lambda}{}_{\mu r}\,T^{\mu}{}_{\lambda}
= -\,p' - (\epsilon+p)\,\alpha' = 0,
\label{eq:fluid_cons_component}
\end{equation}
where primes denote derivatives with respect to $r$, and the relevant Christoffel symbols 
are those listed in \eqref{eq:Gamma_subset}.

Equivalently, one can project the general conservation law $\nabla_{\mu}T^{\mu\nu}=0$ 
orthogonal to the four--velocity to obtain the relativistic Euler equation
\[
(\epsilon+p)\,a_{\nu}=-\nabla_{\nu}p,\qquad 
a_{\nu}=u^{\mu}\nabla_{\mu}u_{\nu}.
\]
In our static metric $a_{r}=\alpha'$, which reproduces 
Eq.~\eqref{eq:fluid_cons_component}.

Equation \eqref{eq:fluid_cons_component} encodes local force balance: 
the outward pressure gradient $-p'$ counters the inward pull of gravity 
through the potential gradient $\alpha'$, weighted by the inertial mass density $(\epsilon+p)$.  
Rearranging gives
\begin{equation}
\frac{dp}{dr}=-\,(\epsilon+p)\,\alpha'(r).
\label{eq:preTOV}
\end{equation}

\subsection*{The TOV pressure equation}

Substituting \eqref{eq:alpha_prime} into \eqref{eq:preTOV} gives the Tolman--Oppenheimer--Volkoff equation:
\begin{equation}
\boxed{\;
\frac{dp}{dr}
= -\,(\epsilon+p)\,\frac{m(r)+4\pi r^{3}p(r)}{r\,[\,r-2m(r)\,]}\,.
\;}
\label{eq:TOV_pressure}
\end{equation}
Together with the mass equation
\begin{equation}
\frac{dm}{dr}=4\pi r^{2}\epsilon(r),\qquad m(0)=0,
\label{eq:TOV_mass_repeat}
\end{equation}
and an equation of state $p=p(\epsilon)$, Eqs.~\eqref{eq:TOV_pressure}--\eqref{eq:TOV_mass_repeat} form a closed system of ODEs
for $(p(r),m(r))$.

\paragraph{Physical interpretation.}
The factor $(\epsilon+p)$ reflects relativistic inertia: pressure adds to the effective weight of the fluid.
The term $m(r)$ is the enclosed gravitational mass, while $4\pi r^{3}p$ encodes the additional self--gravity of pressure.
The geometric denominator $r(r-2m)$ accounts for spatial curvature and redshift effects. In the weak--field, low--pressure limit
($2m/r\ll 1$ and $p\ll\epsilon$), \eqref{eq:TOV_pressure} reduces to Newtonian hydrostatic balance,
\begin{equation}
\frac{dp}{dr}\approx -\,\rho(r)\,\frac{G\,m(r)}{r^{2}},
\qquad \text{with } \epsilon\approx \rho c^{2}\ \text{(restore $G,c$ as needed; here $G=c=1$).}
\end{equation}

\paragraph{Boundary data and matching.}
At the center, regularity imposes finite $\epsilon(0)$ and $p(0)$ and $\alpha'(0)=0$; moreover
$m(r)\sim \tfrac{4\pi}{3}\epsilon(0)\,r^{3}$ as $r\to 0$, so $p'(0)=0$ follows from \eqref{eq:TOV_pressure}.
The stellar radius $R$ is defined by $p(R)=0$. Matching to the exterior Schwarzschild solution sets
$m(R)=M$ and $e^{2\alpha(R)}=1-2M/R$, with the overall normalization of $t$ fixed by asymptotic flatness.
