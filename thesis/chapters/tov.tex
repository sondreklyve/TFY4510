\chapter{Tolman–Oppenheimer–Volkoff Equations}
\label{chap:tov}

\section{Introduction}

The structure of compact stars cannot always be captured within the framework of Newtonian gravity. 
For white dwarfs, where electrons are relativistic but the gravitational field is relatively weak, 
Newtonian gravity combined with special relativity provides an adequate description. 
In contrast, neutron stars and hypothetical hybrid stars reach densities comparable to the nuclear 
saturation density $\rho_{0}$, which is the characteristic density inside atomic nuclei. 
At such densities, the curvature of spacetime becomes so strong that general relativity is indispensable. 
It provides the natural framework to describe how matter curves spacetime and, conversely, 
how this curved geometry governs the equilibrium of stellar matter.

In this chapter we derive the equations governing static, spherically symmetric configurations of 
perfect fluids in general relativity, known as the Tolman–Oppenheimer–Volkoff (TOV) equations \cite{oppenheimer1939}.
They generalize the Newtonian hydrostatic equilibrium equation to curved spacetime. 
Hence, the TOV equations form the central tool for studying compact stars such as neutron stars 
and hybrid stars with exotic cores, while also providing a relativistic extension of the classic treatment 
that suffices for white dwarfs. They link microscopic physics, encoded in the equation of state of dense matter, 
to macroscopic observables such as stellar masses and radii.

Our starting point will be Einstein's field equations, together with the assumption of spherical 
symmetry and a perfect-fluid energy–momentum tensor. From these ingredients we obtain a set of 
coupled ordinary differential equations relating the pressure, energy density, and enclosed 
gravitational mass as functions of the radial coordinate. To close the system, one must specify 
an equation of state, thereby providing the crucial connection between microphysics and 
astrophysical observables.

Historically, these equations were first formulated independently by Tolman \cite{tolman1939} 
and by Oppenheimer and Volkoff. Their pioneering work laid the foundation 
for modern neutron star astrophysics, showing that general relativity predicts a maximum mass for 
compact stars beyond which no stable configuration exists. This insight still guides research into 
the ultimate fate of dense matter and the physics of gravitational collapse.

The derivation presented here follows the standard approach used in the literature,
and closely follows the steps from Carroll’s \emph{Spacetime and Geometry}.
\cite{oppenheimer1939, tolman1939, carroll}. 

\section{Metric ansatz and coordinates}
We now restrict outselves to static, spherically symmetric stellar configurations. 
By Birkhoff’s theorem, the most general spherically symmetric vacuum solution of 
Einstein’s equations is the Schwarzschild metric \cite{carroll}. Inside the star, where matter is present, 
the metric must reduce continuously to the Schwarzschild form at the stellar surface. 
It is therefore natural to adopt Schwarzschild–like (curvature) coordinates and write the 
line element as
\begin{equation}
ds^{2} = A(r)\, dt^{2} - B(r)\, dr^{2} - r^{2}\,(d\theta^{2}+\sin^{2}\theta\, d\phi^{2}),
\label{eq:metric_ansatz}
\end{equation}
where $A(r)$ and $B(r)$ are functions of the radial coordinate only. 
The assumption of staticity ensures that no cross terms such as $dr\,dt$ appear, 
while spherical symmetry fixes the angular part to the standard 
$r^{2} d\Omega^{2}$ form.

\section{Explicit computation of Einstein tensor components}

It is convenient to rewrite the metric functions by
\begin{equation}
A(r)=e^{2\alpha(r)},\qquad B(r)=e^{2\beta(r)},
\end{equation}
so that the line element becomes
\begin{equation}
ds^{2}=e^{2\alpha(r)}\,dt^{2}-e^{2\beta(r)}\,dr^{2}-r^{2}\,\bigl(d\theta^{2}+\sin^{2}\theta\,d\phi^{2}\bigr).
\label{eq:sss_alpha_beta}
\end{equation}
Using the definition of the Christoffel symbols from Eq.~\eqref{eq:christoffel_def}, one finds the nonvanishing components
\begin{align}
\Gamma^{t}{}_{tr} &= \alpha', &
\Gamma^{r}{}_{tt} &= \alpha' e^{2(\alpha-\beta)}, &
\Gamma^{r}{}_{rr} &= \beta', \nonumber\\[4pt]
\Gamma^{r}{}_{\theta\theta} &= -r\,e^{-2\beta}, &
\Gamma^{r}{}_{\phi\phi} &= -r\,e^{-2\beta}\sin^{2}\theta, &
\Gamma^{\theta}{}_{r\theta} &= \Gamma^{\phi}{}_{r\phi}=\tfrac{1}{r}, \nonumber\\[2pt]
\Gamma^{\theta}{}_{\phi\phi} &= -\sin\theta\cos\theta, &
\Gamma^{\phi}{}_{\theta\phi} &= \cot\theta,
\label{eq:Gamma_subset}
\end{align}
where primes denote derivatives with respect to $r$.
From the definition of the Ricci tensor in Eq.~\eqref{eq:ricci_def},
stationarity ($\partial_{t}=0$) and spherical symmetry simplify the components. 
As an example, for $R_{tt}$ one finds
\begin{align}
R_{tt}
&=\partial_{r}\Gamma^{r}{}_{tt}
+\Gamma^{r}{}_{tt}\!\left(\Gamma^{t}{}_{rt}+\Gamma^{r}{}_{rr}+\Gamma^{\theta}{}_{r\theta}+\Gamma^{\phi}{}_{r\phi}\right)
-\Gamma^{t}{}_{tr}\Gamma^{r}{}_{tt}\nonumber\\[2pt]
&=e^{2(\alpha-\beta)}\!\left[\alpha''+(\alpha')^{2}-\alpha'\beta'+\frac{2\alpha'}{r}\right].
\label{eq:Rtt_result}
\end{align}
In addition to $R_{tt}$, one must also evaluate the radial and angular components 
($R_{rr}$, $R_{\theta\theta}$ and $R_{\phi\phi}$). 
The Ricci scalar is then obtained by contraction $R = g^{\mu\nu}R_{\mu\nu}$, which yields
\begin{equation}
R
=2e^{-2\beta}\!\left[\alpha''+(\alpha')^{2}-\alpha'\beta'+\frac{2}{r}\,(\alpha'-\beta')\right]
-\frac{2}{r^{2}}\!\left(1-e^{-2\beta}\right).
\label{eq:R_scalar}
\end{equation}
Finally, using the definition of the Einstein tensor in
Eq.~\eqref{eq:einstein_tensor_def}, the nonvanishing components are found to be
\begin{align}
G_{tt} &= \frac{e^{2(\alpha-\beta)}}{r^{2}}\Big(2r\,\beta'-1+e^{2\beta}\Big), \label{eq:Gtt_final}\\[6pt]
G_{rr} &= \frac{1}{r^{2}}\Big(2r\,\alpha'+1-e^{2\beta}\Big), \nonumber\\[6pt]
G_{\theta\theta} &= e^{-2\beta}\!\left[\alpha''+(\alpha')^{2}-\alpha'\beta'
+\frac{\alpha'-\beta'}{r}\right]r^{2}, \label{eq:GrGth_cov}\\[6pt]
G_{\phi\phi} &= \sin^{2}\theta\,G_{\theta\theta}. \nonumber
\end{align}


\paragraph{Mixed components.}
It is often convenient to work with $G^{\mu}{}_{\nu}=g^{\mu\lambda}G_{\lambda\nu}$. 
For the metric \eqref{eq:sss_alpha_beta} these read
\begin{align}
G^{t}{}_{t} &= \frac{e^{-2\beta}}{r^{2}}\Bigl(2r\,\beta'-1+e^{2\beta}\Bigr),\label{eq:Gtt_mixed}\\
G^{r}{}_{r} &= \frac{e^{-2\beta}}{r^{2}}\Bigl(-2r\,\alpha'-1+e^{2\beta}\Bigr),\label{eq:Grr_mixed}\\
G^{\theta}{}_{\theta} &= G^{\phi}{}_{\phi}
= -\,e^{-2\beta}\!\left[\alpha''+(\alpha')^{2}-\alpha'\beta'+\frac{\alpha'-\beta'}{r}\right].
\label{eq:Gang_mixed}
\end{align}

\section{Einstein equations with a perfect fluid source}

In relativistic astrophysics, stellar matter is commonly modeled as a 
perfect fluid. A perfect fluid is an idealized medium that has 
no viscosity, no heat conduction, and is completely characterized by 
its local rest--frame energy density $\epsilon$ and isotropic pressure $P$. 
This means that, in the rest frame of the fluid, the stress is the same 
in all spatial directions and there are no dissipative processes such 
as friction or heat flow. While real nuclear matter may exhibit more 
complicated transport properties, the perfect fluid approximation 
captures the essential macroscopic features of dense stellar matter 
and provides a tractable starting point for deriving equilibrium equations.

Accordingly, we assume the perfect-fluid form \eqref{eq:perfect_fluid_Tmn}
with energy density $\epsilon(r)$, pressure $P(r)$, and four–velocity
$u^\mu=(e^{-\alpha},0,0,0)$ in the fluid rest frame.
Normalization $u^\mu u_\mu=1$ then implies $u_\mu=(e^{\alpha},0,0,0)$ 
for the metric \eqref{eq:sss_alpha_beta}.  
In mixed form the components take the simple diagonal structure
\begin{equation}
T^{\mu}{}_{\nu}=\mathrm{diag}\!\big(\epsilon,\,-P,\,-P,\,-P\big).
\end{equation}

\subsection*{Independent Einstein equations}

Equating \eqref{eq:Gtt_mixed}--\eqref{eq:Gang_mixed} to $8\pi T^{\mu}{}_{\nu}$ 
via \eqref{eq:einstein_equation} yields three independent equations,
\begin{align}
\frac{e^{-2\beta}}{r^{2}}\Bigl(2r\,\beta'-1+e^{2\beta}\Bigr) &= 8\pi\,\epsilon, 
\label{eq:Ein_tt}\\[6pt]
\frac{e^{-2\beta}}{r^{2}}\Bigl(-2r\,\alpha'-1+e^{2\beta}\Bigr) &= -\,8\pi\,P,
\label{eq:Ein_rr}\\[6pt]
e^{-2\beta}\!\left[\alpha''+(\alpha')^{2}-\alpha'\beta'
   +\tfrac{1}{r}(\alpha'-\beta')\right] &= -\,8\pi\,P.
\label{eq:Ein_thth}
\end{align}
By spherical symmetry $G^{\phi}{}_{\phi}=G^{\theta}{}_{\theta}$, 
so the $\phi\phi$ component provides no new equation beyond \eqref{eq:Ein_thth}.  

Each of these equations has a clear physical interpretation.  
Equation \eqref{eq:Ein_tt} links the $tt$ component of spacetime curvature 
to the local energy density $\epsilon(r)$; it plays the role of a relativistic 
generalization of Poisson’s equation in Newtonian gravity.  
Equation \eqref{eq:Ein_rr} involves the $rr$ component and connects the radial 
metric function $\alpha(r)$ to the local pressure $P(r)$. 
This equation expresses how the pressure gradient balances the inward pull of gravity 
to maintain hydrostatic equilibrium.  
Finally, Eq.~\eqref{eq:Ein_thth} comes from the angular components of the Einstein tensor. 
It requires that the pressure entering the field equations is the same in the radial 
and tangential directions, reflecting the assumption of an isotropic perfect fluid.  

Taken together, these equations encode the interplay between energy density, pressure, 
and spacetime curvature that governs the internal structure of relativistic stars.

\section{Mass function}

It is convenient to introduce the Misner--Sharp mass function, defined as
\begin{equation}
m(r)\equiv \frac{r}{2}\Bigl(1-e^{-2\beta(r)}\Bigr).
\label{eq:ms_mass}
\end{equation}
This quantity measures the deviation of $g_{rr}$ from flat space and reduces to the 
Schwarzschild mass in the exterior vacuum region. Inverting the definition gives
\begin{equation}
e^{2\beta(r)}=\Bigl(1-\frac{2m(r)}{r}\Bigr)^{-1}.
\label{eq:ms_mass_inv}
\end{equation}
Differentiating $m(r)$ with respect to $r$ yields
\begin{equation}
\frac{dm}{dr}=\frac{1}{2}\Bigl[1+e^{-2\beta}\,(\,2r\beta'-1\,)\Bigr].
\label{eq:dm_dr_from_def}
\end{equation}
Comparing \eqref{eq:dm_dr_from_def} with the $tt$--equation \eqref{eq:Ein_tt} immediately leads to
\begin{equation}
\frac{dm}{dr}=4\pi r^{2}\,\epsilon(r), \qquad m(0)=0.
\label{eq:mass_equation}
\end{equation}
For $r>R$, outside the stellar surface, $m(r)$ is constant and equal to the total gravitational 
mass of the star.  

At first sight, Eq.~\eqref{eq:mass_equation} resembles the Newtonian result 
$m(r)=\int_0^r 4\pi r'^2 \rho(r')\,dr'$, i.e.\ the rest--mass integrated over the coordinate volume. 
In general relativity, however, the proper spatial volume element contains an extra factor from the 
metric, $dV = 4\pi r'^2 e^{\beta(r')}dr'$, so the expression above is not a direct proper--volume 
integral of the energy density. Instead, $m(r)$ is defined geometrically: it is the Misner--Sharp 
mass, introduced in Eq.~\eqref{eq:ms_mass}, which enters the metric function $B(r)$ and governs the 
exterior Schwarzschild limit. The difference between the naive proper integral and $m(r)$ accounts 
for the star’s gravitational binding energy, so $m(r)$ should be interpreted as the total enclosed 
gravitational mass rather than a simple sum of local rest masses.


\section{Hydrostatic equilibrium: the TOV pressure equation}

In this section we derive the differential equation governing the pressure profile $P(r)$ 
in a static, spherically symmetric star. The result expresses local force balance between the 
inward pull of gravity and the outward pressure gradient. The derivation rests on two ingredients:
\begin{enumerate}
\item An expression for the radial metric potential $\alpha'(r)$ obtained from the 
$rr$ component of Einstein’s equations Eq.~\eqref{eq:Ein_rr}. 
This connects the gradient of the redshift factor $e^{\alpha}$ to local matter variables.
\item Local energy--momentum conservation, $\nabla_{\mu}T^{\mu}{}_{\nu}=0$, applied to 
the perfect fluid energy--momentum tensor. 
This provides a relation between the pressure gradient and the metric function $\alpha'(r)$.
\end{enumerate}
Combining these two relations yields the hydrostatic equilibrium equation, usually referred 
to as the Tolman--Oppenheimer--Volkoff equation. We will keep all intermediate steps 
explicit in order to make clear how general relativity modifies the familiar Newtonian result.

\subsection*{Solving the rr equation for the metric potential derivative}

From Eq.~\eqref{eq:Ein_rr} we obtain the $rr$ component in terms of the radial pressure profile,
\begin{equation}
\frac{e^{-2\beta}}{r^{2}}\Bigl(-2r\,\alpha'-1+e^{2\beta}\Bigr)=-\,8\pi\,P(r).
\label{eq:Ein_rr_repeat}
\end{equation}
Multiplying by $r^{2}e^{2\beta}$ eliminates the prefactor and, rearranging, we get 
\begin{equation}
2r\,\alpha' = e^{2\beta}\!\left(1+8\pi P\,r^{2}\right) - 1.
\end{equation}
Inserting the inverted definition of the Misner--Sharp mass, given in Eq.~\eqref{eq:ms_mass_inv}, we obtain
\begin{align}
2r\,\alpha' &= \frac{1+8\pi P\,r^{2}}{1-2m/r}-1 \nonumber\\
            &= \frac{1+8\pi P\,r^{2}-(1-2m/r)}{1-2m/r}\nonumber\\
            &= \frac{2m/r+8\pi P\,r^{2}}{1-2m/r}.
\end{align}
Hence
\begin{equation}
\boxed{\;
\alpha'(r)=\frac{m(r)+4\pi r^{3}P(r)}{r\,[\,r-2m(r)\,]}\,.
\;}
\label{eq:alpha_prime}
\end{equation}
This equation determines how the metric function $\alpha(r)$ varies with radius. 
Since $g_{tt}=e^{2\alpha(r)}$, $\alpha(r)$ controls the $tt$--component of the metric and 
therefore the redshift of clocks in the gravitational field. 
In the weak–field limit $r \gg 2m(r)$ and for small pressures, the denominator reduces to $r^{2}$ 
and the equation approaches $\alpha'(r)\simeq m(r)/r^{2}$, 
which is the familiar Newtonian form of the gravitational potential gradient.

\subsection*{Energy–momentum conservation in the radial direction}

For a perfect fluid with $T^{\mu}{}_{\nu}=\mathrm{diag}(\epsilon,-P,-P,-P)$ 
in the static frame and with metric \eqref{eq:sss_alpha_beta}, 
the only nontrivial conservation law is the radial one. 
Evaluating the covariant derivative gives
\begin{equation}
\nabla_{\mu}T^{\mu}{}_{r}
=\partial_{r}T^{r}{}_{r}
+\Gamma^{\mu}{}_{\mu r}\,T^{r}{}_{r}
-\Gamma^{\lambda}{}_{\mu r}\,T^{\mu}{}_{\lambda}
= -\,p' - (\epsilon+p)\,\alpha' = 0,
\label{eq:fluid_cons_component}
\end{equation}
where the relevant Christoffel symbols 
are those listed in \eqref{eq:Gamma_subset}.

This relation expresses local energy–momentum conservation in the radial direction: 
the change of pressure with radius is tied to the gradient of the metric function $\alpha(r)$. 
It can also be written in the standard form of the relativistic Euler equation,
\begin{equation}
(\epsilon+P)\,a_{\nu}=-\nabla_{\nu}p,\qquad 
a_{\nu}=u^{\mu}\nabla_{\mu}u_{\nu},
\end{equation}
where $a_{\nu}$ is the four–acceleration of the fluid. 
For our static configuration the only nonzero component is $a_{r}=\alpha'$, 
so the Euler equation reduces exactly to Eq.~\eqref{eq:fluid_cons_component}.

Equation \eqref{eq:fluid_cons_component} expresses the condition of hydrostatic 
equilibrium in general relativity: the pressure decreases outward in order to 
balance the effect of the gravitational field. 
Rearranging gives
\begin{equation}
\frac{dP}{dr}=-\,(\epsilon+P)\,\alpha'(r).
\label{eq:preTOV}
\end{equation}

\subsection*{The TOV pressure equation}

Substituting \eqref{eq:alpha_prime} into \eqref{eq:preTOV} gives the Tolman--Oppenheimer--Volkoff equation:
\begin{equation}
\boxed{\;
\frac{dP}{dr}
= -\,(\epsilon+P)\,\frac{m(r)+4\pi r^{3}P(r)}{r\,[\,r-2m(r)\,]}\,.
\;}
\label{eq:TOV_pressure}
\end{equation}
Together with the mass equation \eqref{eq:mass_equation} 
and an equation of state $P=P(\epsilon)$, these equations form a closed system of ODEs
for $(P(r),m(r))$.

\paragraph{Physical interpretation.}
Equation \eqref{eq:TOV_pressure} is the relativistic form of hydrostatic equilibrium.  
The factor $(\epsilon+P)$ shows that both energy density $\epsilon$ and pressure $P$ enter the balance 
between the pressure gradient and the gravitational field.  
The quantity $m(r)$ is the mass function introduced in Eq.~\eqref{eq:mass_equation}, and the term 
$4\pi r^{3}P$ appears directly from the Einstein field equations.  
The denominator $r(r-2m)$ reflects the effects of spacetime curvature and reduces to $r^{2}$ in the 
weak–field limit.  
For $2m/r \ll 1$ and $P \ll \epsilon$, the TOV equation reduces to the Newtonian hydrostatic balance,
\begin{equation}
\frac{dP}{dr}\approx -\,\rho(r)\,\frac{G\,m(r)}{r^{2}},
\end{equation}
where $\rho$ is the rest–mass density.  
Since $\epsilon\simeq \rho c^{2}$ in this limit, the relativistic and Newtonian forms are consistent.  

\paragraph{Boundary conditions and matching.}
At the center of the star, regularity requires that $\epsilon(0)$ and $P(0)$ are finite, and that $\alpha'(0)=0$.  
From the mass equation one finds $m(r)\sim \tfrac{4\pi}{3}\,\epsilon(0)\,r^{3}$ as $r\to 0$, 
and Eq.~\eqref{eq:TOV_pressure} then implies $P'(0)=0$.  
The stellar surface $R$ is defined by $P(R)=0$.  
Outside this radius the solution must match smoothly to the vacuum Schwarzschild metric, 
so $m(R)=M$ is the total mass and $e^{2\alpha(R)}=1-2M/R$.  
The integration constant for $\alpha$ is fixed by asymptotic flatness, 
so that $e^{2\alpha}\to 1$ as $r\to\infty$.  
The factor $(1-2M/R)$ is the Schwarzschild redshift factor, and the condition $R>2M$ 
ensures that the star is not a black hole.

\section*{Example: Uniform--density (``incompressible'') star}
\addcontentsline{toc}{section}{Example: Uniform--density (``incompressible'') star}

As an analytically solvable illustration of the TOV system, consider a star with
\begin{equation}
\epsilon(r)\equiv \epsilon_{0}=\text{const}.
\label{eq:const_density_assumption}
\end{equation}
This is \emph{not} a realistic equation of state: because $\epsilon$ does not change when $P$ changes, the adiabatic sound speed
\[
c_{s}^{2} \equiv \frac{dP}{d\epsilon}
\]
would be infinite (since $d\epsilon/dP=0$), i.e.\ acausal. We use the model only as a clean analytic foil and a code check when we later implement a numerical TOV solver.

\subsection*{Mass function and radial metric}
Integrating the mass equation \eqref{eq:mass_equation} gives
\begin{equation}
m(r)=\frac{4\pi}{3}\,\epsilon_{0}\,r^{3},\qquad
M\equiv m(R)=\frac{4\pi}{3}\,\epsilon_{0}\,R^{3}.
\label{eq:const_density_mass}
\end{equation}
From the definition \eqref{eq:ms_mass_inv} it follows that
\begin{equation}
e^{2\beta(r)}=\Bigl(1-\frac{2m(r)}{r}\Bigr)^{-1}
=\left(1-\frac{2M}{R^{3}}\,r^{2}\right)^{-1}.
\label{eq:beta_uniform}
\end{equation}
It is convenient to define
\begin{equation}
\Phi(r)\equiv \sqrt{1-\frac{2m(r)}{r}}=\sqrt{1-\frac{2M}{R^{3}}\,r^{2}},
\qquad
\Phi_{R}\equiv \Phi(R)=\sqrt{1-\frac{2M}{R}}.
\label{eq:Phi_defs}
\end{equation}

\subsection*{Pressure profile}
Combining hydrostatic balance \eqref{eq:preTOV} with \eqref{eq:alpha_prime} and the uniform density \eqref{eq:const_density_mass}, one separates variables and integrates from $r$ to $R$ using $P(R)=0$. The result is
\begin{equation}
P(r)
=\epsilon_{0}\,
\frac{\Phi_{r}-\Phi(R)}{3\,\Phi(R)-\Phi_{r}},
\label{eq:const_density_pressure}
\end{equation}
and, at the center ($\Phi(0)=1$),
\begin{equation}
P_{c}
=\epsilon_{0}\,
\frac{1-\Phi_{R}}{3\,\Phi_{R}-1}.
\label{eq:Pc_const_density}
\end{equation}

\subsection*{Compactness and the Buchdahl bound}
Define the compactness
\begin{equation}
\mathcal{C}\equiv \frac{2M}{R}.
\end{equation}
From \eqref{eq:Pc_const_density}, $P_c$ is finite for $\Phi_R\neq \frac{1}{3}$ and positive only if $3\,\Phi_R-1>0$, i.e.\ $\mathcal{C}<\frac{8}{9}$.
\begin{equation}
\Phi_{R}>\frac{1}{3}
\quad\Longleftrightarrow\quad
\mathcal{C}<\frac{8}{9}
\quad\Longleftrightarrow\quad
R>\frac{9}{4}\,M.
\label{eq:Buchdahl_bound}
\end{equation}
This inequality is the \emph{Buchdahl bound}: for static, spherically symmetric, isotropic fluids with nonincreasing $\epsilon(r)$ one must have $R\ge \frac{9}{4}M$ \cite{buchdahl1959}. The uniform--density solution makes the divergence as the bound is approached explicit: $P_{c}\to\infty$ as $R$ approaches $\frac{9}{4}M$ from above.

\subsection*{Interior time component and surface matching}
With $m(r)$ and $P(r)$ known, the remaining metric function is the redshift potential $e^{2\alpha(r)}$. 
Equation~\eqref{eq:alpha_prime} relates $\alpha'(r)$ to the matter variables; substituting the uniform–density mass profile \eqref{eq:const_density_mass} and the pressure \eqref{eq:const_density_pressure} yields a first–order equation that integrates in closed form. 
Up to an overall constant, the result is
\begin{equation}
e^{\alpha(r)}=C\,\bigl(3\,\Phi_{R}-\Phi(r)\bigr),
\label{eq:alpha_shape}
\end{equation}
where $\Phi(r)$ and $\Phi_{R}$ are defined in \eqref{eq:Phi_defs}. 
Continuity of the $tt$–component at the surface fixes $C$ via $e^{2\alpha(R)}=1-2M/R=\Phi_{R}^{2}$, hence $C=\tfrac{1}{2}$. 
Therefore
\begin{equation}
e^{2\alpha(r)}=\frac{1}{4}\,\bigl(3\,\Phi_{R}-\Phi(r)\bigr)^{2},
\label{eq:e2alpha_uniform}
\end{equation}
and the interior line element is
\begin{equation}
ds^{2}
=\frac{1}{4}\,\bigl(3\,\Phi_{R}-\Phi(r)\bigr)^{2}\,dt^{2}
-\left(1-\frac{2M}{R^{3}}\,r^{2}\right)^{-1}\!dr^{2}
-r^{2}\,d\Omega^{2},\qquad (r\le R).
\label{eq:interior_schwarzschild}
\end{equation}
At the center $r=0$ we have $\Phi(0)=1$, so $e^{\alpha(0)}=\tfrac{1}{2}(3\Phi_{R}-1)$ and $e^{2\beta(0)}=1$ are finite, and from \eqref{eq:alpha_prime} with $m(r)\sim \tfrac{4\pi}{3}\epsilon_{0}r^{3}$ it follows that $\alpha'(0)=0$. 
Thus the solution is regular at $r=0$. 
For $R>\frac{9}{4}M$ the metric functions remain smooth and finite throughout the interior; only the central pressure $P_{c}$ diverges as the Buchdahl limit $R\to \frac{9}{4}M$ is approached.

\subsection*{Summary and use}
The uniform--density model is analytically solvable but acausal (infinite $c_{s}$). It is useful as a pedagogical foil and as a numerical check: a TOV integrator supplied with \eqref{eq:const_density_assumption} should reproduce \eqref{eq:const_density_pressure} and \eqref{eq:interior_schwarzschild}, and should approach the compactness limit \eqref{eq:Buchdahl_bound} from below in $\mathcal{C}$ as the central pressure is increased.
