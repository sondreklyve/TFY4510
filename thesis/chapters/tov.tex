\chapter{Tolman–Oppenheimer–Volkoff Equations}

\section{Introduction}

The structure of highly compact stars cannot always be captured within the framework of Newtonian gravity. 
For white dwarfs, where electrons are relativistic but the gravitational field is relatively weak, 
Newtonian gravity combined with special relativity provides an adequate description. 
In contrast, neutron stars and hypothetical hybrid stars reach densities comparable to nuclear matter, 
and the curvature of spacetime becomes so strong that general relativity is indispensable. 
It provides the natural framework to describe how matter curves spacetime and, 
conversely, how this curved geometry governs the equilibrium of stellar matter.

In this chapter we derive the equations governing static, spherically symmetric configurations of 
perfect fluids in general relativity, known as the Tolman–Oppenheimer–Volkoff (TOV) equations. 
They generalize the Newtonian hydrostatic equilibrium equation to curved spacetime. 
Hence, the TOV equations form the central tool for studying compact stars such as neutron stars 
and hybrid stars with exotic cores, while also providing a relativistic extension of the classical treatment 
that suffices for white dwarfs. They link microscopic physics, encoded in the equation of state of dense matter, 
to macroscopic observables such as stellar masses and radii.

Our starting point will be Einstein's field equations, together with the assumption of spherical 
symmetry and a perfect-fluid energy–momentum tensor. From these ingredients we obtain a set of 
coupled ordinary differential equations relating the pressure, energy density, and enclosed 
gravitational mass as functions of the radial coordinate. To close the system, one must specify 
an equation of state, thereby providing the crucial connection between microphysics and 
astrophysical observables.

Historically, these equations were first formulated independently by Tolman \cite{tolman1939} 
and by Oppenheimer and Volkoff \cite{oppenheimer1939}. Their pioneering work laid the foundation 
for modern neutron star astrophysics, showing that general relativity predicts a maximum mass for 
compact stars beyond which no stable configuration exists. This insight still guides research into 
the ultimate fate of dense matter and the physics of gravitational collapse.

The derivation presented here follows the standard approach used in the literature 
\cite{oppenheimer1939, tolman1939, carroll}. 

\paragraph{Acknowledgment.} 
The organization and several intermediate steps are adapted from Carroll’s 
\emph{Spacetime and Geometry} \cite{carroll}, rewritten for the mostly minus signature.

\section{Metric ansatz and coordinates}
We now restrict outselves to static, spherically symmetric stellar configurations. 
By Birkhoff’s theorem, the most general spherically symmetric vacuum solution of 
Einstein’s equations is the Schwarzschild metric. Inside the star, where matter is present, 
the metric must reduce continuously to the Schwarzschild form at the stellar surface. 
It is therefore natural to adopt Schwarzschild–like (curvature) coordinates and write the 
line element as
\begin{equation}
ds^{2} = A(r)\, dt^{2} - B(r)\, dr^{2} - r^{2}\,(d\theta^{2}+\sin^{2}\theta\, d\phi^{2}),
\label{eq:metric_ansatz}
\end{equation}
where $A(r)$ and $B(r)$ are functions of the radial coordinate only. 
The assumption of staticity ensures that no cross terms such as $dr\,dt$ appear, 
while spherical symmetry fixes the angular part to the standard 
$r^{2} d\Omega^{2}$ form.

\section{Explicit computation of Einstein tensor components}

It is convenient to rewrite the metric functions by
\begin{equation}
A(r)=e^{2\alpha(r)},\qquad B(r)=e^{2\beta(r)},
\end{equation}
so that the line element becomes
\begin{equation}
ds^{2}=e^{2\alpha(r)}\,dt^{2}-e^{2\beta(r)}\,dr^{2}-r^{2}\,\bigl(d\theta^{2}+\sin^{2}\theta\,d\phi^{2}\bigr).
\label{eq:sss_alpha_beta}
\end{equation}
Primes denote derivatives with respect to $r$.

Using the definition of the Christoffel symbols from Eq.~\eqref{eq:christoffel_def}, one finds the nonvanishing components
\begin{align}
\Gamma^{t}{}_{tr} &= \alpha', &
\Gamma^{r}{}_{tt} &= \alpha' e^{2(\alpha-\beta)}, &
\Gamma^{r}{}_{rr} &= \beta', \nonumber\\[4pt]
\Gamma^{r}{}_{\theta\theta} &= -r\,e^{-2\beta}, &
\Gamma^{r}{}_{\phi\phi} &= -r\,e^{-2\beta}\sin^{2}\theta, &
\Gamma^{\theta}{}_{r\theta} &= \Gamma^{\phi}{}_{r\phi}=\tfrac{1}{r}, \nonumber\\[2pt]
\Gamma^{\theta}{}_{\phi\phi} &= -\sin\theta\cos\theta, &
\Gamma^{\phi}{}_{\theta\phi} &= \cot\theta,
\label{eq:Gamma_subset}
\end{align}

From the definition of the Ricci tensor in Eq.~\eqref{eq:ricci_def}, stationarity ($\partial_{t}=0$) and spherical symmetry simplify the components. For $R_{tt}$ one finds
\begin{align}
R_{tt}
&=\partial_{r}\Gamma^{r}{}_{tt}
+\Gamma^{r}{}_{tt}\!\left(\Gamma^{t}{}_{rt}+\Gamma^{r}{}_{rr}+\Gamma^{\theta}{}_{r\theta}+\Gamma^{\phi}{}_{r\phi}\right)
-\Gamma^{t}{}_{tr}\Gamma^{r}{}_{tt}\nonumber\\[2pt]
&=e^{2(\alpha-\beta)}\!\left[\alpha''+(\alpha')^{2}-\alpha'\beta'+\frac{2\alpha'}{r}\right].
\label{eq:Rtt_result}
\end{align}
Contracting gives the Ricci scalar,
\begin{equation}
R
=2e^{-2\beta}\!\left[\alpha''+(\alpha')^{2}-\alpha'\beta'+\frac{2}{r}\,(\alpha'-\beta')\right]
-\frac{2}{r^{2}}\!\left(1-e^{-2\beta}\right).
\label{eq:R_scalar}
\end{equation}

Finally, using the definition of the Einstein tensor in
Eq.~\eqref{eq:einstein_tensor_def}, the $tt$--component becomes
\begin{equation}
\boxed{\,G_{tt}=\frac{e^{2(\alpha-\beta)}}{r^{2}}\Big(2r\,\beta'-1+e^{2\beta}\Big)\,.}
\label{eq:Gtt_final}
\end{equation}
The other nonzero components are
\begin{equation}
\boxed{\,G_{rr}=\frac{1}{r^{2}}\Big(2r\,\alpha'+1-e^{2\beta}\Big)\,,}\qquad
\boxed{\,G_{\theta\theta}=e^{-2\beta}\!\left[\alpha''+(\alpha')^{2}-\alpha'\beta'+\frac{\alpha'-\beta'}{r}\right]r^{2}\,,}
\label{eq:GrGth_cov}
\end{equation}
and by spherical symmetry
\begin{equation}
G_{\phi\phi}=\sin^{2}\theta\,G_{\theta\theta}.
\end{equation}

\paragraph{Mixed components.}
It is often convenient to work with $G^{\mu}{}_{\nu}=g^{\mu\lambda}G_{\lambda\nu}$. 
For the metric \eqref{eq:sss_alpha_beta} these read
\begin{align}
G^{t}{}_{t} &= \frac{e^{-2\beta}}{r^{2}}\Bigl(2r\,\beta'-1+e^{2\beta}\Bigr),\label{eq:Gtt_mixed}\\
G^{r}{}_{r} &= \frac{e^{-2\beta}}{r^{2}}\Bigl(-2r\,\alpha'-1+e^{2\beta}\Bigr),\label{eq:Grr_mixed}\\
G^{\theta}{}_{\theta} &= G^{\phi}{}_{\phi}
= -\,e^{-2\beta}\!\left[\alpha''+(\alpha')^{2}-\alpha'\beta'+\frac{\alpha'-\beta'}{r}\right].
\label{eq:Gang_mixed}
\end{align}

\section{Einstein equations with a perfect fluid source}

In relativistic astrophysics, stellar matter is commonly modeled as a 
perfect fluid. A perfect fluid is an idealized medium that has 
no viscosity, no heat conduction, and is completely characterized by 
its local rest--frame energy density $\epsilon$ and isotropic pressure $p$. 
This means that, in the rest frame of the fluid, the stress is the same 
in all spatial directions and there are no dissipative processes such 
as friction or heat flow. While real nuclear matter may exhibit more 
complicated transport properties, the perfect fluid approximation 
captures the essential macroscopic features of dense stellar matter 
and provides a tractable starting point for deriving equilibrium equations.

Accordingly, we assume the perfect-fluid form \eqref{eq:perfect_fluid_Tmn}
with energy density $\epsilon(r)$, pressure $p(r)$, and four–velocity
$u^\mu=(e^{-\alpha},0,0,0)$ in the fluid rest frame.
Normalization $u^\mu u_\mu=1$ then implies $u_\mu=(e^{\alpha},0,0,0)$ 
for the metric \eqref{eq:sss_alpha_beta}.  
In mixed form the components take the simple diagonal structure
\begin{equation}
T^{\mu}{}_{\nu}=\mathrm{diag}\!\big(\epsilon,\,-p,\,-p,\,-p\big).
\end{equation}

\subsection*{Independent Einstein equations}

Equating \eqref{eq:Gtt_mixed}--\eqref{eq:Gang_mixed} to $8\pi T^{\mu}{}_{\nu}$ 
via \eqref{eq:einstein_equation} yields three independent equations,
\begin{align}
\frac{e^{-2\beta}}{r^{2}}\Bigl(2r\,\beta'-1+e^{2\beta}\Bigr) &= 8\pi\,\epsilon, 
\label{eq:Ein_tt}\\[6pt]
\frac{e^{-2\beta}}{r^{2}}\Bigl(-2r\,\alpha'-1+e^{2\beta}\Bigr) &= -\,8\pi\,p,
\label{eq:Ein_rr}\\[6pt]
e^{-2\beta}\!\left[\alpha''+(\alpha')^{2}-\alpha'\beta'
   +\tfrac{1}{r}(\alpha'-\beta')\right] &= -\,8\pi\,p.
\label{eq:Ein_thth}
\end{align}
By spherical symmetry $G^{\phi}{}_{\phi}=G^{\theta}{}_{\theta}$, 
so the $\phi\phi$ component provides no new equation beyond \eqref{eq:Ein_thth}.  

Each of these equations has a clear physical interpretation.  
Equation \eqref{eq:Ein_tt} links the $tt$ component of spacetime curvature 
to the local energy density $\epsilon(r)$: it plays the role of a relativistic 
generalization of Poisson’s equation in Newtonian gravity.  
Equation \eqref{eq:Ein_rr} involves the $rr$ component and connects the radial 
metric function $\alpha(r)$ to the local pressure $p(r)$. 
This equation expresses how the pressure gradient balances the inward pull of gravity 
to maintain hydrostatic equilibrium.  
Finally, Eq.~\eqref{eq:Ein_thth} comes from the angular components of the Einstein tensor. 
It ensures that the pressure is isotropic, i.e.\ that the stress measured in the radial direction 
is the same as in the tangential directions, consistent with the perfect fluid assumption.  

Taken together, these equations encode the interplay between energy density, pressure, 
and spacetime curvature that governs the internal structure of relativistic stars.

\section{Mass function}

It is convenient to introduce the Misner--Sharp mass function, defined as
\begin{equation}
m(r)\equiv \frac{r}{2}\Bigl(1-e^{-2\beta(r)}\Bigr).
\label{eq:ms_mass}
\end{equation}
This quantity measures the deviation of $g_{rr}$ from flat space and reduces to the 
Schwarzschild mass in the exterior vacuum region. Physically, $m(r)$ represents the total 
mass--energy contained within a sphere of areal radius $r$. Inverting the definition gives
\begin{equation}
e^{2\beta(r)}=\Bigl(1-\frac{2m(r)}{r}\Bigr)^{-1}.
\label{eq:ms_mass_inv}
\end{equation}
Differentiating $m(r)$ with respect to $r$ yields
\begin{equation}
\frac{dm}{dr}=\frac{1}{2}\Bigl(1+e^{-2\beta}\,[\,2r\beta'-1\,]\Bigr).
\label{eq:dm_dr_from_def}
\end{equation}
Comparing \eqref{eq:dm_dr_from_def} with the $tt$--equation \eqref{eq:Ein_tt} immediately leads to
\begin{equation}
\boxed{\;\frac{dm}{dr}=4\pi r^{2}\,\epsilon(r)\;,}
\label{eq:mass_equation}
\end{equation}
with the regularity condition $m(0)=0$ at the center. 
For $r>R$, outside the stellar surface, $m(r)$ is constant and equal to the total gravitational 
mass of the star.

It is worth emphasizing the meaning of Eq.~\eqref{eq:mass_equation} in general relativity. 
In Newtonian gravity the enclosed mass is obtained by integrating the rest--mass density 
over the Euclidean volume element, $m(r)=\int_0^r 4\pi r'^2\rho(r')\,dr'$. 
In curved spacetime, the proper volume element carries an extra factor from the metric, 
$dV = 4\pi r'^2 e^{\beta(r')}dr'$, so one might expect an additional Jacobian factor 
to appear in the integral of the energy density. However, the function $m(r)$ here is not 
defined as a naive proper--volume integral. It is the Misner--Sharp mass, introduced in 
Eq.~\eqref{eq:ms_mass}, a quasi--local quantity defined geometrically from the metric itself. 
Because of this definition, the differential relation 
\eqref{eq:mass_equation} follows directly from the Einstein equations, and the factor 
$e^{\beta}$ is already encoded in the geometry. Thus the equation should be interpreted as: 
the rate of change of the enclosed gravitational mass with respect to radius is proportional 
to the local energy density times the coordinate--space volume element $4\pi r^2dr$. 
This highlights a subtle but important distinction between Newtonian mass integrals and the 
general relativistic notion of quasi--local mass.

\section{Hydrostatic equilibrium: the TOV pressure equation}

In this section we derive the differential equation governing the pressure profile $p(r)$ 
in a static, spherically symmetric star. The result expresses local force balance between the 
inward pull of gravity and the outward pressure gradient. The derivation rests on two ingredients:
\begin{enumerate}
\item An expression for the radial metric potential $\alpha'(r)$ obtained from the 
$rr$ component of Einstein’s equations Eq.~\eqref{eq:Ein_rr}. 
This connects the gradient of the redshift factor $e^{\alpha}$ to local matter variables.
\item Local energy--momentum conservation, $\nabla_{\mu}T^{\mu}{}_{\nu}=0$, applied to 
the perfect fluid energy--momentum tensor. 
This provides a relation between the pressure gradient and the metric function $\alpha'(r)$.
\end{enumerate}
Combining these two relations yields the hydrostatic equilibrium equation, usually referred 
to as the Tolman--Oppenheimer--Volkoff (TOV) equation. We will keep all intermediate steps 
explicit in order to make clear how general relativity modifies the familiar Newtonian result.

\subsection*{Solving the rr equation for the metric potential derivative}

From Eq.~\eqref{eq:Ein_rr}, the $rr$ component of Einstein’s equations reads
\begin{equation}
\frac{e^{-2\beta}}{r^{2}}\Bigl(-2r\,\alpha'-1+e^{2\beta}\Bigr)=-\,8\pi\,p(r).
\label{eq:Ein_rr_repeat}
\end{equation}
Multiplying by $r^{2}e^{2\beta}$ eliminates the prefactor and, rearranging, we get 
\begin{equation}
2r\,\alpha' = e^{2\beta}\!\left(1+8\pi p\,r^{2}\right) - 1.
\end{equation}
Inserting the inverted definition of the Misner--Sharp mass, given in Eq.~\eqref{eq:ms_mass_inv}, we obtain
\begin{align}
2r\,\alpha' &= \frac{1+8\pi p\,r^{2}}{1-2m/r}-1 \nonumber\\
            &= \frac{1+8\pi p\,r^{2}-(1-2m/r)}{1-2m/r}\nonumber\\
            &= \frac{8\pi p\,r^{2}+2m/r}{1-2m/r}.
\end{align}
Hence
\begin{equation}
\boxed{\;
\alpha'(r)=\frac{m(r)+4\pi r^{3}p(r)}{r^{2}\!\bigl(1-2m(r)/r\bigr)}
=\frac{m(r)+4\pi r^{3}p(r)}{r\,[\,r-2m(r)\,]}\,.
\;}
\label{eq:alpha_prime}
\end{equation}

This relation shows that $\alpha(r)$ plays the role of a relativistic gravitational potential. 
The combination $m(r)+4\pi r^{3}p(r)$ acts as the effective gravitational mass, 
demonstrating that in general relativity pressure contributes to the active source of gravity 
along with energy density.

\subsection*{Energy–momentum conservation in the radial direction}

For a perfect fluid with $T^{\mu}{}_{\nu}=\mathrm{diag}(\epsilon,-p,-p,-p)$ 
in the static frame and with metric \eqref{eq:sss_alpha_beta}, 
the only nontrivial conservation law is the radial one. 
Evaluating the covariant derivative gives
\begin{equation}
\nabla_{\mu}T^{\mu}{}_{r}
=\partial_{r}T^{r}{}_{r}
+\Gamma^{\mu}{}_{\mu r}\,T^{r}{}_{r}
-\Gamma^{\lambda}{}_{\mu r}\,T^{\mu}{}_{\lambda}
= -\,p' - (\epsilon+p)\,\alpha' = 0,
\label{eq:fluid_cons_component}
\end{equation}
where primes denote derivatives with respect to $r$, and the relevant Christoffel symbols 
are those listed in \eqref{eq:Gamma_subset}.

Equivalently, one can project the general conservation law $\nabla_{\mu}T^{\mu\nu}=0$ 
orthogonal to the four--velocity to obtain the relativistic Euler equation
\[
(\epsilon+p)\,a_{\nu}=-\nabla_{\nu}p,\qquad 
a_{\nu}=u^{\mu}\nabla_{\mu}u_{\nu}.
\]
In our static metric $a_{r}=\alpha'$, which reproduces 
Eq.~\eqref{eq:fluid_cons_component}.

Equation \eqref{eq:fluid_cons_component} encodes local force balance: 
the outward pressure gradient $-p'$ counters the inward pull of gravity 
through the potential gradient $\alpha'$, weighted by the inertial mass density $(\epsilon+p)$.  
Rearranging gives
\begin{equation}
\frac{dp}{dr}=-\,(\epsilon+p)\,\alpha'(r).
\label{eq:preTOV}
\end{equation}

\subsection*{The TOV pressure equation}

Substituting \eqref{eq:alpha_prime} into \eqref{eq:preTOV} gives the Tolman--Oppenheimer--Volkoff equation:
\begin{equation}
\boxed{\;
\frac{dp}{dr}
= -\,(\epsilon+p)\,\frac{m(r)+4\pi r^{3}p(r)}{r\,[\,r-2m(r)\,]}\,.
\;}
\label{eq:TOV_pressure}
\end{equation}
Together with the mass equation \eqref{eq:mass_equation} 
and an equation of state $p=p(\epsilon)$, these equations form a closed system of ODEs
for $(p(r),m(r))$.

\paragraph{Physical interpretation.}
The factor $(\epsilon+p)$ reflects the relativistic notion of inertia: not only the energy density $\epsilon$ 
but also the pressure $p$ contributes to the source of gravity. In Newtonian physics pressure merely 
transmits forces; in general relativity it also gravitates. The term $m(r)$ is the total gravitational mass 
enclosed within radius $r$, while the additional contribution $4\pi r^{3}p$ encodes the self--gravity of 
pressure. The denominator $r(r-2m)$ carries the geometric effects of spatial curvature and gravitational 
redshift. In the weak--field, low--pressure limit ($2m/r\ll 1$ and $p\ll\epsilon$), the Tolman–Oppenheimer–Volkoff 
equation reduces to the familiar Newtonian hydrostatic balance,
\begin{equation}
\frac{dp}{dr}\approx -\,\rho(r)\,\frac{G\,m(r)}{r^{2}},
\end{equation}
where $\rho$ is the rest--mass density. In relativistic notation one has $\epsilon\simeq \rho c^{2}$, so that 
$\epsilon$ and $\rho$ differ by factors of $c^{2}$. Thus in this limit the term $(\epsilon+p)$ effectively reduces 
to $\epsilon$, recovering the Newtonian form. This shows how the TOV equation generalizes the classical 
hydrostatic equilibrium equation to strongly curved spacetime.  

\paragraph{Boundary conditions and matching.}
At the center of the star, regularity requires that $\epsilon(0)$ and $p(0)$ are finite, and that $\alpha'(0)=0$. 
From the definition of the mass function one finds $m(r)\sim \tfrac{4\pi}{3}\,\epsilon(0)\,r^{3}$ as $r\to 0$, 
so that $p'(0)=0$ follows from \eqref{eq:TOV_pressure}. The stellar surface $R$ is defined by the condition 
$p(R)=0$. Beyond this radius, the solution must match smoothly to the vacuum Schwarzschild exterior. 
This implies $m(R)=M$, the total gravitational mass of the star, and fixes the metric function via 
$e^{2\alpha(R)}=1-2M/R$. The integration constant for $\alpha$ is determined by requiring asymptotic flatness, 
so that $e^{2\alpha}\to 1$ as $r\to\infty$. The factor $(1-2M/R)$ is precisely the Schwarzschild redshift factor, 
and the critical radius $R=2M$ corresponds to the Schwarzschild radius of a black hole. For stable stars, 
one must have $R>2M$.
