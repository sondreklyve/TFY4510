\chapter{Conclusions and Outlook}
\label{chap:conclusion}

The aim of this work has been to develop a coherent and progressively more
realistic description of neutron--star structure, beginning with the equations
of hydrostatic equilibrium in general relativity and culminating in a
self--consistent calculation of dense matter within the relativistic mean--field
$npe\mu$ model. Each step in this progression served to isolate how
microphysical assumptions about matter translate into macroscopic predictions
for the mass--radius relation of neutron stars.

In the case of spherical, non--rotating stars, hydrostatic equilibrium in
general relativity is described by the Tolman--Oppenheimer--Volkoff equations.
These equations make explicit how pressure, energy density, and spacetime
curvature are coupled in a self--gravitating fluid, and they show that no
stellar model is complete without a physically motivated equation of state.
Using this framework, ideal Fermi gases provided the first concrete example.
The non–interacting, zero–temperature Fermi gas captures the essential role of
degeneracy pressure and illustrates how the stiffness of an equation of state
depends on whether the fermions are non–relativistic or ultra–relativistic.
Although highly idealized, these models establish the basic connection between
microscopic physics and macroscopic stellar structure.

The next step was the construction of ideal neutron--star models based on a
pure neutron Fermi gas. Solving the TOV equations with this idealized equation
of state yielded the classic Oppenheimer--Volkoff limit of
$M_{\mathrm{max}}\!\approx\!0.7\,M_{\odot}$, far below the masses of observed
neutron stars. These results confirmed the physical picture anticipated from
the analytic limits: relativistic gravitational effects strengthen the tendency
toward collapse, while degeneracy pressure becomes progressively less effective
at stabilizing high–density matter. The resulting mass–radius curves exhibited
the expected turnover at the maximum mass, and the accompanying radial–mode
analysis verified the standard stability criterion. These calculations also
established the numerical infrastructure later used for more realistic
equations of state.

The principal part of project was the implementation and study of the
relativistic mean–field $npe\mu$ model. Here, interactions between nucleons are
mediated by scalar, vector, and isovector meson fields, calibrated to reproduce
empirical bulk properties of saturated nuclear matter. Electric charge neutrality and
$\beta$–equilibrium were enforced through the inclusion of electrons and
muons. Solving the coupled nonlinear field equations at each density produced a
complete microscopic equation of state for uniform matter. This model captures
several qualitative features expected of neutron–star interiors, including the
growth of the proton fraction at higher densities, the appearance of muons once
the electron chemical potential becomes large enough, and a realistic
competition between attractive scalar and repulsive vector interactions.

To obtain a stellar model that spans the full density range, the microscopic
description of the core was matched to a crust equation of state based on the FPS
parametrization. This produced a complete $P(\epsilon)$ relation suitable
for TOV integration. The resulting mass–radius curve showed reasonable radii
for typical neutron–star masses and behaved smoothly across the crust–core
interface.

However, when confronted with current observational constraints—including radio
timing of massive pulsars, NICER mass–radius measurements, and the
semiparametric EoS posteriors of Ng~et~al.—the limitations of the $npe\mu$
model became clear. Most notably, the maximum mass of the sequence lies below
that of the heaviest firmly measured pulsars. The predicted radii at the NICER
masses also deviate from the central values of the observational contours.
These mismatches indicate that the minimal nucleonic mean–field model, even
when calibrated to nuclear bulk properties, lacks enough stiffness at high
densities to support the observed population of heavy neutron stars. This is
consistent with the broader literature, where additional degrees of freedom,
more sophisticated treatments of nuclear correlations, or density–dependent
couplings are often required to achieve agreement with observations.

In summary, the results of this work show that the $npe\mu$ model provides a
clear and controlled improvement over the idealized Fermi–gas description. It
reproduces many qualitative aspects of dense matter and yields mass–radius
curves that fall within a reasonable range for intermediate densities. At the
same time, the quantitative discrepancies with astrophysical data demonstrate
that this minimal approach is not sufficient as a complete description of
neutron–star interiors. A more realistic equation of state must include
additional physical mechanisms that soften or stiffen the matter in appropriate
density regimes.

\section*{Outlook}

The results of this work show that, while the relativistic mean--field
$npe\mu$ model captures several qualitative features of neutron--star matter,
it is not sufficient to reproduce current observational constraints. This
naturally motivates extensions of the underlying microphysical description,
while retaining the same relativistic framework for stellar structure.

A first direction concerns refinements within the hadronic sector itself. The
RMF model employed here uses fixed couplings calibrated at nuclear saturation
density, which limits its flexibility at higher densities. More sophisticated
variants, such as density--dependent RMF models, allow the effective
interactions to evolve with density and can lead to a stiffer equation of state
in the neutron--star core \cite{glendenning2000}. Similarly, alternative
hadronic descriptions constrained by chiral effective field theory at low
densities provide a more systematic connection to nuclear forces and can reduce
theoretical uncertainties in the density range relevant for typical neutron
stars \cite{Hebeler2015Review}.

At higher densities, additional forms of matter may become energetically
favourable. Within a hadronic description, this can include the appearance of
hyperons or other baryonic excitations, which introduce new degrees of freedom
and tend to soften the equation of state \cite{glendenning2000}. While such
effects are expected on general grounds, their inclusion must be accompanied by
sufficiently repulsive interactions in order to remain compatible with the
existence of massive neutron stars. Exploring this balance is an active area of
research and a natural extension of the present work.

An alternative possibility is that dense matter undergoes a phase transition
to a qualitatively different state at supranuclear densities, namely deconfined
quark matter. In this case, the relevant degrees of freedom change from hadrons
to quarks, and the equation of state must be constructed from a distinct
microscopic model. Matching a hadronic equation of state to a quark matter
description within the TOV framework allows the study of hybrid stars and the
impact of phase transitions on stellar masses, radii, and stability
\cite{glendenning2000,Alford2008}.

Finally, continued progress on the observational side will further guide and
constrain these theoretical developments. Improved mass--radius measurements
from X--ray timing, tighter bounds on tidal deformabilities from gravitational
waves, and the discovery of heavier neutron stars will increasingly restrict
the range of viable equations of state. Within this context, the framework
developed in this project can serve as a starting point for systematic studies
of more advanced dense--matter models, enabling direct comparison between
theory and observation.
