\chapter{Conclusion and Outlook}
\label{chap:conclusion}

The aim of this thesis has been to develop a coherent and progressively more
realistic description of neutron--star structure, beginning with the
relativistic equations that govern hydrostatic equilibrium and culminating in a
self–consistent calculation of dense matter within the relativistic mean–field
$npe\mu$ model. Each step in this progression served to isolate how
microphysical assumptions about matter translate into macroscopic predictions
for the mass–radius relation of neutron stars.

The starting point was the derivation of the
Tolman--Oppenheimer--Volkoff equations, which form the backbone of any
non–rotating neutron--star model. Their structure makes clear that gravity,
pressure, and spacetime curvature are inseparably linked, and that no stellar
model is complete without a physically motivated equation of state. Using this
framework, ideal Fermi gases provided the first concrete examples. The
non–interacting, zero–temperature Fermi gas captures the essential role of
degeneracy pressure and illustrates how the stiffness of an equation of state
depends on whether the fermions are non–relativistic or ultra–relativistic.
Solving the TOV equations with this idealized microphysics yielded the
classical Oppenheimer--Volkoff limit: a maximum mass far below what is observed
in nature. Although quantitatively unrealistic, these models were valuable for
building numerical intuition and for demonstrating how relativistic corrections
enforce upper bounds on the mass of compact stars.

The next step was the construction of ideal neutron--star models based on a
pure neutron Fermi gas. These results largely confirmed the physical picture
anticipated from the analytic limits: relativistic gravitational effects
strengthen the tendency toward collapse, while degeneracy pressure becomes
progressively less effective at stabilizing high–density matter. The resulting
mass–radius curves exhibited the expected turnover at the maximum mass, and the
accompanying radial–mode analysis verified the standard stability criterion.
These calculations established the numerical infrastructure later used for more
realistic equations of state.

The principal part of this thesis was the implementation and study of the
relativistic mean–field $npe\mu$ model. Here, interactions between nucleons are
mediated by scalar, vector, and isovector meson fields, calibrated to reproduce
empirical bulk properties of saturated nuclear matter. Charge neutrality and
$\beta$–equilibrium were enforced through the inclusion of electrons and
muons. Solving the coupled nonlinear field equations at each density produced a
complete microscopic equation of state for uniform matter. This model captures
several qualitative features expected of neutron–star interiors, including the
growth of the proton fraction at higher densities, the appearance of muons once
the electron chemical potential becomes large enough, and a realistic
competition between attractive scalar and repulsive vector interactions.

To obtain a stellar model that spans the full density range, the microscopic
core description was matched to a crust equation of state based on the FPS
parametrization. This produced a complete $P(\varepsilon)$ relation suitable
for TOV integration. The resulting mass–radius curve showed reasonable radii
for typical neutron–star masses and behaved smoothly across the crust–core
interface.

However, when confronted with current observational constraints—including radio
timing of massive pulsars, NICER mass–radius measurements, and the
semiparametric EoS posteriors of Ng~et~al.—the limitations of the $npe\mu$
model became clear. Most notably, the maximum mass of the sequence lies below
that of the heaviest firmly measured pulsars. The predicted radii at the NICER
masses also deviate from the central values of the observational contours.
These mismatches indicate that the minimal nucleonic mean–field model, even
when calibrated to nuclear bulk properties, lacks enough stiffness at high
densities to support the observed population of heavy neutron stars. This is
consistent with the broader literature, where additional degrees of freedom,
more sophisticated treatments of nuclear correlations, or density–dependent
couplings are often required to achieve agreement with observations.

In summary, the results of this thesis show that the $npe\mu$ model provides a
clear and controlled improvement over idealized Fermi–gas descriptions. It
reproduces many qualitative aspects of dense matter and yields mass–radius
curves that fall within a plausible range for intermediate densities. At the
same time, the quantitative discrepancies with astrophysical data demonstrate
that this minimal approach is not sufficient as a complete description of
neutron–star interiors. A more realistic equation of state must include
additional physical mechanisms that soften or stiffen the matter in appropriate
density regimes.

\section*{Outlook}

Several natural extensions follow from the present work.

\textbf{Improved nuclear microphysics:}  
Density–dependent RMF models, Skyrme–type functionals, or
chiral–effective–field–theory–based constraints at low densities could be
incorporated to refine the stiffness of the EoS.

\textbf{Additional degrees of freedom:}  
At supranuclear densities, hyperons, $\Delta$–resonances, meson condensates, or
deconfined quark matter may appear. Their inclusion typically softens the EoS,
requiring repulsive interactions or phase–transition effects to remain
consistent with the observed $2\,M_{\odot}$ stars.

\textbf{Phase transitions and hybrid stars:}  
A natural next step is to explore hadron–quark transitions within the same TOV
framework. Matching the RMF model to a quark–matter EoS, such as quark–meson
or NJL–type models, would allow a systematic study of hybrid–star sequences.

\textbf{Confronting new observational data:}  
Future NICER measurements, improved gravitational–wave constraints on tidal
deformabilities, and radio timing of even heavier pulsars will further restrict
the allowed EoS space. Incorporating these results into the modelling pipeline
developed here is a direct continuation of this work.

The overall conclusion is that the methods and models developed in this thesis
establish a solid baseline for neutron–star calculations, while the
discrepancies with observations clearly motivate the exploration of more
sophisticated descriptions of dense matter.
