\chapter{Ideal Fermi Gases}

\section{Introduction}

The central microscopic mechanism that allows compact stars to resist gravitational collapse
is the Pauli exclusion principle, which forbids fermions from occupying the same quantum state.
Even in the absence of thermal motion, a collection of fermions must successively fill momentum
states, starting from the lowest energy and extending up to a sharp cutoff. This cutoff is
characterized by the \emph{Fermi momentum} $p_F$, the largest momentum occupied in the ground
state of the system, and the associated \emph{Fermi energy} $E_F$, the energy of the highest
occupied single--particle state. The resulting degeneracy pressure, which arises entirely from the
quantum mechanical exclusion principle rather than from thermal motion, provides the dominant
contribution to the internal support of dense stellar matter. 

The simplest theoretical description of such systems is the \emph{ideal, zero--temperature Fermi
gas}. In this model, fermions are treated as non--interacting particles that occupy all quantum
states inside a sphere of radius $p_F$ in momentum space, known as the \emph{Fermi surface}.
Despite its simplicity, the ideal Fermi gas captures the essential qualitative physics of compact
stars~\cite{pathria2011}. For instance, the equilibrium of white dwarfs can be understood by modeling the electrons
as an \emph{ultrarelativistic} Fermi gas, meaning that the typical electron momentum greatly
exceeds the rest mass ($p \gg m$). In this regime, the pressure scales linearly with energy density,
leading to the Chandrasekhar mass limit~\cite{chandrasekhar1931,chandrasekhar1935,shapiro1983}.
Likewise, an \emph{ideal neutron gas} provided the basis of the pioneering Oppenheimer--Volkoff
study of neutron stars~\cite{oppenheimer1939}.

From the perspective of general relativity, the ideal Fermi gas plays a crucial role as it provides
an explicit equation of state, $p = p(\epsilon)$, which closes the Tolman--Oppenheimer--Volkoff
equations derived in the previous chapter. While real neutron star matter is far more complicated,
involving strong nuclear interactions, finite temperatures, and possibly exotic components such as
hyperons or deconfined quarks, the ideal Fermi gas remains a natural starting point. It offers both
analytic insight in certain limits and a benchmark against which more sophisticated models can be
compared.

The purpose of this chapter is therefore twofold: first, to review the theoretical foundations of
the ideal Fermi gas at zero temperature, deriving expressions for number density, energy density,
pressure, and chemical potential; and second, to connect these results to the astrophysical
context of compact stars. We emphasize the non--relativistic and ultra--relativistic limits, where
the expressions simplify considerably and provide physical intuition. Finally, we will discuss how
the ideal Fermi gas equation of state can be coupled to the TOV equations to describe equilibrium
configurations of idealized neutron stars.


\section{Quantum Statistics at Zero Temperature}

Fermions are particles with half--integer spin that obey the Pauli exclusion principle: no two
fermions can occupy the same quantum state simultaneously. As a consequence, when a macroscopic
system of fermions is cooled to zero temperature, the available single--particle states are filled
sequentially from the lowest energy upward. This filling behavior is described by the
Fermi--Dirac distribution,
\begin{equation}
f(E) = \frac{1}{e^{(E-\mu)/T}+1},
\label{eq:fermi_dirac_distribution}
\end{equation}
where $E$ is the single--particle energy, $\mu$ the chemical potential, and $T$ the temperature
(in natural units $k_B=1$). 

\subsection*{Zero--temperature limit}
In the limit $T \to 0$, the distribution becomes a sharp step function,
\begin{equation}
f(E) =
\begin{cases}
1, & E \leq \mu, \\
0, & E > \mu,
\end{cases}
\label{eq:zeroT_distribution}
\end{equation}
so that every state with energy below the chemical potential is fully occupied, while all higher--energy
states are empty. At zero temperature, the chemical potential takes on a special role: it equals the
\emph{Fermi energy},
\begin{equation}
E_F = \mu(T=0),
\end{equation}
the energy of the highest occupied single--particle state in the system. From the
special--relativistic dispersion relation,
\begin{equation}
E(p) = \sqrt{p^2+m^2},
\label{eq:SR_dispersion}
\end{equation}
with $m$ the particle rest mass and $p = |\vec{p}|$ the momentum magnitude. One can then define
the corresponding \emph{Fermi momentum},
\begin{equation}
p_F = \sqrt{E_F^2 - m^2},
\end{equation}
which sets the radius of the \emph{Fermi surface} in momentum space. These two quantities,
$E_F$ and $p_F$, are the natural scales of a degenerate Fermi system: once specified, they
determine the characteristic energies, momenta, and thermodynamic properties of the gas.

\subsection*{Physical meaning of $E_F$ and $p_F$}
The Fermi energy $E_F$ measures the energy cost of adding one more particle to the system at zero
temperature, since all lower--energy states are already occupied. The Fermi momentum $p_F$ is
therefore the largest momentum carried by any fermion in the ground state. Together they define
the ``surface'' of occupied states in momentum space --- the Fermi surface --- which plays a central
role in both condensed matter and astrophysical systems. Even at $T=0$, fermions cannot all sit in
the lowest momentum state; instead, the exclusion principle forces them to occupy an entire sphere
of states up to $p_F$. This crowding of states gives rise to degeneracy pressure, independent of
temperature.


\section{Thermodynamics of the Ideal Fermi Gas}
\label{sec:thermo_ideal_fermi}

Having established the statistical foundations and the concepts of Fermi energy and Fermi
momentum, we now derive the thermodynamic quantities of a degenerate Fermi gas at zero
temperature by integrating over occupied momentum states up to $p_F$. The momentum--space
volume element is $d^3p=4\pi p^2\,dp$, and the density of states per unit volume is
$\tfrac{g}{(2\pi)^3}\,d^3p$, where $g=2s+1$ accounts for spin degeneracy. The following expressions are
standard and can be found in e.g. Shapiro \& Teukolsky or Carroll~\cite{shapiro1983,carroll}.

\subsection*{Number density}
The \emph{number density} $n$ denotes the number of particles per unit volume in the fluid
rest frame ($n \equiv N/V$). At zero temperature it is determined by how many
single--particle momentum states are filled up to the Fermi momentum $p_F$.
Because of the Pauli principle, each momentum state can be occupied by several fermions
that differ only in internal quantum numbers. For a particle of spin $s$, the available spin
projections $m_s=-s,-s+1,\dots,s$ give a total \emph{degeneracy}
\begin{equation}
g \;=\; 2s+1.
\end{equation}
Thus, each momentum state admits up to $g$ fermions. For spin-$\tfrac12$ particles
such as electrons, neutrons, or protons one has $g=2$.

Counting the total number of occupied states inside the Fermi sphere then gives
\begin{equation}
n \;=\; g \int_{|\vec p|\le p_F}\frac{d^3p}{(2\pi)^3}
\;=\; \frac{g}{2\pi^2}\int_0^{p_F}p^2\,dp
\;=\; \frac{g}{6\pi^2}\,p_F^{3}.
\label{eq:fermi_number_density}
\end{equation}
This fundamental relation fixes the Fermi momentum in terms of $n$,
\begin{equation}
p_F \;=\; \Big(\tfrac{6\pi^2}{g}\,n\Big)^{1/3},
\label{eq:pF_of_n}
\end{equation}
and connects the microscopic filling of momentum states to the macroscopic particle density.

\subsection*{Energy density}
The total energy density follows from summing the single--particle energies $E(p)=\sqrt{p^2+m^2}$ 
over all occupied momentum states. In terms of the general phase--space integral,
\begin{equation}
\epsilon \;=\; g \int_{|\vec p|\le p_F}\frac{d^3p}{(2\pi)^3}\,E(p),
\end{equation}
this gives
\begin{equation}
\epsilon \;=\; \frac{g}{2\pi^2}\int_0^{p_F}\sqrt{p^2+m^2}\,p^2\,dp.
\label{eq:energy_density_integral}
\end{equation}
To simplify the evaluation it is convenient to introduce the dimensionless variable $x=p/m$, 
with upper limit $x_F=p_F/m$, so that
\begin{equation}
\epsilon \;=\; \frac{g\,m^4}{2\pi^2}\int_0^{x_F}\!\sqrt{1+x^2}\,x^2\,dx.
\end{equation}
Carrying out the integral yields the compact analytic form
\begin{equation}
\epsilon \;=\; \frac{g\,m^4}{16\pi^2}
\Big[\,x_F\sqrt{1+x_F^2}\,(2x_F^2+1)\;-\;\sinh^{-1}x_F\,\Big].
\label{eq:energy_density}
\end{equation}
This expression makes explicit the crossover between the non--relativistic regime 
($x_F\ll 1$, where $\epsilon \approx m n$ with small kinetic corrections) 
and the ultra--relativistic regime ($x_F\gg 1$, where $\epsilon \propto p_F^4$).

\subsection*{Pressure}
Microscopically, pressure can be defined as the flux of momentum carried by the particles across a unit area. 
In relativistic kinetic theory this notion is captured by the spatial diagonal components of the energy--momentum tensor,
\begin{equation}
T^{\mu\nu} \;=\; g \int \frac{d^3p}{(2\pi)^3}\,\frac{p^\mu p^\nu}{E(p)}\,f(p),
\end{equation}
with $E(p)=\sqrt{p^2+m^2}$ and $f(p)$ the occupation number. 
At zero temperature, the distribution is simply a step function $f(p)=\Theta(p_F-p)$, filling all states up to the Fermi momentum. 
The pressure is then given by the diagonal spatial component, for instance $T^{xx}$,
\begin{equation}
p \;=\; T^{xx} \;=\; g \int \frac{d^3p}{(2\pi)^3}\,\frac{p_x^2}{E(p)}\,f(p).
\end{equation}
Because the Fermi sphere is isotropic, the average over directions yields $\langle p_x^2 \rangle = p^2/3$, so that
\begin{equation}
p \;=\; \frac{g}{3}\int \frac{d^3p}{(2\pi)^3}\,\frac{p^2}{E(p)}\,f(p).
\end{equation}
Switching to spherical momentum coordinates, $d^3p=4\pi p^2 dp$, gives the momentum--flux integral
\begin{equation}
p \;=\; \frac{g}{6\pi^2}\int_0^{p_F}\frac{p^4}{\sqrt{p^2+m^2}}\,dp
\;=\; \frac{g\,m^4}{6\pi^2}\int_0^{x_F}\frac{x^4}{\sqrt{1+x^2}}\,dx,
\label{eq:pressure_integral}
\end{equation}
which evaluates to
\begin{equation}
p \;=\; \frac{g\,m^4}{48\pi^2}
\Big[\,x_F\sqrt{1+x_F^2}\,(2x_F^2-3)\;+\;3\,\sinh^{-1}x_F\,\Big].
\label{eq:pressure}
\end{equation}

\subsection*{Chemical potential}
The chemical potential measures the change in energy when adding one particle to the system, 
and at zero temperature it coincides with the Fermi energy. 
Intuitively, this is because the highest occupied single--particle state at $T=0$ 
sits exactly at the Fermi surface, so adding one more fermion forces it into that state. 
Formally, one finds
\begin{equation}
\mu \;=\; E_F \;=\; \sqrt{p_F^2+m^2}
\;=\; m\,\sqrt{1+x_F^2}
\;=\; \sqrt{\,m^2 + \Big(\tfrac{6\pi^2}{g}\,n\Big)^{2/3}\,},
\label{eq:mu_zeroT}
\end{equation}
where the last equality follows from the relation between Fermi momentum and number density, 
Eq.~\eqref{eq:pF_of_n}.

Two useful thermodynamic identities hold in the degenerate limit $T=0$: 
\begin{equation}
d\epsilon \;=\; \mu\,dn,
\qquad
\epsilon + p \;=\; \mu\,n.
\label{eq:zeroT_identities}
\end{equation}
The first expresses that the chemical potential is the derivative of the energy density 
with respect to particle number, while the second is the Gibbs--Duhem relation (or enthalpy identity) 
with the entropy term $Ts$ vanishing at zero temperature. 
These relations are easily checked by differentiating the explicit integral expressions 
for $\epsilon$ and $n$: from \eqref{eq:energy_density_integral} and \eqref{eq:fermi_number_density} one finds
\[
\mu \;=\; \frac{d\epsilon/dp_F}{dn/dp_F} \;=\; \sqrt{p_F^2+m^2},
\]
which confirms Eq.~\eqref{eq:mu_zeroT}.

\section{Equation of State and Limiting Cases}

The expressions derived above provide $n$, $\epsilon$ and $p$ as explicit functions of the
Fermi momentum $p_F$ (or equivalently the dimensionless ratio $x_F=p_F/m$). In practice,
this means that the ideal Fermi gas equation of state is most conveniently written in
\emph{parametric form}:
\begin{align}
\epsilon &= \epsilon(x_F), \\
p &= p(x_F),
\end{align}
with $\epsilon(x_F)$ and $p(x_F)$ given by
Eqs.~\eqref{eq:energy_density} and \eqref{eq:pressure}. The chemical potential
is simultaneously determined by Eq.~\eqref{eq:mu_zeroT}. Eliminating $x_F$ gives a
closed functional relation $p(\epsilon)$, but the analytic form is not especially
transparent; in astrophysical applications one typically evaluates the parametric
expressions numerically.

Nevertheless, in the two limiting regimes $p_F \ll m$ (non--relativistic) and
$p_F \gg m$ (ultra--relativistic), the integrals admit simple expansions that are
physically illuminating.

\subsection*{Non--relativistic limit}
In the non--relativistic regime $x_F=p_F/m\ll 1$, single--particle energies are
$E(p)\simeq m+\frac{p^2}{2m}$. The pressure and number density at $T=0$ are
\begin{equation}
p \;\simeq\; \frac{g}{30\pi^2\,m}\,p_F^5,
\qquad
n \;=\; \frac{g}{6\pi^2}\,p_F^3.
\label{eq:NR_p_and_n}
\end{equation}
In this limit the \emph{total} energy density is dominated by rest mass,
\begin{equation}
\epsilon \;\simeq\; m\,n,
\label{eq:NR_eps_total}
\end{equation}
with a subleading kinetic piece (of order $p_F^5/m$) that we neglect when relating
$\epsilon$ to $p$.

Eliminating $p_F$ between \eqref{eq:NR_p_and_n} gives
\[
p_F^5 \;=\; \frac{30\pi^2\,m}{g}\,p,
\qquad
p_F^3 \;=\; \biggl(\frac{30\pi^2\,m}{g}\,p\biggr)^{\!3/5}.
\]
Hence
\begin{align}
n \;&=\; \frac{g}{6\pi^2}\,p_F^3
\;=\; \frac{g}{6\pi^2}\,\biggl(\frac{30\pi^2\,m}{g}\,p\biggr)^{\!3/5},
\\[2pt]
\epsilon(p) \;&=\; m\,n
\;=\; \Bigl(\tfrac{30^{3/5}}{6}\,\pi^{-4/5}\Bigr)\,
g^{2/5}\,m^{8/5}\,p^{3/5}.
\end{align}
Thus, in the non--relativistic limit the equation of state exhibits the polytropic scaling
\begin{equation}
\epsilon \;\propto\; p^{3/5}.
\end{equation}

\subsection*{Ultra--relativistic limit}
For large Fermi momenta, $x_F \gg 1$, the rest mass becomes negligible and
$E(p) \approx p$. The integrals then yield
\begin{align}
\epsilon &\;\approx\; \frac{g}{8\pi^2}\,p_F^4, \\
p &\;\approx\; \frac{g}{24\pi^2}\,p_F^4,
\end{align}
so that the equation of state approaches
\begin{equation}
\epsilon \;=\; 3\, p.
\label{eq:ultrarel_eos}
\end{equation}
This is identical to the relation for a gas of massless particles such as
photons or neutrinos, consistent with the fact that for $p \gg m$ the particle
dispersion relation becomes effectively massless.

\subsection*{Physical interpretation}
The two limiting behaviors highlight the changing \emph{stiffness} of the Fermi gas
equation of state, i.e.\ how strongly the pressure responds to increasing energy density. 
In the non--relativistic regime, the pressure scales as $n^{5/3}$, which rises faster 
than linearly with density but remains subdominant to the rest--mass energy in $\epsilon$. 
In contrast, in the ultra--relativistic regime the pressure is a fixed fraction ($1/3$) 
of the energy density, producing a much softer relation between $p$ and $\epsilon$. 
The transition between these two regimes occurs when $p_F \sim m$, i.e.\ when
the typical fermion momentum becomes comparable to the rest mass. In compact
stars, the electron component typically lies in the ultra--relativistic regime,
whereas neutrons can remain non--relativistic in less massive stars but
become increasingly relativistic toward higher central densities.
In all cases, the degeneracy pressure encapsulated by the Fermi gas equation
of state provides the fundamental microscopic mechanism preventing
gravitational collapse.

\begin{figure}[ht]
\centering
\includegraphics[width=0.7\textwidth]{figures/fermi_eos.pdf}
\caption[Stiffness of the Fermi gas equation of state]{
Ratio of pressure to energy density $p/\epsilon$ as a function of the
dimensionless Fermi momentum $x_F = p_F/m$. 
In the non--relativistic regime $x_F \ll 1$, rest--mass energy dominates
and $p/\epsilon \to 0$. 
In the ultra--relativistic regime $x_F \gg 1$, the relation approaches
$p = \epsilon/3$, indicated by the dashed line.
}
\label{fig:fermi_eos}
\end{figure}

\section{Astrophysical Relevance}

The ideal Fermi gas model, while highly simplified, captures the essential
microscopic mechanism that allows compact stars to resist gravitational
collapse: degeneracy pressure. Its relevance can be illustrated in two
canonical astrophysical contexts.

\paragraph{White dwarfs.}
In white dwarfs, the pressure support is provided almost entirely by
degenerate electrons. Typical central densities are so high that the
electron Fermi momentum satisfies $p_F \gg m_e$, placing the system in
the ultra--relativistic regime. As a consequence, the equation of state
approaches $p = \epsilon/3$, and the stellar mass becomes essentially
independent of radius. Balancing this relativistic degeneracy pressure
against gravity yields the famous Chandrasekhar mass limit,
$M_{\rm Ch} \simeq 1.4\,M_\odot$, beyond which no stable white dwarf
configurations exist. This provides the theoretical foundation for our
understanding of type~Ia supernova progenitors.

\paragraph{Neutron stars.}
In neutron stars, the relevant fermions are neutrons, whose rest mass is
comparable to their Fermi momentum at typical nuclear densities. As a
result, the system lies in the intermediate regime between the
non--relativistic and ultra--relativistic limits. The pioneering work of
Oppenheimer and Volkoff modeled neutron stars as ideal neutron gases and
found a maximum mass of order $0.7\,M_\odot$. While this value is far
below the $\sim 2\,M_\odot$ neutron stars observed today~\cite{haensel2007}, the calculation
was historically crucial: it demonstrated that relativistic gravity,
together with a microscopic equation of state, predicts an upper mass
limit for neutron stars analogous to Chandrasekhar’s result for white
dwarfs.

\paragraph{Limitations.}
The ideal Fermi gas neglects several important physical ingredients. Real
neutron star matter is strongly interacting, with short--range nuclear
forces significantly modifying the pressure at supranuclear densities.
Charge neutrality and beta equilibrium imply a mixture of neutrons,
protons, electrons, and muons, rather than a pure single--species Fermi
gas. At sufficiently high densities, more exotic components such as
hyperons, meson condensates, or deconfined quark matter may appear.
Moreover, finite temperature effects can be relevant in newly born
protoneutron stars. All of these factors alter the equation of state,
and hence the predicted stellar structure.

\paragraph{Summary.}
Despite these limitations, the ideal Fermi gas remains an indispensable
baseline model. It provides analytic control in limiting regimes,
a clear physical picture of degeneracy pressure, and a convenient
parametric equation of state that can be directly coupled to the
Tolman--Oppenheimer--Volkoff equations. As such, it serves both as a
pedagogical introduction to compact star physics and as a benchmark
against which more realistic models of dense matter can be tested.
