\chapter{The \texorpdfstring{$npe\mu$}{npemu} model}
\label{chap:npe_mu}

\section{Introduction}

In the previous chapters we developed and tested the structure equations for
neutron stars using simple models of matter. These models were useful for
building intuition about how gravity and pressure balance to form stable
configurations, and for exploring how different equations of state affect the
stellar mass and radius. However, at densities near and above the nuclear
saturation point, the assumptions of non–interacting particles become too
simple. The internal structure of nucleons and their mutual interactions start
to play a major role in determining the properties of matter.

To describe these effects in a practical way, we now turn to the relativistic
mean–field (RMF) approximation. This framework models dense nuclear matter through
average fields that represent the main attractive and repulsive parts of the
nuclear force \cite{glendenning2000,pogliano2017}. 
It provides a way to include essential interaction physics while
keeping the formulation simple enough for use in stellar calculations.

In this chapter we use the RMF approximation to describe matter composed of neutrons,
protons, electrons, and muons ($npe\mu$). By imposing charge neutrality and
$\beta$–equilibrium, this composition forms a minimal realistic model of a cold
neutron–star core. By studying existing RMF equations of state for such matter,
we can examine how nuclear interactions influence the pressure and density
profiles and how this, in turn, affects the mass–radius relation obtained from
the structure equations.

\section{Bulk Properties of Nuclear Matter}
\label{sec:bulk_properties}

A useful way to test any microscopic model of nuclear interactions is to see
whether it can reproduce the known bulk properties of symmetric nuclear matter.
These are macroscopic quantities that characterize infinite, uniform matter
composed of equal numbers of neutrons and protons, in its most stable (saturated)
state \cite{glendenning2000}. They summarize how the nuclear force behaves on average and provide
empirical reference points for any equation of state.

The most important of these bulk properties are:
\begin{itemize}
  \item the \textbf{binding energy per nucleon} $B/A$, which measures how tightly the
        nucleons are bound together at equilibrium;
  \item the \textbf{saturation density} $\rho_{0}$, i.e.\ the baryon number density at which
        the energy per particle reaches its minimum;
  \item the \textbf{effective nucleon mass} $m^{*}$, which reflects how the scalar field
        modifies the nucleon rest mass in the medium;
  \item the \textbf{compression modulus} $K$, which quantifies how stiff or soft the
        matter is against uniform compression;
  \item the \textbf{symmetry energy coefficient} $a_{\mathrm{sym}}$, which describes how
        the energy increases when the number of neutrons and protons becomes unequal.
\end{itemize}

The simplest RMF description of such matter is the
\emph{$\sigma$–$\omega$ model} \cite{glendenning2000,pogliano2017}.
In this approach, the strong nuclear interaction is
represented by the exchange of two meson fields: an attractive scalar field
($\sigma$) and a repulsive vector field ($\boldsymbol{\omega}$). The scalar attraction binds the
nucleons, while the vector repulsion prevents the system from collapsing at high
density. The balance between these two effects produces saturation naturally, in
contrast to non–relativistic models where it must be inserted by hand.

The Lagrangian density of the linear $\sigma$–$\omega$ model is
\begin{align}
\mathcal{L}
&= \bar{\psi}\bigl[\gamma_{\mu}(i\partial^{\mu} - g_{\omega}\omega^{\mu})
      - (m - g_{\sigma}\sigma)\bigr]\psi
   + \tfrac{1}{2}(\partial_{\mu}\sigma\,\partial^{\mu}\sigma - m_{\sigma}^{2}\sigma^{2})
   - \tfrac{1}{4}\omega_{\mu\nu}\omega^{\mu\nu}
   + \tfrac{1}{2}m_{\omega}^{2}\omega_{\mu}\omega^{\mu},
\label{eq:lagrangian_sigmaomega}
\end{align}
where $\psi$ is the nucleon field, $\omega_{\mu\nu}=\partial_{\mu}\omega_{\nu}-\partial_{\nu}\omega_{\mu}$
is the field tensor of the vector meson, and $m$, $m_{\sigma}$, $m_{\omega}$ are the
nucleon and meson masses, respectively.

From Eq.~\eqref{eq:lagrangian_sigmaomega} it is evident that the interaction
between nucleons and the meson fields is governed solely by the coupling
constants $g_{\sigma}$ and $g_{\omega}$. Since there is one scalar and one vector
interaction channel, the theory contains only two independent coupling
strengths, which appear in the dimensionless ratios $g_{\sigma}/m_{\sigma}$ and
$g_{\omega}/m_{\omega}$. Consequently, only two bulk nuclear observables—typically the
binding energy per nucleon and the saturation density—can be fitted
independently. The other quantities then follow automatically as predictions
from the model, but the resulting values are not realistic. In particular, the
predicted compression modulus $K$ is much too large, typically around
$K \simeq 550\,\mathrm{MeV}$, whereas empirical analyses of nuclear resonances give
$K \approx 230\,\mathrm{MeV}$ \cite{glendenning2000}. The effective mass and
symmetry energy are also found to deviate significantly from experiment.

To correct these deficiencies, the model is extended in two main ways.
First, one introduces non–linear self–interactions of the scalar field
$\sigma$, adding cubic and quartic terms to the Lagrangian.
These modify the density dependence of the attractive potential and thereby
soften the equation of state, bringing the compression modulus into agreement
with empirical data.
Second, one includes an additional isovector meson field ($\boldsymbol{\rho}$), which
couples to the difference between neutron and proton densities.
The $\rho$ meson restores isospin symmetry and governs how the energy changes
with neutron–proton asymmetry, allowing the model to reproduce the correct
symmetry energy.

With these improvements, the model can be tuned to
reproduce all five key bulk properties listed above.
Once calibrated in this way, it provides a realistic and internally consistent
equation of state that can be used to describe the dense matter inside neutron–star
cores.

At this point the model can be generalized from symmetric nuclear matter to
neutron–rich matter, as found in the interiors of neutron stars. In such
environments, weak interactions continuously convert neutrons into protons and
leptons until the system reaches $\beta$–equilibrium, where
\[
\mu_{n} = \mu_{p} + \mu_{e} = \mu_{p} + \mu_{\mu}.
\]
To maintain overall charge neutrality, the total positive charge of protons must
be balanced by the negative charge of the leptons,
\[
n_{p} = n_{e} + n_{\mu}.
\]
Electrons and muons are therefore included as free, relativistic Fermi gases,
while the baryons (neutrons and protons) continue to interact through the
exchange of the $\sigma$, $\omega$, and $\rho$ mesons introduced above.

The resulting composition—neutrons ($n$), protons ($p$), electrons ($e$), and
muons ($\mu$)—defines the so–called \emph{$npe\mu$ model}. It represents the
minimal, physically consistent description of cold matter in the
stellar core. Within this framework, the baryonic sector is governed by the
RMF approximation tuned to reproduce nuclear bulk properties,
while the leptonic sector enforces charge neutrality and $\beta$–stability.
Together, these ingredients yield a barotropic equation of state $P(\varepsilon)$
that can be directly used in the Tolman–Oppenheimer–Volkoff equations to compute
the mass–radius relation of neutron stars \cite{glendenning2000,pogliano2017}.


\section{RMF Lagrangian Formulation}
\label{sec:rmf_lagrangian}

To translate the qualitative picture of the previous section into a quantitative
framework, we formulate the RMF approximation in terms of a
Lagrangian density that describes the interactions among nucleons and meson
fields. From this Lagrangian, the field equations and thermodynamic quantities
follow in a systematic way through the Euler–Lagrange formalism. In the present
context, the aim is not to explore field theory in detail, but rather to obtain
a practical and self–consistent expression for the energy density and pressure
of uniform matter in $\beta$–equilibrium.

In RMF theory, the total Lagrangian is written as the sum of contributions from
each sector,
\begin{equation}
\mathcal{L}
= \mathcal{L}_{N}
+ \mathcal{L}_{\sigma}
+ \mathcal{L}_{\omega}
+ \mathcal{L}_{\rho}
+ \mathcal{L}_{\ell},
\label{eq:L_total}
\end{equation}
where $\mathcal{L}_{N}$ denotes the nucleonic Dirac term, 
$\mathcal{L}_{\sigma}$, $\mathcal{L}_{\omega}$, and $\mathcal{L}_{\rho}$ are the
scalar, isoscalar–vector, and isovector–vector meson sectors, respectively, and
$\mathcal{L}_{\ell}$ accounts for the free leptons (electrons and muons).  The
explicit expressions are
\begin{align}
\mathcal{L}_{N}
&= 
\bar{\psi}\!\left[
i\gamma^{\mu}\!\left(\partial_{\mu}
+ i g_{\omega}\,\omega_{\mu}
+ \tfrac{1}{2} g_{\rho}\,\boldsymbol{\tau}\!\cdot\!\boldsymbol{\rho}_{\mu}\right)
- \bigl(m - g_{\sigma}\sigma\bigr)
\right]\!\psi,
\label{eq:L_N}\\[0.5em]
\mathcal{L}_{\sigma}
&=
\frac{1}{2}\bigl(\partial_{\mu}\sigma\,\partial^{\mu}\sigma - m_{\sigma}^{2}\sigma^{2}\bigr)
- U(\sigma),
\qquad
U(\sigma)
= \frac{1}{3}\,m_{n}b(g_{\sigma}\sigma)^{3}
+ \frac{1}{4}\,c(g_{\sigma}\sigma)^{4},
\label{eq:L_sigma}\\[0.5em]
\mathcal{L}_{\omega}
&=
-\frac{1}{4}\omega_{\mu\nu}\omega^{\mu\nu}
+\frac{1}{2}m_{\omega}^{2}\omega_{\mu}\omega^{\mu},
\qquad
\omega_{\mu\nu}=\partial_{\mu}\omega_{\nu}-\partial_{\nu}\omega_{\mu},
\label{eq:L_omega}\\[0.5em]
\mathcal{L}_{\rho}
&=
-\frac{1}{4}\boldsymbol{\rho}_{\mu\nu}\!\cdot\!\boldsymbol{\rho}^{\mu\nu}
+\frac{1}{2}m_{\rho}^{2}\boldsymbol{\rho}_{\mu}\!\cdot\!\boldsymbol{\rho}^{\mu},
\qquad
\boldsymbol{\rho}_{\mu\nu}
=\partial_{\mu}\boldsymbol{\rho}_{\nu}-\partial_{\nu}\boldsymbol{\rho}_{\mu},
\label{eq:L_rho}\\[0.5em]
\mathcal{L}_{\ell}
&=
\sum_{\lambda=e,\mu}
\bar{\psi}_{\lambda}\bigl(i\gamma^{\mu}\partial_{\mu}-m_{\lambda}\bigr)\psi_{\lambda}.
\label{eq:L_leptons}
\end{align}

The constants $m$, $m_{\sigma}$, $m_{\omega}$, and $m_{\rho}$ are the masses of
the nucleon and mesons, and $g_{\sigma}$, $g_{\omega}$, $g_{\rho}$ are the
respective coupling constants.  The parameters $b$ and $c$ control the
non–linear self–interactions of the scalar field and are fixed empirically to
reproduce the saturation properties of nuclear matter
\cite{glendenning2000}.

In the following sections, we examine each sector in turn. By using the
RMF approximation, we derive the field
equations, the single–particle spectrum, and the corresponding contributions to
the total energy density and pressure. 

\subsection{Nucleon term}
\label{subsec:nucleon_term}

The nucleonic part of the Lagrangian, Eq.~\eqref{eq:L_N}, describes a relativistic
fermion field interacting with meson fields through Yukawa-type couplings.
Varying the Lagrangian with respect to $\bar\psi$ gives the Dirac equation
\begin{equation}
\Bigl[\gamma^\mu\!\bigl(i\partial_\mu - g_\omega\,\omega_\mu - \tfrac{1}{2}g_\rho\,\boldsymbol{\tau}\!\cdot\!\boldsymbol{\rho}_\mu\bigr)
 - (m - g_\sigma\,\sigma)\Bigr]\psi = 0.
\end{equation}
In the RMF approximation appropriate for infinite, uniform matter, spatial
gradients vanish and only the temporal components of the vector fields remain
nonzero. The expectation values of the meson fields are denoted
$\langle\sigma\rangle$, $\langle\omega_0\rangle$, and
$\langle\rho_{03}\rangle$, while the corresponding spatial
components vanish. The nucleons then move independently in the presence of
constant background fields that represent the average effect of their
interactions. The expectation value of the scalar field shifts the nucleon
rest mass to an effective value
\begin{equation}
m^{*} = m - g_\sigma\,\langle\sigma\rangle,
\end{equation}
which reduces the single-particle energy relative to the vacuum and
corresponds to an attractive potential. The time components of the vector
fields, on the other hand, act as constant potentials that add repulsive
energy shifts proportional to the baryon and isospin densities. For a nucleon
species $B$ (proton or neutron) with isospin projection $I_3 = +\tfrac{1}{2}$
for protons and $I_3 = -\tfrac{1}{2}$ for neutrons, the single-particle energy
spectrum becomes
\begin{equation}
e_B(p) = g_\omega\,\langle\omega_0\rangle + I_B\,g_\rho\,\langle\rho_{03}\rangle
         + \sqrt{p^{2} + m^{*2}}.
\label{eq:single_particle_energy}
\end{equation}
At zero temperature, each species fills all momentum states up to its Fermi
momentum $p_{F,B}$, giving the number densities
\[
n_B = \frac{p_{F,B}^{3}}{3\pi^{2}}, \qquad
n_B^{\mathrm{tot}} = n_p + n_n.
\]
The corresponding kinetic contribution to the total energy density follows from
the sum of the single-particle energies over occupied states,
\begin{equation}
\epsilon_{N}^{\mathrm{kin}}
  = \frac{1}{\pi^{2}}\sum_{B=n,p}
    \int_{0}^{p_{F,B}}\!dp\,p^{2}\sqrt{p^{2} + m^{*2}},
\label{eq:epsilon_nucleon}
\end{equation}
while the kinetic (degeneracy) pressure, obtained from the diagonal component of the
energy–momentum tensor or equivalently from the thermodynamic relation
$P = n_B^{2}\,d(\epsilon/n_B)/dn_B$, takes the form
\begin{equation}
P_{N}^{\mathrm{kin}}
  = \frac{1}{3\pi^{2}}\sum_{B=n,p}
    \int_{0}^{p_{F,B}}\!dp\,\frac{p^{4}}{\sqrt{p^{2} + m^{*2}}}.
\label{eq:pressure_nucleon}
\end{equation}
These two integrals represent the free Fermi–gas contributions modified by the
effective mass $m^{*}$. The vector fields enter
the single-particle energies in Eq.~\eqref{eq:single_particle_energy} as constant
shifts. Their effect on the thermodynamics appears not through the integrals
above, but through separate classical field–energy terms associated with
$\langle\omega_0\rangle$ and $\langle\rho_{03}\rangle$. These will be added when the corresponding
meson sectors are discussed below.

The chemical potentials of the nucleons, which determine the conditions for
$\beta$-equilibrium, follow directly from the Fermi energies. For each species,
\begin{align}
\mu_p &= g_\omega\,\langle\omega_0\rangle + \tfrac{1}{2}g_\rho\,\langle\rho_{03}\rangle
         + \sqrt{p_{F,p}^{2} + m^{*2}},\\
\mu_n &= g_\omega\,\langle\omega_0\rangle - \tfrac{1}{2}g_\rho\,\langle\rho_{03}\rangle
         + \sqrt{p_{F,n}^{2} + m^{*2}}.
\end{align}
The difference between neutron and proton chemical potentials is thus governed
by the $\rho$ field, while their common shift arises from the $\omega$ field.
The effective mass $m^{*}$ encapsulates the attractive scalar interaction,
and the competition between these scalar and vector terms determines the net
binding and saturation of nuclear matter. Together, Eqs.~\eqref{eq:epsilon_nucleon}
and \eqref{eq:pressure_nucleon} provide the nucleonic (kinetic) parts of the
energy density and pressure, to which the meson field energies will now be
added to obtain the full equation of state.

\subsection{Scalar field term}
\label{subsec:sigma_term}

The scalar $\sigma$ field represents the attractive part of the strong
interaction between nucleons. Its Lagrangian, Eq.~\eqref{eq:L_sigma}, contains a
quadratic mass term and additional cubic and quartic self–interactions,
\[
\mathcal{L}_{\sigma}
= \frac{1}{2}\bigl(\partial_{\mu}\sigma\,\partial^{\mu}\sigma - m_{\sigma}^{2}\sigma^{2}\bigr)
- \frac{1}{3}\,m_{n}b\,(g_{\sigma}\sigma)^{3}
- \frac{1}{4}\,c\,(g_{\sigma}\sigma)^{4}.
\]
The self–interaction $U(\sigma)$ softens the scalar attraction at high
density and is essential to reproduce realistic nuclear compressibility.
Varying the Lagrangian with respect to $\sigma$ yields the field equation
\begin{equation}
\partial_{\mu}\partial^{\mu}\sigma
+ m_{\sigma}^{2}\sigma
+ m_{n}b\,(g_{\sigma}\sigma)^{2}
+ c\,(g_{\sigma}\sigma)^{3}
= g_{\sigma}\,\bar{\psi}\psi.
\end{equation}
In uniform matter, spatial derivatives vanish and the field is constant.
Within the mean–field approximation we assume factorization of powers of the
field, $\langle\sigma^{n}\rangle \approx \langle\sigma\rangle^{n}$, which leads to
\begin{equation}
m_{\sigma}^{2}\,\langle\sigma\rangle
+ m_{n}b\,(g_{\sigma}\langle\sigma\rangle)^{2}
+ c\,(g_{\sigma}\langle\sigma\rangle)^{3}
= g_{\sigma}\,\langle\bar{\psi}\psi\rangle.
\label{eq:sigma_mf_eq}
\end{equation}
where $\langle\bar{\psi}\psi\rangle$ is the scalar density of nucleons,
\begin{equation}
\langle\bar{\psi}\psi\rangle
= \sum_{B=n,p}\frac{1}{\pi^{2}}\!
  \int_{0}^{p_{F,B}}\!dp\,p^{2}\,
  \frac{m^{*}}{\sqrt{p^{2}+m^{*2}}}.
\label{eq:scalar_density}
\end{equation}
This self–consistency relation couples the scalar mean field $\langle\sigma\rangle$ to the
occupied Fermi seas and must be solved together with the vector–meson and
equilibrium conditions at each baryon density.

The scalar field contributes to the total energy density both directly, through
its potential energy, and indirectly, through the modification of the nucleon
masses in the baryon integral. The contribution to the energy density associated
with the $\sigma$ field is 
as
\begin{equation}
\epsilon_{\sigma}
= \frac{1}{2}m_{\sigma}^{2}\langle\sigma\rangle^{2}
  + \frac{1}{3}m_{n}b\,(g_{\sigma}\langle\sigma\rangle)^{3}
  + \frac{1}{4}c\,(g_{\sigma}\langle\sigma\rangle)^{4}
\label{eq:epsilon_sigma}
\end{equation}
The corresponding pressure follows from the
energy–momentum tensor or from the thermodynamic identity
$P = \sum_i \mu_i n_i - \epsilon$, yielding
\begin{equation}
P_{\sigma}
= -\,\frac{1}{2}m_{\sigma}^{2}\langle\sigma\rangle^{2}
  - \frac{1}{3}m_{n}b\,(g_{\sigma}\langle\sigma\rangle)^{3}
  - \frac{1}{4}c\,(g_{\sigma}\langle\sigma\rangle)^{4}
\label{eq:pressure_sigma}
\end{equation}
These expressions represent the pure field contributions of the scalar meson to
the total energy density and pressure. The $\sigma$ field lowers the total
energy by generating an attractive interaction between nucleons and thus plays
a central role in producing nuclear binding and saturation in the relativistic
framework.

\subsection{Vector field term}
\label{subsec:omega_term}

The vector $\omega$ field represents the short–range repulsive part of the
nuclear interaction. It couples to the conserved baryon current and provides a
repulsive potential that grows with density, counteracting the attraction from
the scalar field and ensuring that nuclear matter saturates at a finite
density. The corresponding part of the Lagrangian is
\[
\mathcal{L}_{\omega}
= -\frac{1}{4}\,\omega_{\mu\nu}\omega^{\mu\nu}
  + \frac{1}{2}\,m_{\omega}^{2}\,\omega_{\mu}\omega^{\mu},
\qquad
\omega_{\mu\nu} = \partial_{\mu}\omega_{\nu} - \partial_{\nu}\omega_{\mu}.
\]
Variation with respect to $\omega_{\mu}$ gives the field equation
\[
\partial_{\nu}\omega^{\mu\nu} + m_{\omega}^{2}\omega^{\mu}
= g_{\omega}\,\bar{\psi}\gamma^{\mu}\psi.
\]
For uniform, static matter, spatial derivatives vanish and only the time
component survives. Denoting its expectation value by
$\langle\omega^{0}\rangle$, one obtains the simple algebraic
relation
\begin{equation}
m_{\omega}^{2}\,\langle\omega_{0}\rangle = g_{\omega}\,n_{B},
\label{eq:omega_field_eq}
\end{equation}
where $n_{B} = \langle\psi^{\dagger}\psi\rangle = n_{p} + n_{n}$ is the total
baryon density. The $\omega$ field therefore grows linearly with density and
acts as a uniform repulsive potential felt equally by all nucleons.

Substituting this mean field into the energy–momentum tensor gives the
contribution of the $\omega$ field to the energy density and pressure,
\begin{equation}
\epsilon_{\omega} = \frac{1}{2}\,m_{\omega}^{2}\,\langle\omega_{0}\rangle^{2},
\qquad
P_{\omega} = \frac{1}{2}\,m_{\omega}^{2}\,\langle\omega_{0}\rangle^{2}.
\label{eq:omega_eps_P}
\end{equation}
Using Eq.~\eqref{eq:omega_field_eq}, these expressions can be written directly
in terms of the baryon density as
\[
\epsilon_{\omega} = P_{\omega}
= \frac{1}{2}\,\Bigl(\frac{g_{\omega}}{m_{\omega}}\Bigr)^{2} n_{B}^{2}.
\]
The $\omega$ field thus contributes an equal, positive amount to the energy
density and pressure, reflecting the purely repulsive character of the
interaction it mediates. This repulsion stiffens the equation of state at high
density and plays a key role in determining the maximum mass of neutron stars.
The competition between the attractive $\sigma$ field and the repulsive
$\omega$ field establishes the saturation point of nuclear matter and controls
the overall stiffness of the equation of state in the RMF approximation.

\subsection{Isovector field term}
\label{subsec:rho_term}

The $\rho$ meson introduces the dependence of the nuclear interaction on
isospin asymmetry, that is, on the difference between the neutron and proton
densities. It provides an additional repulsive contribution that grows with the
neutron–proton imbalance and determines the symmetry energy of nuclear matter.
The Lagrangian for the $\rho$ field is
\[
\mathcal{L}_{\rho}
= -\frac{1}{4}\,\boldsymbol{\rho}_{\mu\nu}\!\cdot\!\boldsymbol{\rho}^{\mu\nu}
  + \frac{1}{2}\,m_{\rho}^{2}\,
    \boldsymbol{\rho}_{\mu}\!\cdot\!\boldsymbol{\rho}^{\mu},
\qquad
\boldsymbol{\rho}_{\mu\nu}
  = \partial_{\mu}\boldsymbol{\rho}_{\nu} - \partial_{\nu}\boldsymbol{\rho}_{\mu}.
\]
Here, $\boldsymbol{\rho}_{\mu}$ is an isovector field with three components in
isospin space. It couples to the third component of the nucleon isospin
operator, and in uniform matter only this component contributes. The Euler–Lagrange equation for $\boldsymbol{\rho}_{\mu}$ reads
\[
\partial_{\nu}\boldsymbol{\rho}^{\mu\nu}
+ m_{\rho}^{2}\,\boldsymbol{\rho}^{\mu}
= g_{\rho}\,\bar{\psi}\gamma^{\mu}\,\boldsymbol{\tau}\psi.
\]
In the RMF approximation, only the time component of the third isospin direction
remains nonzero, $\langle\rho_{03}\rangle$, while all others
vanish. The equation of motion then reduces to
\begin{equation}
m_{\rho}^{2}\,\langle\rho_{03}\rangle
= \frac{1}{2}\,g_{\rho}\,(n_{p}-n_{n}),
\label{eq:rho_field_eq}
\end{equation}
where $n_{p}$ and $n_{n}$ are the proton and neutron densities. The $\rho$
field is therefore directly proportional to the neutron–proton imbalance and
vanishes in symmetric nuclear matter where $n_{p}=n_{n}$.

The single–particle energies of nucleons include an additional term
$\pm \tfrac{1}{2} g_{\rho}\,\langle\rho_{03}\rangle$, which increases the energy of
neutrons relative to protons when the matter becomes neutron-rich. This effect
raises the energy of asymmetric configurations and defines the symmetry energy
coefficient of nuclear matter.

The energy density and pressure associated with the uniform $\rho$ field follow
from its classical potential energy,
\begin{equation}
\epsilon_{\rho} = \frac{1}{2}\,m_{\rho}^{2}\,\langle\rho_{03}\rangle^{2},
\qquad
P_{\rho} = \frac{1}{2}\,m_{\rho}^{2}\,\langle\rho_{03}\rangle^{2}.
\label{eq:rho_eps_P}
\end{equation}
Using Eq.~\eqref{eq:rho_field_eq}, these can be written in terms of the baryon
densities as
\[
\epsilon_{\rho} = P_{\rho}
= \frac{1}{8}\,
  \Bigl(\frac{g_{\rho}}{m_{\rho}}\Bigr)^{2}
  (n_{p}-n_{n})^{2}.
\]
This term always increases the total energy and pressure, providing a repulsive
contribution that penalizes large isospin asymmetry. In neutron-star matter,
where $n_{n}>n_{p}$, the $\rho$ meson is essential to ensure that the chemical
potentials of neutrons, protons, and leptons can satisfy
$\beta$-equilibrium and charge neutrality simultaneously. Its inclusion allows
the model to reproduce the empirical symmetry energy coefficient of nuclear
matter and thus to describe correctly the composition and stiffness of
asymmetric matter in neutron-star cores.

\subsection{Leptonic sector}
\label{subsec:leptons}

The leptons (electrons and muons) are included to ensure charge neutrality and
$\beta$–equilibrium in stellar matter. They do not participate in the strong
interaction and are therefore treated as free relativistic Fermi gases. The
corresponding part of the Lagrangian is
\[
\mathcal{L}_{\ell}
= \sum_{\lambda=e,\mu}
  \bar{\psi}_{\lambda}\bigl(i\gamma^{\mu}\partial_{\mu}
  - m_{\lambda}\bigr)\psi_{\lambda}.
\]
Each lepton species $\lambda$ occupies all momentum states up to its Fermi
momentum $p_{F,\lambda}$, defined by the number density
\[
n_{\lambda} = \frac{p_{F,\lambda}^{3}}{3\pi^{2}}.
\]
In cold, degenerate matter the leptons obey the dispersion relation
$e_{\lambda}(p)=\sqrt{p^{2}+m_{\lambda}^{2}}$, with chemical potential
$\mu_{\lambda}=\sqrt{p_{F,\lambda}^{2}+m_{\lambda}^{2}}$. The muon appears only
when the electron chemical potential exceeds the muon rest mass,
$\mu_{e}\ge m_{\mu}$; below that threshold, muons are absent and all negative
charge is carried by electrons.

The energy density and pressure of the leptons are obtained by integrating the
relativistic Fermi–gas expressions,
\begin{align}
\epsilon_{\ell}
&= \sum_{\lambda=e,\mu}\frac{1}{\pi^{2}}
   \int_{0}^{p_{F,\lambda}}\!dp\,p^{2}\sqrt{p^{2}+m_{\lambda}^{2}},
\\[0.4em]
P_{\ell}
&= \sum_{\lambda=e,\mu}\frac{1}{3\pi^{2}}
   \int_{0}^{p_{F,\lambda}}\!dp\,\frac{p^{4}}{\sqrt{p^{2}+m_{\lambda}^{2}}}.
\end{align}
These integrals have analytic forms that can be evaluated numerically or
approximated in the ultrarelativistic limit ($p_{F,\lambda}\!\gg\!m_{\lambda}$)
as $\epsilon_{\ell}\simeq3P_{\ell}\simeq p_{F,\lambda}^{4}/(4\pi^{2})$.

The leptons interact only through the electromagnetic field, which in bulk
neutral matter averages to zero, so their only role in the RMF approximation is to
balance the positive charge of the protons and to enforce the conditions of
chemical equilibrium. Charge neutrality requires
\[
n_{p} = n_{e} + n_{\mu},
\]
while $\beta$–equilibrium, maintained by weak interactions, imposes the chemical
relations
\[
\mu_{n} = \mu_{p} + \mu_{e}, \qquad
\mu_{e} = \mu_{\mu},
\]
the latter holding whenever muons are present. Together with the baryonic field
equations, these constraints determine the composition of the system at each
density.

\subsection{Self–consistent mean–field system and equation of state}
\label{subsec:rmf_selfconsistent}

The RMF approximation used above leads to a closed set of
algebraic relations that determine the composition and thermodynamic properties
of matter at each baryon density.  For uniform, cold matter the meson fields are
replaced by their constant expectation values $\langle\sigma\rangle$, $\langle\omega_{0}\rangle$ and
$\langle\rho_{03}\rangle$, while the leptons are described by free Fermi gases.  The
quantities to be determined self–consistently are the mean fields, the Fermi
momenta of all particle species, and the associated chemical potentials.

The three meson fields obey the mean–field equations
\begin{align}
m_{\sigma}^{2}\,\langle\sigma\rangle
+ m_{n}b\,(g_{\sigma}\langle\sigma\rangle)^{2}
+ c\,(g_{\sigma}\langle\sigma\rangle)^{3}
&= g_{\sigma}\,
   \sum_{B=n,p}\frac{1}{\pi^{2}}
   \int_{0}^{p_{F,B}}\!dp\,p^{2}\,
   \frac{m^{*}}{\sqrt{p^{2}+m^{*2}}},
\\[0.5em]
m_{\omega}^{2}\,\langle\omega_{0}\rangle
&= g_{\omega}\,(n_{p}+n_{n}),
\\[0.5em]
m_{\rho}^{2}\,\langle\rho_{03}\rangle
&= \tfrac{1}{2}\,g_{\rho}\,(n_{p}-n_{n}).
\end{align}
with the effective nucleon mass $m^{*}=m-g_{\sigma}\langle\sigma\rangle$.  At the same
time, the conditions of charge neutrality and $\beta$–equilibrium link the
baryonic and leptonic sectors,
\begin{align}
n_{p} &= n_{e} + n_{\mu},\\
\mu_{n} &= \mu_{p} + \mu_{e},\\
\mu_{e} &= \mu_{\mu},
\end{align}
where the chemical potentials are
\begin{align}
\mu_{n} &= g_{\omega}\,\langle\omega_{0}\rangle
          - \tfrac{1}{2}\,g_{\rho}\,\langle\rho_{03}\rangle
          + \sqrt{p_{F,n}^{2}+m^{*2}},\\[0.4em]
\mu_{p} &= g_{\omega}\,\langle\omega_{0}\rangle
          + \tfrac{1}{2}\,g_{\rho}\,\langle\rho_{03}\rangle
          + \sqrt{p_{F,p}^{2}+m^{*2}},\\[0.4em]
\mu_{e} &= \sqrt{p_{F,e}^{2}+m_{e}^{2}},\\[0.4em]
\mu_{\mu} &= \sqrt{p_{F,\mu}^{2}+m_{\mu}^{2}}.
\end{align}
For a chosen total baryon density $n_{B}=n_{p}+n_{n}$, these equations form a
nonlinear system that is solved iteratively.  One starts with a trial value of
$\bar\sigma$, computes $m^{*}$, determines the Fermi momenta consistent with
charge neutrality and $\beta$–equilibrium, and then updates the fields until
convergence is achieved.  The procedure yields, for each density point, the
composition of the matter $(n_{p},n_{n},n_{e},n_{\mu})$, the mean fields
$(\langle\sigma\rangle,\langle\omega_{0}\rangle,\langle\rho_{03}\rangle)$, and the corresponding thermodynamic
quantities.

The total energy density and pressure follow by adding the contributions from
all sectors:
\begin{align}
\epsilon &=
  \frac{1}{2}\,m_{\sigma}^{2}\,\langle\sigma\rangle^{2}
  + \frac{1}{3}\,m_{n}b\,(g_{\sigma}\langle\sigma\rangle)^{3}
  + \frac{1}{4}\,c\,(g_{\sigma}\langle\sigma\rangle)^{4}
  + \frac{1}{2}\,m_{\omega}^{2}\,\langle\omega_{0}\rangle^{2}
  + \frac{1}{2}\,m_{\rho}^{2}\,\langle\rho_{03}\rangle^{2}
\\[0.3em]
&\quad
  + \sum_{B=n,p}\frac{1}{\pi^{2}}
    \int_{0}^{p_{F,B}}\!dp\,p^{2}\sqrt{p^{2}+m^{*2}}
\\[0.3em]
&\quad
  + \sum_{\lambda=e,\mu}\frac{1}{\pi^{2}}
    \int_{0}^{p_{F,\lambda}}\!dp\,p^{2}
    \sqrt{p^{2}+m_{\lambda}^{2}},
\\[1.0em]
P &=
 -\,\frac{1}{2}\,m_{\sigma}^{2}\,\langle\sigma\rangle^{2}
 - \frac{1}{3}\,m_{n}b\,(g_{\sigma}\langle\sigma\rangle)^{3}
 - \frac{1}{4}\,c\,(g_{\sigma}\langle\sigma\rangle)^{4}
 + \frac{1}{2}\,m_{\omega}^{2}\,\langle\omega_{0}\rangle^{2}
 + \frac{1}{2}\,m_{\rho}^{2}\,\langle\rho_{03}\rangle^{2}
\\[0.3em]
&\quad
  + \sum_{B=n,p}\frac{1}{3\pi^{2}}
    \int_{0}^{p_{F,B}}\!dp\,\frac{p^{4}}{\sqrt{p^{2}+m^{*2}}}
\\[0.3em]
&\quad
  + \sum_{\lambda=e,\mu}\frac{1}{3\pi^{2}}
    \int_{0}^{p_{F,\lambda}}\!dp\,\frac{p^{4}}{\sqrt{p^{2}+m_{\lambda}^{2}}}.
\end{align}
These relations define a barotropic equation of state $P(\epsilon)$ that can be used
directly in the TOV equations to compute the global
structure of neutron stars.  Once calibrated to reproduce the empirical bulk
properties of symmetric nuclear matter at saturation, the RMF approximation provides
a consistent microscopic description of dense, asymmetric matter and forms the
basis for many modern neutron-star equations of state.
