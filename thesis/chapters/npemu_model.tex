\chapter{The \texorpdfstring{$npe\mu$}{npemu} model}
\label{chap:npe_mu}

In the previous chapters we developed and tested the structure equations for
neutron stars using simple models of matter. These models were useful for
building intuition about how gravity and pressure balance to form stable
configurations, and for exploring how different equations of state affect the
stellar mass and radius. However, at densities near and above the nuclear
saturation density, the assumptions of non–interacting particles become too
simple. The internal structure of nucleons and their mutual interactions start
to play a major role in determining the properties of matter.

To describe these effects in a practical way, we now turn to the relativistic
mean–field (RMF) approximation. This framework models dense nuclear matter through
average fields that represent the main attractive and repulsive parts of the
nuclear force \cite{glendenning2000,pogliano2017}. 
It provides a way to include essential interaction physics while
keeping the formulation simple enough for use in stellar calculations.

In this chapter we use the RMF approximation to describe matter composed of neutrons,
protons, electrons, and muons ($npe\mu$). By imposing charge neutrality and
$\beta$–equilibrium, this composition forms a minimal realistic model of a cold
neutron–star. By studying existing RMF equations of state for such matter,
we can examine how nuclear interactions influence the pressure and density
profiles and how this, in turn, affects the mass–radius relation obtained from
the structure equations.

\section{Bulk Properties of Nuclear Matter}
\label{sec:bulk_properties}

A useful way to test any microscopic model of nuclear interactions is to see
whether it can reproduce the known bulk properties of symmetric nuclear matter.
These are macroscopic quantities that characterize infinite, uniform matter
composed of equal numbers of neutrons and protons, in its most stable (saturated)
state \cite{glendenning2000}. They summarize how the nuclear force behaves on average and provide
empirical reference points for any equation of state.

The most important of these bulk properties are:
\begin{itemize}
  \item the \textbf{binding energy per nucleon} $B/A$, which measures how tightly the
        nucleons are bound together at equilibrium;
  \item the \textbf{saturation density} $\rho_{0}$, i.e.\ the baryon number density at which
        the energy per nucleon reaches its minimum;
  \item the \textbf{effective nucleon mass} $m^{*}$, which reflects how the scalar field
        modifies the nucleon rest mass in the medium;
  \item the \textbf{compression modulus} $K$, which quantifies how stiff or soft the
        matter is against uniform compression;
  \item the \textbf{symmetry energy coefficient} $a_{\mathrm{sym}}$, which describes how
        the energy increases when the number of neutrons and protons becomes unequal.
\end{itemize}

The simplest RMF description of such matter is the
\emph{$\sigma$–$\omega$ model} \cite{glendenning2000,pogliano2017}.
In this approach, the strong nuclear interaction is
represented by the exchange of two meson fields: an attractive scalar field
($\sigma$) and a repulsive vector field ($\boldsymbol{\omega}$). The scalar attraction binds the
nucleons, while the vector repulsion prevents the system from collapsing at high
density. The balance between these two effects produces saturation naturally, in
contrast to non–relativistic models where it must be inserted by hand.

The Lagrangian density of the linear $\sigma$–$\omega$ model is
\begin{align}
\mathcal{L}
&= \bar{\psi}\bigl[\gamma_{\mu}(i\partial^{\mu} - g_{\omega}\omega^{\mu})
      - (m - g_{\sigma}\sigma)\bigr]\psi
   + \tfrac{1}{2}(\partial_{\mu}\sigma\,\partial^{\mu}\sigma - m_{\sigma}^{2}\sigma^{2})
   - \tfrac{1}{4}\omega_{\mu\nu}\omega^{\mu\nu}
   + \tfrac{1}{2}m_{\omega}^{2}\omega_{\mu}\omega^{\mu},
\label{eq:lagrangian_sigmaomega}
\end{align}
where $\psi$ is the nucleon field, $\omega_{\mu\nu}=\partial_{\mu}\omega_{\nu}-\partial_{\nu}\omega_{\mu}$
is the field tensor of the vector meson, and $m$, $m_{\sigma}$, $m_{\omega}$ are the
nucleon and meson masses, respectively.

From Eq.~\eqref{eq:lagrangian_sigmaomega} it is evident that the interaction
between nucleons and the meson fields is governed solely by the coupling
constants $g_{\sigma}$ and $g_{\omega}$. Since there is one scalar and one vector
interaction channel, the theory contains only two independent coupling
strengths, which appear in the dimensionless ratios $g_{\sigma}/m_{\sigma}$ and
$g_{\omega}/m_{\omega}$. Consequently, only two bulk nuclear observables—typically the
binding energy per nucleon and the saturation density—can be fitted
independently. The other quantities then follow automatically as predictions
from the model, but the resulting values are not realistic. In particular, the
predicted compression modulus $K$ is much too large, typically around
$K \simeq 550\,\mathrm{MeV}$, whereas empirical analyses of nuclear resonances give
$K \approx 230\,\mathrm{MeV}$ \cite{glendenning2000}. The effective mass and
symmetry energy are also found to deviate significantly from experiment.

To correct these deficiencies, the model is extended in two main ways.
First, one introduces non–linear self–interactions of the scalar field
$\sigma$, adding cubic and quartic terms to the Lagrangian.
These modify the density dependence of the attractive potential and thereby
soften the equation of state, bringing the compression modulus into agreement
with empirical data.
Second, one includes an additional isovector meson field ($\boldsymbol{\rho}$), which
couples to the difference between neutron and proton densities.
The $\rho$ meson restores isospin symmetry and governs how the energy changes
with neutron–proton asymmetry, allowing the model to reproduce the correct
symmetry energy.

With these improvements, the model can be tuned to
reproduce all five key bulk properties listed above.
Once calibrated in this way, it provides a realistic and internally consistent
equation of state that can be used to describe the dense matter inside neutron–star
cores.

At this point the model can be generalized from symmetric nuclear matter to
neutron–rich matter, as found in the interiors of neutron stars. In such
environments, weak interactions continuously convert neutrons into protons and
leptons until the system reaches $\beta$–equilibrium, where
\[
\mu_{n} = \mu_{p} + \mu_{e} = \mu_{p} + \mu_{\mu}.
\]
To maintain overall charge neutrality, the total positive charge of protons must
be balanced by the negative charge of the leptons,
\[
n_{p} = n_{e} + n_{\mu}.
\]
Electrons and muons are therefore included as free, relativistic Fermi gases,
while the baryons (neutrons and protons) continue to interact through the
exchange of the $\sigma$, $\omega$, and $\rho$ mesons introduced above.

The resulting composition—neutrons ($n$), protons ($p$), electrons ($e$), and
muons ($\mu$)—defines the so–called \emph{$npe\mu$ model}. It represents the
minimal, physically consistent description of cold matter in the
stellar core. Within this framework, the baryonic sector is governed by the
RMF approximation tuned to reproduce nuclear bulk properties,
while the leptonic sector enforces charge neutrality and $\beta$–stability.
Together, these ingredients yield a barotropic equation of state $P(\varepsilon)$
that can be directly used in the Tolman–Oppenheimer–Volkoff equations to compute
the mass–radius relation of neutron stars \cite{glendenning2000,pogliano2017}.


\section{RMF Lagrangian Formulation}
\label{sec:rmf_lagrangian}

To translate the qualitative picture of the previous section into a quantitative
framework, we formulate the RMF approximation in terms of a
Lagrangian density that describes the interactions among nucleons and meson
fields. From this Lagrangian, the field equations and thermodynamic quantities
follow in a systematic way through the Euler–Lagrange formalism. In the present
context, the aim is not to explore field theory in detail, but rather to obtain
a practical and self–consistent expression for the energy density and pressure
of uniform matter in $\beta$–equilibrium.

In RMF theory, the total Lagrangian is written as the sum of contributions from
each sector,
\begin{equation}
\mathcal{L}
= \mathcal{L}_{N}
+ \mathcal{L}_{\sigma}
+ \mathcal{L}_{\omega}
+ \mathcal{L}_{\rho}
+ \mathcal{L}_{\ell},
\label{eq:L_total}
\end{equation}
where $\mathcal{L}_{N}$ denotes the nucleonic Dirac term, 
$\mathcal{L}_{\sigma}$, $\mathcal{L}_{\omega}$, and $\mathcal{L}_{\rho}$ are the
scalar, isoscalar–vector, and isovector–vector meson sectors, respectively, and
$\mathcal{L}_{\ell}$ accounts for the free leptons (electrons and muons).  The
explicit expressions are
\begin{align}
\mathcal{L}_{N}
&= 
\bar{\psi}\!\left[
i\gamma^{\mu}\!\left(\partial_{\mu}
+ i g_{\omega}\,\omega_{\mu}
+ \tfrac{1}{2} g_{\rho}\,\boldsymbol{\tau}\!\cdot\!\boldsymbol{\rho}_{\mu}\right)
- \bigl(m - g_{\sigma}\sigma\bigr)
\right]\!\psi,
\label{eq:L_N}\\[0.5em]
\mathcal{L}_{\sigma}
&=
\frac{1}{2}\bigl(\partial_{\mu}\sigma\,\partial^{\mu}\sigma - m_{\sigma}^{2}\sigma^{2}\bigr)
- U(\sigma),
\qquad
U(\sigma)
= \frac{1}{3}\,m_{n}b(g_{\sigma}\sigma)^{3}
+ \frac{1}{4}\,c(g_{\sigma}\sigma)^{4},
\label{eq:L_sigma}\\[0.5em]
\mathcal{L}_{\omega}
&=
-\frac{1}{4}\omega_{\mu\nu}\omega^{\mu\nu}
+\frac{1}{2}m_{\omega}^{2}\omega_{\mu}\omega^{\mu},
\qquad
\omega_{\mu\nu}=\partial_{\mu}\omega_{\nu}-\partial_{\nu}\omega_{\mu},
\label{eq:L_omega}\\[0.5em]
\mathcal{L}_{\rho}
&=
-\frac{1}{4}\boldsymbol{\rho}_{\mu\nu}\!\cdot\!\boldsymbol{\rho}^{\mu\nu}
+\frac{1}{2}m_{\rho}^{2}\boldsymbol{\rho}_{\mu}\!\cdot\!\boldsymbol{\rho}^{\mu},
\qquad
\boldsymbol{\rho}_{\mu\nu}
=\partial_{\mu}\boldsymbol{\rho}_{\nu}-\partial_{\nu}\boldsymbol{\rho}_{\mu},
\label{eq:L_rho}\\[0.5em]
\mathcal{L}_{\ell}
&=
\sum_{\lambda=e,\mu}
\bar{\psi}_{\lambda}\bigl(i\gamma^{\mu}\partial_{\mu}-m_{\lambda}\bigr)\psi_{\lambda}.
\label{eq:L_leptons}
\end{align}
The constants $m$, $m_{\sigma}$, $m_{\omega}$, and $m_{\rho}$ are the masses of
the nucleon and mesons, and $g_{\sigma}$, $g_{\omega}$, $g_{\rho}$ are the
respective coupling constants.  The parameters $b$ and $c$ control the
non–linear self–interactions of the scalar field and are fixed empirically to
reproduce the saturation properties of nuclear matter
\cite{glendenning2000}.

In the following sections, we examine each sector in turn. By using the
RMF approximation, we derive the field
equations, the single–particle spectrum, and the corresponding contributions to
the total energy density and pressure. 

\subsection{Nucleon term}
\label{subsec:nucleon_term}

The nucleonic part of the Lagrangian, Eq.~\eqref{eq:L_N}, describes a relativistic
fermion field interacting with meson fields through Yukawa-type couplings.
Varying the Lagrangian with respect to $\bar\psi$ gives the Dirac equation
\begin{equation}
\Bigl[\gamma^\mu\!\bigl(i\partial_\mu - g_\omega\,\omega_\mu - \tfrac{1}{2}g_\rho\,\boldsymbol{\tau}\!\cdot\!\boldsymbol{\rho}_\mu\bigr)
 - (m - g_\sigma\,\sigma)\Bigr]\psi = 0.
\end{equation}
In the RMF approximation appropriate for infinite, uniform matter, spatial
gradients vanish and only the temporal components of the vector fields remain
nonzero. The expectation values of the meson fields are denoted
$\langle\sigma\rangle$, $\langle\omega_0\rangle$, and
$\langle\rho_{03}\rangle$, while the corresponding spatial
components vanish. The nucleons then move independently in the presence of
constant background fields that represent the average effect of their
interactions. The expectation value of the scalar field shifts the nucleon
rest mass to an effective value
\begin{equation}
m^{*} = m - g_\sigma\,\langle\sigma\rangle,
\end{equation}
which reduces the single-particle energy relative to the vacuum and
corresponds to an attractive potential. The time components of the vector
fields, on the other hand, act as constant potentials that add repulsive
energy shifts proportional to the baryon and isospin densities. For a nucleon
species $B$ (proton or neutron) with isospin projection $I_3 = +\tfrac{1}{2}$
for protons and $I_3 = -\tfrac{1}{2}$ for neutrons, the single-particle energy
spectrum becomes
\begin{equation}
e_B(p) = g_\omega\,\langle\omega_0\rangle + I_B\,g_\rho\,\langle\rho_{03}\rangle
         + \sqrt{p^{2} + m^{*2}}.
\label{eq:single_particle_energy}
\end{equation}
At zero temperature, each species fills all momentum states up to its Fermi
momentum $p_{F,B}$, giving the number densities
\[
n_B = \frac{p_{F,B}^{3}}{3\pi^{2}}, \qquad
n_B^{\mathrm{tot}} = n_p + n_n.
\]
The corresponding kinetic contribution to the total energy density follows from
the sum of the single-particle energies over occupied states,
\begin{equation}
\epsilon_{N}^{\mathrm{kin}}
  = \frac{1}{\pi^{2}}\sum_{B=n,p}
    \int_{0}^{p_{F,B}}\!dp\,p^{2}\sqrt{p^{2} + m^{*2}},
\label{eq:epsilon_nucleon}
\end{equation}
while the kinetic (degeneracy) pressure, obtained from the diagonal component of the
energy–momentum tensor or equivalently from the thermodynamic relation
$P = n_B^{2}\,d(\epsilon/n_B)/dn_B$, takes the form
\begin{equation}
P_{N}^{\mathrm{kin}}
  = \frac{1}{3\pi^{2}}\sum_{B=n,p}
    \int_{0}^{p_{F,B}}\!dp\,\frac{p^{4}}{\sqrt{p^{2} + m^{*2}}}.
\label{eq:pressure_nucleon}
\end{equation}
These two integrals represent the free Fermi–gas contributions modified by the
effective mass $m^{*}$. In evaluating them, one usually adopts the
\emph{no-sea approximation}, meaning that the negative–energy states of the
Dirac sea are omitted. Physically, this corresponds to subtracting the divergent
vacuum contribution from the filled negative–energy spectrum, effectively
renormalizing the zero of energy such that only excitations above the vacuum
contribute to the thermodynamics. This approximation is well justified in
nuclear matter, where the vacuum polarization effects are small compared with
the mean-field contributions from the occupied positive–energy states.

The vector fields enter
the single-particle energies in Eq.~\eqref{eq:single_particle_energy} as constant
shifts. Their effect on the thermodynamics appears not through the integrals
above, but through separate classical field–energy terms associated with
$\langle\omega_0\rangle$ and $\langle\rho_{03}\rangle$. These will be added when the corresponding
meson sectors are discussed below.

The chemical potentials of the nucleons, which determine the conditions for
$\beta$-equilibrium, follow directly from the Fermi energies. For each species,
\begin{align}
\mu_p &= g_\omega\,\langle\omega_0\rangle + \tfrac{1}{2}g_\rho\,\langle\rho_{03}\rangle
         + \sqrt{p_{F,p}^{2} + m^{*2}},\\
\mu_n &= g_\omega\,\langle\omega_0\rangle - \tfrac{1}{2}g_\rho\,\langle\rho_{03}\rangle
         + \sqrt{p_{F,n}^{2} + m^{*2}}.
\end{align}
The difference between neutron and proton chemical potentials is thus governed
by the $\rho$ field, while their common shift arises from the $\omega$ field.
The effective mass $m^{*}$ encapsulates the attractive scalar interaction,
and the competition between these scalar and vector terms determines the net
binding and saturation of nuclear matter. Together, Eqs.~\eqref{eq:epsilon_nucleon}
and \eqref{eq:pressure_nucleon} provide the nucleonic (kinetic) parts of the
energy density and pressure, to which the meson field energies will now be
added to obtain the full equation of state.

\subsection{Scalar field term}
\label{subsec:sigma_term}

The scalar $\sigma$ field represents the attractive part of the strong
interaction between nucleons. Its Lagrangian, Eq.~\eqref{eq:L_sigma}, contains a
quadratic mass term and additional cubic and quartic self–interactions,
\[
\mathcal{L}_{\sigma}
= \frac{1}{2}\bigl(\partial_{\mu}\sigma\,\partial^{\mu}\sigma - m_{\sigma}^{2}\sigma^{2}\bigr)
- \frac{1}{3}\,m_{n}b\,(g_{\sigma}\sigma)^{3}
- \frac{1}{4}\,c\,(g_{\sigma}\sigma)^{4}.
\]
The self–interaction $U(\sigma)$ softens the scalar attraction at high
density and is essential to reproduce realistic nuclear compressibility.
Varying the Lagrangian with respect to $\sigma$ yields the field equation
\begin{equation}
\partial_{\mu}\partial^{\mu}\sigma
+ m_{\sigma}^{2}\sigma
+ m_{n}b\,(g_{\sigma}\sigma)^{2}
+ c\,(g_{\sigma}\sigma)^{3}
= g_{\sigma}\,\bar{\psi}\psi.
\end{equation}
In uniform matter, spatial derivatives vanish and the field is constant.
Within the mean–field approximation we assume factorization of powers of the
field, $\langle\sigma^{n}\rangle \approx \langle\sigma\rangle^{n}$, which leads to
\begin{equation}
m_{\sigma}^{2}\,\langle\sigma\rangle
+ m_{n}b\,(g_{\sigma}\langle\sigma\rangle)^{2}
+ c\,(g_{\sigma}\langle\sigma\rangle)^{3}
= g_{\sigma}\,\langle\bar{\psi}\psi\rangle.
\label{eq:sigma_mf_eq}
\end{equation}
where $\langle\bar{\psi}\psi\rangle$ is the scalar density of nucleons,
\begin{equation}
\langle\bar{\psi}\psi\rangle
= \frac{1}{\pi^{2}}\!\sum_{B=n,p}
  \int_{0}^{p_{F,B}}\!dp\,p^{2}\,
  \frac{m^{*}}{\sqrt{p^{2}+m^{*2}}}.
\label{eq:scalar_density}
\end{equation}
This self–consistency relation couples the scalar mean field $\langle\sigma\rangle$ to the
occupied Fermi seas and must be solved together with the vector–meson and
equilibrium conditions at each baryon density.

The scalar field also contributes to the total energy density of the system.
Starting from the Lagrangian \eqref{eq:L_sigma} the corresponding Hamiltonian 
density follows from the Legendre transform
\[
\mathcal{H}_{\sigma}
= \pi_{\sigma}\,\dot{\sigma} - \mathcal{L}_{\sigma},
\qquad
\pi_{\sigma} = \frac{\partial\mathcal{L}_{\sigma}}{\partial\dot{\sigma}}
              = \dot{\sigma}.
\]
In static, uniform matter the time derivative $\dot{\sigma}$ and spatial
gradients vanish, so that $\pi_{\sigma}=0$ and the Hamiltonian density reduces to
\[
\mathcal{H}_{\sigma}
= \frac{1}{2}m_{\sigma}^{2}\sigma^{2} + U(\sigma).
\]
The energy density associated with the scalar field is then given by the
expectation value of this Hamiltonian density,
\begin{equation}
\epsilon_{\sigma}
= \frac{1}{2}m_{\sigma}^{2}\langle\sigma\rangle^{2}
  + \frac{1}{3}m_{n}b\,(g_{\sigma}\langle\sigma\rangle)^{3}
  + \frac{1}{4}c\,(g_{\sigma}\langle\sigma\rangle)^{4}.
\label{eq:epsilon_sigma}
\end{equation}
The corresponding pressure follows from the
energy–momentum tensor or from the thermodynamic identity
$P = \sum_i \mu_i n_i - \epsilon$, yielding
\begin{equation}
P_{\sigma}
= -\,\frac{1}{2}m_{\sigma}^{2}\langle\sigma\rangle^{2}
  - \frac{1}{3}m_{n}b\,(g_{\sigma}\langle\sigma\rangle)^{3}
  - \frac{1}{4}c\,(g_{\sigma}\langle\sigma\rangle)^{4}.
\label{eq:pressure_sigma}
\end{equation}
These expressions represent the pure field contributions of the scalar meson to
the total energy density and pressure. The $\sigma$ field lowers the total
energy by generating an attractive interaction between nucleons and thus plays
a central role in producing nuclear binding and saturation in the relativistic
framework.

\subsection{Vector field term}
\label{subsec:omega_term}

The vector $\omega$ field represents the short–range repulsive part of the
nuclear interaction. It couples to the conserved baryon current and provides a
repulsive potential that grows with density, counteracting the attraction from
the scalar field and ensuring that nuclear matter saturates at a finite
density. The corresponding part of the Lagrangian is
\[
\mathcal{L}_{\omega}
= -\frac{1}{4}\,\omega_{\mu\nu}\omega^{\mu\nu}
  + \frac{1}{2}\,m_{\omega}^{2}\,\omega_{\mu}\omega^{\mu},
\qquad
\omega_{\mu\nu} = \partial_{\mu}\omega_{\nu} - \partial_{\nu}\omega_{\mu}.
\]
Variation with respect to $\omega_{\mu}$ gives the field equation
\begin{equation}
\partial_{\nu}\omega^{\mu\nu} + m_{\omega}^{2}\omega^{\mu}
= g_{\omega}\,\bar{\psi}\gamma^{\mu}\psi.
\end{equation}
In uniform, static matter all derivatives vanish,
$\partial_{\nu}\omega^{\mu\nu} = 0$, and the field equation reduces to
\begin{equation}
m_{\omega}^{2}\,\langle\omega^{\mu}\rangle
= g_{\omega}\,\langle\bar{\psi}\gamma^{\mu}\psi\rangle.
\end{equation}
Because the system is at rest in the mean–field frame, the spatial components
$\langle\omega^{i}\rangle$ vanish and only the temporal component
$\langle\omega^{0}\rangle$ remains nonzero. This yields
\begin{equation}
m_{\omega}^{2}\,\langle\omega^{0}\rangle = g_{\omega}\,n_{B},
\label{eq:omega_field_eq}
\end{equation}
where $n_{B} = \langle\psi^{\dagger}\psi\rangle = n_{p} + n_{n}$ is the total
baryon density. The $\omega$ field therefore grows linearly with density and
acts as a uniform repulsive potential felt equally by all nucleons.

The corresponding energy density can be derived directly from the Hamiltonian
density of the $\omega$ field. Starting from the Lagrangian \eqref{eq:L_omega}
the canonical momentum is
$\pi_{\omega}^{\mu} = \frac{\partial\mathcal{L}_{\omega}}{\partial(\partial_{0}\omega_{\mu})}
= \omega^{0\mu}$.
In uniform, static matter, all derivatives of $\omega_{\mu}$ vanish, implying
$\pi_{\omega}^{\mu}=0$ and $\omega_{\mu\nu}=0$. The Hamiltonian density then
reduces to
\[
\mathcal{H}_{\omega} = \pi_{\omega}^{\mu}\dot{\omega}_{\mu} - \mathcal{L}_{\omega}
= \frac{1}{2}\,m_{\omega}^{2}\,\omega_{\mu}\omega^{\mu}.
\]
Taking the expectation value gives the energy density associated with the mean
field,
\begin{equation}
\epsilon_{\omega}
= \frac{1}{2}\,m_{\omega}^{2}\,\langle\omega^{0}\rangle^{2}.
\label{eq:epsilon_omega}
\end{equation}
Since the $\omega$ field behaves as a uniform classical field with vanishing
derivatives, its contribution to the pressure is obtained from the spatial
components of the energy–momentum tensor,
\begin{equation}
P_{\omega}
= \frac{1}{2}\,m_{\omega}^{2}\,\langle\omega^{0}\rangle^{2}.
\label{eq:pressure_omega}
\end{equation}
Using Eq.~\eqref{eq:omega_field_eq}, both quantities can be expressed in terms
of the baryon density as
\[
\epsilon_{\omega} = P_{\omega}
= \frac{1}{2}\,\Bigl(\frac{g_{\omega}}{m_{\omega}}\Bigr)^{2} n_{B}^{2}.
\]
The $\omega$ field thus contributes an equal, positive amount to the energy
density and pressure, reflecting the purely repulsive character of the
interaction it mediates. This repulsion stiffens the equation of state at high
density and plays a key role in determining the maximum mass of neutron stars.
The competition between the attractive $\sigma$ field and the repulsive
$\omega$ field establishes the saturation point of nuclear matter and controls
the overall stiffness of the equation of state in the RMF approximation.

\subsection{Isovector field term}
\label{subsec:rho_term}

The $\rho$ meson introduces the dependence of the nuclear interaction on
isospin asymmetry, that is, on the difference between the neutron and proton
densities. It provides an additional repulsive contribution that grows with the
neutron–proton imbalance and determines the symmetry energy of nuclear matter.
The Lagrangian for the $\rho$ field is
\[
\mathcal{L}_{\rho}
= -\frac{1}{4}\,\boldsymbol{\rho}_{\mu\nu}\!\cdot\!\boldsymbol{\rho}^{\mu\nu}
  + \frac{1}{2}\,m_{\rho}^{2}\,
    \boldsymbol{\rho}_{\mu}\!\cdot\!\boldsymbol{\rho}^{\mu},
\qquad
\boldsymbol{\rho}_{\mu\nu}
  = \partial_{\mu}\boldsymbol{\rho}_{\nu} - \partial_{\nu}\boldsymbol{\rho}_{\mu}.
\]
Here, $\boldsymbol{\rho}_{\mu}$ is an isovector field with three components in
isospin space. It couples to the third component of the nucleon isospin
operator, and in uniform matter only this component contributes. The Euler–Lagrange equation for $\boldsymbol{\rho}_{\mu}$ reads
\[
\partial_{\nu}\boldsymbol{\rho}^{\mu\nu}
+ m_{\rho}^{2}\,\boldsymbol{\rho}^{\mu}
= g_{\rho}\,\bar{\psi}\gamma^{\mu}\,\boldsymbol{\tau}\psi.
\]
In the RMF approximation, only the time component of the third isospin direction
remains nonzero, $\langle\rho^0_3\rangle$, while all others
vanish. The equation of motion then reduces to
\begin{equation}
m_{\rho}^{2}\,\langle\rho^0_3\rangle
= \frac{1}{2}\,g_{\rho}\,(n_{p}-n_{n}),
\label{eq:rho_field_eq}
\end{equation}
where $n_{p}$ and $n_{n}$ are the proton and neutron densities. The $\rho$
field is therefore directly proportional to the neutron–proton imbalance and
vanishes in symmetric nuclear matter where $n_{p}=n_{n}$.

The single–particle energies of nucleons include an additional term
$\pm \tfrac{1}{2} g_{\rho}\,\langle\rho^0_3\rangle$, which increases the energy of
neutrons relative to protons when the matter becomes neutron-rich. This effect
raises the energy of asymmetric configurations and defines the symmetry energy
coefficient of nuclear matter.

The energy density and pressure associated with the uniform $\rho$ field follow
from its classical potential energy,
\begin{equation}
\epsilon_{\rho} = \frac{1}{2}\,m_{\rho}^{2}\,\langle\rho^0_3\rangle^{2},
\qquad
P_{\rho} = \frac{1}{2}\,m_{\rho}^{2}\,\langle\rho^0_3\rangle^{2}.
\label{eq:rho_eps_P}
\end{equation}
Using Eq.~\eqref{eq:rho_field_eq}, these can be written in terms of the baryon
densities as
\[
\epsilon_{\rho} = P_{\rho}
= \frac{1}{8}\,
  \Bigl(\frac{g_{\rho}}{m_{\rho}}\Bigr)^{2}
  (n_{p}-n_{n})^{2}.
\]
This term always increases the total energy and pressure, providing a repulsive
contribution that penalizes large isospin asymmetry. In neutron-star matter,
where $n_{n}>n_{p}$, the $\rho$ meson is essential to ensure that the chemical
potentials of neutrons, protons, and leptons can satisfy
$\beta$-equilibrium and charge neutrality simultaneously. Its inclusion allows
the model to reproduce the empirical symmetry energy coefficient of nuclear
matter and thus to describe correctly the composition and stiffness of
asymmetric matter in neutron-star cores.

\subsection{Leptonic sector}
\label{subsec:leptons}

The leptons are included to ensure charge neutrality and
$\beta$–equilibrium in stellar matter. They do not participate in the strong
interaction and are therefore treated as free relativistic fermions. The
corresponding part of the Lagrangian is
\[
\mathcal{L}_{\ell}
= \sum_{\lambda=e,\mu}
  \bar{\psi}_{\lambda}\bigl(i\gamma^{\mu}\partial_{\mu}
  - m_{\lambda}\bigr)\psi_{\lambda}.
\]
Each lepton species $\lambda$ occupies all momentum states up to its Fermi
momentum $p_{F,\lambda}$, defined by the number density
\[
n_{\lambda} = \frac{p_{F,\lambda}^{3}}{3\pi^{2}}.
\]
In cold, degenerate matter the leptons obey the dispersion relation
$e_{\lambda}(p)=\sqrt{p^{2}+m_{\lambda}^{2}}$, with chemical potentials
$\mu_{\lambda}=\sqrt{p_{F,\lambda}^{2}+m_{\lambda}^{2}}$. The leptons interact
only through the electromagnetic field, which in bulk neutral matter averages
to zero. Their role in the RMF approximation is therefore to balance the
positive charge of the protons and to satisfy the conditions of chemical
equilibrium.

Charge neutrality requires
\[
n_{p} = n_{e} + n_{\mu},
\]
while weak interactions enforce $\beta$–equilibrium through
\[
\mu_{n} = \mu_{p} + \mu_{e}, \qquad
\mu_{e} = \mu_{\mu}.
\]
The second condition holds whenever muons are present. It implies that muons
begin to appear only once the electron chemical potential reaches the muon rest
mass, $\mu_{e} \ge m_{\mu}$; below this threshold
only electrons carry the negative charge.

The energy density and pressure of the leptons are obtained from the standard
relativistic Fermi–gas integrals,
\begin{align}
\epsilon_{\ell}
&= \frac{1}{\pi^{2}}\sum_{\lambda=e,\mu}
   \int_{0}^{p_{F,\lambda}}\!dp\,p^{2}\sqrt{p^{2}+m_{\lambda}^{2}},
\\[0.4em]
P_{\ell}
&= \frac{1}{3\pi^{2}}\sum_{\lambda=e,\mu}
   \int_{0}^{p_{F,\lambda}}\!dp\,\frac{p^{4}}{\sqrt{p^{2}+m_{\lambda}^{2}}}.
\end{align}
These expressions can be evaluated numerically or approximated in the
ultrarelativistic limit ($p_{F,\lambda}\!\gg\!m_{\lambda}$) as
$\epsilon_{\ell}\simeq3P_{\ell}\simeq p_{F,\lambda}^{4}/(4\pi^{2})$.

\subsection{Self–consistent mean–field system and equation of state}
\label{subsec:rmf_selfconsistent}

The RMF approximation used above leads to a closed set of
algebraic relations that determine the composition and thermodynamic properties
of matter at each baryon density.  For uniform, cold matter the meson fields are
replaced by their constant expectation values $\langle\sigma\rangle$, $\langle\omega_{0}\rangle$ and
$\langle\rho_{03}\rangle$, while the leptons are described by free Fermi gases.  The
quantities to be determined self–consistently are the mean fields, the Fermi
momenta of all particle species, and the associated chemical potentials.

The three meson fields obey the mean–field equations
\begin{align}
m_{\sigma}^{2}\,\langle\sigma\rangle
+ m_{n}b\,(g_{\sigma}\langle\sigma\rangle)^{2}
+ c\,(g_{\sigma}\langle\sigma\rangle)^{3}
&= g_{\sigma}\,
   \frac{1}{\pi^{2}}\sum_{B=n,p}
   \int_{0}^{p_{F,B}}\!dp\,p^{2}\,
   \frac{m^{*}}{\sqrt{p^{2}+m^{*2}}},
\\[0.5em]
m_{\omega}^{2}\,\langle\omega_{0}\rangle
&= g_{\omega}\,(n_{p}+n_{n}),
\\[0.5em]
m_{\rho}^{2}\,\langle\rho_{03}\rangle
&= \tfrac{1}{2}\,g_{\rho}\,(n_{p}-n_{n}).
\end{align}
with the effective nucleon mass $m^{*}=m-g_{\sigma}\langle\sigma\rangle$.  At the same
time, the conditions of charge neutrality and $\beta$–equilibrium link the
baryonic and leptonic sectors,
\begin{align}
n_{p} &= n_{e} + n_{\mu},\\
\mu_{n} &= \mu_{p} + \mu_{e},\\
\mu_{e} &= \mu_{\mu},
\end{align}
where the chemical potentials are
\begin{align}
\mu_{n} &= g_{\omega}\,\langle\omega_{0}\rangle
          - \tfrac{1}{2}\,g_{\rho}\,\langle\rho_{03}\rangle
          + \sqrt{p_{F,n}^{2}+m^{*2}},\\[0.4em]
\mu_{p} &= g_{\omega}\,\langle\omega_{0}\rangle
          + \tfrac{1}{2}\,g_{\rho}\,\langle\rho_{03}\rangle
          + \sqrt{p_{F,p}^{2}+m^{*2}},\\[0.4em]
\mu_{e} &= \sqrt{p_{F,e}^{2}+m_{e}^{2}},\\[0.4em]
\mu_{\mu} &= \sqrt{p_{F,\mu}^{2}+m_{\mu}^{2}}.
\end{align}
For a chosen total baryon density $n_{B}=n_{p}+n_{n}$, these equations form a
nonlinear system that is solved iteratively.  One starts with a trial value of
$\bar\sigma$, computes $m^{*}$, determines the Fermi momenta consistent with
charge neutrality and $\beta$–equilibrium, and then updates the fields until
convergence is achieved.  The procedure yields, for each density, the
composition of the matter $(n_{p},n_{n},n_{e},n_{\mu})$, the mean fields
$(\langle\sigma\rangle,\langle\omega_{0}\rangle,\langle\rho_{03}\rangle)$, and the corresponding thermodynamic
quantities.

The total energy density and pressure follow by adding the contributions from
all sectors:
\begin{equation}
\begin{aligned}
\epsilon &=
  \frac{1}{2}\,m_{\sigma}^{2}\,\langle\sigma\rangle^{2}
  + \frac{1}{3}\,m_{n}b\,(g_{\sigma}\langle\sigma\rangle)^{3}
  + \frac{1}{4}\,c\,(g_{\sigma}\langle\sigma\rangle)^{4}
  + \frac{1}{2}\,m_{\omega}^{2}\,\langle\omega^{0}\rangle^{2}
  + \frac{1}{2}\,m_{\rho}^{2}\,\langle\rho^{0}_{3}\rangle^{2} \\[0.3em]
&\quad
  + \frac{1}{\pi^{2}}\sum_{B=n,p}
    \int_{0}^{p_{F,B}}\!dp\,p^{2}\sqrt{p^{2}+m^{*2}} \\[0.3em]
&\quad
  + \frac{1}{\pi^{2}}\sum_{\lambda=e,\mu}
    \int_{0}^{p_{F,\lambda}}\!dp\,p^{2}
    \sqrt{p^{2}+m_{\lambda}^{2}}.
\end{aligned}
\label{eq:energy_density_total}
\end{equation}

\begin{equation}
\begin{aligned}
P &=
 -\,\frac{1}{2}\,m_{\sigma}^{2}\,\langle\sigma\rangle^{2}
 - \frac{1}{3}\,m_{n}b\,(g_{\sigma}\langle\sigma\rangle)^{3}
 - \frac{1}{4}\,c\,(g_{\sigma}\langle\sigma\rangle)^{4}
 + \frac{1}{2}\,m_{\omega}^{2}\,\langle\omega^{0}\rangle^{2}
 + \frac{1}{2}\,m_{\rho}^{2}\,\langle\rho^{0}_{3}\rangle^{2} \\[0.3em]
&\quad
  + \frac{1}{3\pi^{2}}\sum_{B=n,p}
    \int_{0}^{p_{F,B}}\!dp\,\frac{p^{4}}{\sqrt{p^{2}+m^{*2}}} \\[0.3em]
&\quad
  + \frac{1}{3\pi^{2}}\sum_{\lambda=e,\mu}
    \int_{0}^{p_{F,\lambda}}\!dp\,\frac{p^{4}}{\sqrt{p^{2}+m_{\lambda}^{2}}}.
\end{aligned}
\label{eq:pressure_total}
\end{equation}
These relations define a barotropic equation of state $P(\epsilon)$ that can be used
directly in the TOV equations to compute the macroscopic quantities
of neutron stars.  Once calibrated to reproduce the empirical bulk
properties of symmetric nuclear matter at saturation, the RMF approximation provides
a consistent microscopic description of dense, asymmetric matter and forms the
basis for many modern neutron-star equations of state.


\section{Numerical results}
\label{sec:npe_mu_results}

The numerical implementation of the $npe\mu$ model follows the mean–field
framework developed in the previous sections and is based closely on the code
structure described in the master’s thesis of Pogliano~\cite{pogliano2017}.
For each chosen mass density $\rho$ on a fixed grid, the coupled
nonlinear equations for the meson mean fields, Fermi momenta, and chemical
potentials are solved iteratively. At convergence, this procedure yields the
self–consistent composition $(n_{n},n_{p},n_{e},n_{\mu})$ and the corresponding
energy density and pressure. In this section we first present the resulting
composition of $npe\mu$ matter, and then use it to construct and analyse the
equation of state for the neutron–star core.

\subsection{Composition of \texorpdfstring{$npe\mu$}{npemu} matter}
\label{subsec:npemu_composition_results}

Figure~\ref{fig:npemu_composition} shows the equilibrium composition obtained
from the numerical solution of the mean–field system. The particle densities
are plotted on a logarithmic scale as functions of the total mass density $\rho$.
All quantities entering the plot are determined self–consistently from the
RMF field equations and the equilibrium conditions derived in
Sec.~\ref{subsec:rmf_selfconsistent}; no additional phenomenological input is
introduced at this stage.

\begin{figure}[ht]
  \centering
  \includegraphics[width=0.8\textwidth]{figures/npemu/comp}
  \caption[Composition of \texorpdfstring{$npe\mu$}{npemu} matter]{
    Equilibrium particle densities in the $npe\mu$ model as functions of the
    mass density $\rho$, obtained from the self–consistent RMF calculation.
    The curves show neutrons, protons, electrons, and muons on a logarithmic
    scale. At low densities only neutrons, a small proton component, and
    electrons are present. Once the electron chemical potential reaches the
    muon rest mass, muons appear and increasingly share the negative charge
    with electrons at higher densities.
  }
  \label{fig:npemu_composition}
\end{figure}

At the lowest densities shown, the system is overwhelmingly neutron–rich.
The proton density is suppressed by the symmetry–energy cost of creating
protons, which is controlled by the isovector $\rho$–meson coupling in
Eqs.~\eqref{eq:L_rho} and \eqref{eq:rho_field_eq}. A small but finite proton
density is nevertheless required by the equilibrium relations between nucleon
chemical potentials, and the electron density closely tracks it through the
charge–neutrality condition. In this regime the lepton sector is effectively
purely electronic; the electron chemical potential remains below the muon
rest mass and the muon density is zero in the solution.

As $\rho$ increases, the balance between the scalar attraction and vector
repulsion in the baryonic sector modifies the single–particle energies in
Eq.~\eqref{eq:single_particle_energy} and gradually drives up the proton
fraction. Through the equilibrium relations for the chemical potentials,
this growth feeds directly into the electron Fermi momentum. The electron
chemical potential $\mu_{e}$ eventually reaches the muon rest mass, 
at the density where the muon number density becomes nonzero, as seen 
in Fig.~\ref{fig:npemu_composition}.
Beyond this threshold it becomes favourable to populate muon states, and the
code finds a rapidly increasing muon density $n_{\mu}$.

Once muons are present, the negative charge is shared between electrons and
muons. Further increases in $\rho$ are then accommodated partly by raising
$n_{\mu}$ rather than pushing $\mu_{e}$ to ever larger values. This behaviour
is visible in the plot as a transfer of weight from the electron to the muon
curve at higher densities, while the total lepton density continues to follow
the proton density as required by charge neutrality. The precise way in which
the proton fraction grows with $\rho$, and hence how the lepton densities
split between electrons and muons, is governed by the density dependence of
the symmetry energy encoded in the $\rho$–meson sector,
Eqs.~\eqref{eq:rho_field_eq} and \eqref{eq:rho_eps_P}.

At the highest densities in Fig.~\ref{fig:npemu_composition}, the solution
approaches the characteristic composition of a neutron–star core in this
model: neutrons carry the bulk of the baryon number, the proton fraction is
non–negligible but remains subdominant, and the leptonic charge is shared
between electrons and muons in a ratio set by the equilibrium conditions and
the underlying RMF couplings. In this sense the plot provides a direct
numerical realization of the qualitative picture developed earlier: the
$\sigma$ and $\omega$ fields fix the overall binding and stiffness, the
$\rho$ field penalizes isospin asymmetry and controls the proton fraction,
and the lepton sector adjusts to enforce the equilibrium conditions.

\subsection{Naive polytropic extension}
\label{subsec:polytrope_results}
With the composition determined at each density, the next step is to
construct the corresponding equation of state $P(\epsilon)$ for uniform
$npe\mu$ matter. The numerical implementation solves the coupled mean–field
equations and equilibrium conditions together with the thermodynamic
relations of Sec.~\ref{subsec:rmf_selfconsistent}. For each mass density
$\rho$ on the grid, the code returns the converged meson fields, particle
densities, energy density, and pressure, thereby producing the EoS of the
uniform core.

However, the $npe\mu$ model is applicable only in the density range where
matter is homogeneous. Below the crust–core transition the system becomes
non–uniform: nuclei form a Coulomb lattice, neutron drip occurs, and the
microphysics is dominated by nuclear structure rather than the uniform
mean–field dynamics. The tabulated core EoS therefore terminates at its
lowest density point and cannot be used to describe the outer layers.

To obtain a complete EoS suitable for TOV integration, the core
must be supplemented by a separate description of the crust. This extension
provides the physically consistent low-density behaviour and ensures that the
pressure decreases smoothly to zero at the stellar surface, thereby allowing
the boundary conditions of the TOV problem to be satisfied.

As a first step towards extending the core $npe\mu$ EoS to
lower densities, we adopt a deliberately simple phenomenological
model for the crust. The idea is to replace the detailed
low-density microphysics by a single polytropic relation between pressure
and energy density,
\begin{equation}
  P_{\text{crust}}(\epsilon)
  = K\,\epsilon^{\Gamma},
\end{equation}
where $\Gamma$ controls the stiffness of the crust and $K$ is fixed by the
requirement that the polytrope matches continuously onto the RMF $npe\mu$ core
EoS at a chosen joining point. 
In practice, the matching point is taken from the low-density end of the
numerical table used internally by the code to define where the uniform
$npe\mu$ core calculation stops. The corresponding energy density
$\epsilon_{\mathrm{match}}$ and pressure $P_{\mathrm{match}}$ are then used to
fix the constant $K$ so that the polytrope connects continuously to the
$npe\mu$ core EoS.
\begin{equation}
  K = \frac{P_{\mathrm{match}}}{\epsilon_{\mathrm{match}}^{\Gamma}}.
\end{equation}
Below $\epsilon_{\mathrm{match}}$ the pressure is given by the polytrope,
and above $\epsilon_{\mathrm{match}}$ the EoS coincides with the uniform
$npe\mu$ core.

The numerical implementation uses a polytropic relation in the low density 
region while keeping the original $npe\mu$
core unchanged. In this way the red curve in
Fig.~\ref{fig:poly_eos} represents the full EoS with a polytropic crust
and an unchanged $npe\mu$ core; the other curves show, for comparison, the
pure core EoS and the version extended to lower densities with the original
tabulated crust.

\begin{figure}[ht]
  \centering
  \includegraphics[width=0.8\textwidth]{figures/npemu/poly_eos}
  \caption[Polytropic crust joined to \texorpdfstring{$npe\mu$}{npemu} core]{
    Equation of state with a single polytropic crust ($\Gamma = 1.2$)
    matched to the RMF $npe\mu$ core. The red curve shows the combined
    model: at low densities the pressure follows the polytropic relation,
    while at higher densities it coincides with the uniform $npe\mu$ core
    EoS. For comparison, the pure core EoS and the reference model with a
    tabulated crust are also indicated.
  }
  \label{fig:poly_eos}
\end{figure}

For the results shown here, the polytropic index was simply chosen by
inspection as $\Gamma = 1.2$. This value gives a low-density behaviour
that looks reasonable when plotted together with the core EoS and does not
produce any obvious pathologies in the resulting stellar models. Since the
polytropic crust is only intended as an agnostic representation of the
outer layers, and the detailed microphysics is not yet included, the exact
choice of $\Gamma$ is not critical at this stage.

The effect of this naive crust prescription on the stellar structure is
illustrated in Fig.~\ref{fig:poly_mr}, which shows the mass–radius
relation and the gravitational mass as a function of central energy
density for the combined polytrope + $npe\mu$ model.
The high-mass part of the sequence is essentially governed by
the pure core EoS, as expected: for massive stars the structure is
dominated by the inner core, and the detailed crust EoS contributes only a
small correction. The maximum mass and the radius
are therefore controlled almost entirely by the RMF $npe\mu$ core.

\begin{figure}[ht]
  \centering
  \includegraphics[width=1.0\textwidth]{figures/npemu/poly_mr}
  \caption[Mass–radius relation for polytropic crust model]{
    Left: mass–radius relation for the model with a polytropic crust
    ($\Gamma = 1.2$) matched to the RMF $npe\mu$ core. The solid part of
    the curve indicates stable configurations and the dashed part the
    unstable branch. Right: gravitational mass as a function of central
    energy density for the same model. The polytropic crust mainly affects
    the low-mass, large-radius end of the sequence, while the maximum mass
    and the high-mass branch remain controlled by the core EoS.
  }
  \label{fig:poly_mr}
\end{figure}

At lower masses the crust becomes comparatively more important, and the
polytropic choice leaves a clearer imprint on the mass–radius curve. As
the central density is reduced, the radius increases and the
mass decreases. This behaviour
is visible in the left panel of Fig.~\ref{fig:poly_mr}: below about
$1\,M_{\odot}$ the curve bends towards radii of order $20$–$25\,$km,
reflecting the increasing fractional thickness of the low-density
envelope. In terms of the right panel, the polytropic crust shapes the
low-density end of the $M(\epsilon_{c})$ relation but leaves the location
of the maximum mass, and the onset of instability, 
determined by the $npe\mu$ core description.

Overall, the naive polytropic extension provides a convenient way to close
the EoS at low densities and to explore how a generic crust-like envelope
affects the global stellar properties. However, the construction is
deliberately agnostic: the choice of $\Gamma$ is not tied to nuclear
microphysics, and the polytrope does not resolve the sequence of phases in
the outer and inner crust.


\subsection{Crust equation of state and final model}
\label{subsec:crust_results}

Below the crust--core transition density, neutron-star matter is no longer
uniform and cannot be described by the $npe\mu$ mean–field model. Instead,
the microphysics is governed by nuclear structure, Coulomb interactions, and
the equilibrium between nuclei, electrons, and (above neutron drip) free
neutrons. A realistic description of the outer layers must therefore follow
the established crust Equation of State (EoS) models originally developed by
Baym, Pethick, and Sutherland (BPS)~\cite{bps1971} and later refined in works
by Haensel and Pichon~\cite{haensel_pichon1994}, Friedman and
Pandharipande~\cite{fps1981}, and the Skyrme–Lyon (SLy) model~\cite{sly2001}.

At the lowest densities, cold catalyzed matter minimizes the Gibbs free
energy by forming a Coulomb lattice of ${}^{56}\mathrm{Fe}$ nuclei immersed in a
degenerate electron gas. As the pressure increases, heavier and increasingly
neutron-rich nuclei become energetically favored. Each change of the
ground-state nucleus corresponds to a first-order phase transition at a fixed
pressure, with a discrete jump in the energy density. Detailed calculations
identify roughly a dozen such transitions before reaching the next major
threshold~\cite{haensel_pichon1994}, as nuclei gradually approach the limits
of stability.

At a density of
\[
  \rho_{\mathrm{drip}} \simeq 4\times 10^{11}\,\mathrm{g\,cm^{-3}},
\]
the neutron chemical potential reaches the continuum threshold and neutrons
begin to drip out of nuclei. This marks the onset of the inner crust,
where the system consists of a lattice of extremely neutron-rich nuclei,
degenerate electrons, and a free neutron gas. With increasing density the
nuclear clusters grow in size and neutron excess, while the surrounding
neutron gas becomes more significant.

Close to the crust--core interface, typically around
$\rho_{\mathrm{cc}}\sim 10^{14}\,\mathrm{g\,cm^{-3}}$, the competition between
nuclear surface tension and Coulomb energy destabilizes spherical nuclei.
This produces a sequence of non-spherical nuclear shapes—rods, slabs, tubes,
and bubbles—collectively known as ``nuclear pasta''. Although these phases
have important implications for transport and elasticity, their effect on the
pressure is modest, and smooth parametrized EoSs such as FPS or SLy capture
the thermodynamics to sufficient accuracy for global stellar structure
calculations.

In this work we adopt the Friedman–Pandharipande–Skyrme (FPS) crust EoS
in the analytic form provided by the fit of Haensel and Pichon and compiled
in later works (see e.g.\ the parametrization reproduced in
Pogliano~\cite{pogliano2017}).
The fit expresses $\log_{10}P$ as a smooth function of $\log_{10}\rho$ using
sigmoid transitions between different density regimes:
\begin{equation}
\begin{aligned}
\tilde{P} =
\frac{a_{1} + a_{2}\tilde{\rho} + a_{3}\tilde{\rho}^{3}}{
1 + a_{4}\tilde{\rho}}\, f_{0}\!\left(a_{5}(\tilde{\rho}-a_{6})\right)
+ (a_{7} + a_{8}\tilde{\rho}) \, f_{0}\!\left(a_{9}(a_{10}-\tilde{\rho})\right)\\
+ (a_{11} + a_{12}\tilde{\rho}) \, f_{0}\!\left(a_{13}(a_{14}-\tilde{\rho})\right)
+ (a_{15} + a_{16}\tilde{\rho}) \, f_{0}\!\left(a_{17}(a_{18}-\tilde{\rho})\right),
\label{eq:fps_fit}
\end{aligned}
\end{equation}
where $\tilde{P}=\log_{10}(P/\mathrm{dyne\,cm^{-2}})$,
$\tilde{\rho}=\log_{10}(\rho/\mathrm{g\,cm^{-3}})$, and
\begin{equation}
  f_{0}(x)=\frac{1}{e^{x}+1}.
\end{equation}
The fit parameters $a_{1}$–$a_{18}$ are listed in Table~\ref{tab:fps_params}.

\begin{table}[ht]
  \centering
  \caption{Parameters of the FPS analytic crust EoS.}
  \label{tab:fps_params}
  \renewcommand{\arraystretch}{1.2}
  \begin{tabular}{|cccccc|}
    \hline
    $a_{1}$ & $a_{2}$ & $a_{3}$ & $a_{4}$ & $a_{5}$ & $a_{6}$ \\
    11.4950 & $-22.775$ & 1.5707 & 4.3 & 14.08 & 27.80 \\
    \hline
    $a_{7}$ & $a_{8}$ & $a_{9}$ & $a_{10}$ & $a_{11}$ & $a_{12}$ \\
    $-1.653$ & 1.50 & 14.67 & 6.22 & 6.121 & 0.005925 \\
    \hline
    $a_{13}$ & $a_{14}$ & $a_{15}$ & $a_{16}$ & $a_{17}$ & $a_{18}$ \\
    0.16326 & 6.48 & 11.4971 & 19.105 & 0.8938 & 6.54 \\
    \hline
  \end{tabular}
\end{table}


This parametrization provides a smooth and accurate description of the entire
crust, including neutron drip and the approach to the crust--core transition,
and is easily matched to the uniform $npe\mu$ core at
$\rho=\rho_{\mathrm{cc}}$.  
The resulting combined EoS is thermodynamically consistent: pressure and
chemical potentials are derived from a single underlying energy density and
satisfy the standard zero-temperature thermodynamic relations across the full
density range, including the matching region.  
It is also numerically stable under interpolation on the grid used in our TOV
solver, and can therefore be used directly in the stellar structure
calculations presented in the following subsections.

Figures~\ref{fig:crust_eos} and \ref{fig:crust_mr} show the full EoS and the
resulting stellar models when the FPS crust is used instead of the naive
polytrope. The crust part of the EoS (red curve in
Fig.~\ref{fig:crust_eos}) reproduces the characteristic sequence of low-density
phases and matches continuously onto the $npe\mu$ core. 
Notice also the expected phase-transition break around
$\rho \sim 4 \times 10^{11}\,\mathrm{g/cm^{3}}$,
where neutrons begin to drip out of nuclei.


\begin{figure}[ht]
  \centering
  \includegraphics[width=0.8\textwidth]{figures/npemu/crust_eos}
  \caption[Crust + core equation of state]{
    Full equation of state combining the FPS crust with the RMF $npe\mu$
    core. The crust (red curve) reproduces the sequence of low-density
    phases and joins continuously onto the uniform core matter. For comparison,
    the pure core EoS beyond the branching point is also shown.}
  \label{fig:crust_eos}
\end{figure}

The corresponding mass–radius relation is shown in
Fig.~\ref{fig:crust_mr}. As expected, the high-mass region of the sequence
is unchanged, since the structure of massive neutron stars is determined
almost entirely by the dense core and is insensitive to crust physics. The
differences appear only in the low-mass, large-radius part of the curve.
The polytropic crust produces comparatively larger radii at a given
mass, or equivalently lower masses at the same radius. 
This behaviour reflects the fact that the $\Gamma=1.2$ polytrope has
a somewhat higher pressure at low energy densities, making the outer layers
less compact. The FPS crust, with its lower pressure in the same regime,
yields more compact configurations and therefore lies above the polytropic
curve in the mass-radius plot.

The central-density relation shows essentially no difference between the two
models: both reach the same maximum mass at nearly identical central
densities. This confirms that the crust affects only the low-density
envelope and leaves the core-dominated part of the sequence unchanged.

\begin{figure}[ht]
  \centering
  \includegraphics[width=1.0\textwidth]{figures/npemu/crust_mr}
  \caption[Mass–radius relation with FPS crust]{
    Left: mass–radius relation for the combined FPS crust + RMF $npe\mu$
    core model. Right: gravitational mass as a function of central energy
    density. The crust affects the low-mass, large-radius part of the
    sequence while leaving the high-mass branch, and in particular the
    maximum mass, essentially unchanged.}
  \label{fig:crust_mr}
\end{figure}

Taken together, these results provide a complete equation of state that spans
the full range from the stellar surface to the centre. The RMF $npe\mu$ model
governs the core, where uniform nuclear matter is an excellent approximation,
while the FPS crust supplies the correct microphysical behaviour at
sub-nuclear densities. The resulting combined EoS is internally consistent and
removes the arbitrariness introduced by a naive polytropic extension at low
densities, and it yields radii in a realistic range for typical neutron-star
masses. At the same time, it does not satisfy all current astrophysical
constraints: in particular, the maximum mass predicted by this model falls
below the heaviest precisely measured neutron stars. This mismatch shows that,
although the model provides a useful and controlled baseline, it is not fully
satisfactory as a quantitative description of neutron-star matter.

\subsection{Comparison with observationally inferred EoS and mass--radius constraints}
\label{subsec:bands_results}

To assess the astrophysical plausibility of the combined FPS~+~$npe\mu$ equation of state,
we compare our model with observationally inferred constraints obtained in the recent
semiparametric analysis of Ng \textit{et al.}~\cite{Ng_2025}.
Their work synthesizes information from three independent classes of neutron-star 
observations: (i) precise radio timing of massive pulsars (PSRs), (ii) gravitational-wave
tidal deformability measurements from binary neutron-star mergers (GWs), and
(iii) simultaneous mass--radius measurements from NICER X-ray pulse-profile modelling.  
These data are incorporated through a hierarchical Bayesian framework that produces
posterior distributions for the pressure--density relation and for the mass--radius curve.  
The resulting ``PSRs~+~GWs~+~X-ray'' credible bands represent one of the most up-to-date and
model-independent constraints available.

The observational bands used here are reconstructed from the public data
release accompanying Ng~\textit{et al.}~\cite{Ng_2025}, using the
\texttt{NLSLTR\_EOS\_PSR+GW+XRay\_posterior\_samples.h5} dataset. This file
contains a posterior ensemble of equations of state conditioned on massive
radio pulsars, gravitational-wave tidal deformabilities, and NICER X-ray
mass--radius measurements (the ``PSR+GW+X-ray'' posterior of their analysis).
For each posterior sample, the \texttt{eos} group provides a tabulated
pressure--energy-density relation, which we use to construct the
pressure--density band shown below.
In all figures we restrict the axes to the same density, pressure, mass, and
radius ranges displayed in Fig.~4 of Ng~\textit{et al.}, and we limit the
datasets to the 90\% credible range as in the original publication.

Figure~\ref{fig:eos_band} shows the pressure--density relation of the
$npe\mu$ core EoS together with the observed PSR+GW+X-ray band from
Ng~\textit{et al.}. Only the uniform-matter part of the model is used here.
For context, the crust--core transition of the crust models we have used lies around
$\log_{10}\rho \approx 14$, which is near the lower end of the density range
covered by the Ng~\textit{et al.} band.

\begin{figure}[ht]
  \centering
  \includegraphics[width=0.80\textwidth]{figures/npemu/eos_band}
  \caption[Core EoS compared with observational posterior band]{
    Pressure--density relation of the RMF $npe\mu$ core EoS compared with the 
    90\% PSR+GW+X-ray posterior band from Ng \textit{et al.}.}
  \label{fig:eos_band}
\end{figure}

Across most of the range the $npe\mu$ core lies comfortably inside the
posterior band. A mild deviation appears toward the lower-density side of the
mid-range region, where the model is slightly above the posterior band. 
This indicates that the RMF model we have used may omit softening mechanisms 
that would reduce the pressure near and somewhat above the crust--core 
transition density.
There is also a very slight deviation around $\log_{10}\rho \sim 15$, where the model
is below the band.
Nevertheless, the core EoS remains mostly within the 90\% credible region.

\smallskip

For the mass--radius comparison we complement the $npe\mu$ model with a set of 
individual neutron stars whose masses are known with high precision.  
These systems—PSR~J0348+0432~\cite{Antoniadis_2013_J0348},
PSR~J1614$-$2230~\cite{Demorest_2010_J1614},
PSR~J2215+5135~\cite{Linares_2018_J2215}, and
PSR~J0952$-$0607~\cite{Romani_2022_J0952}—are all in binary orbits where the 
geometry and orbital motion can be measured accurately.  
The mass determinations rely on purely gravitational and kinematic effects, such as 
the relativistic time delay experienced by the pulsar signal as it passes near the 
companion, or the orbital motion of the companion star measured in optical spectra.  
Because these techniques depend only on well-understood gravity and orbital 
mechanics, the resulting masses are essentially free from assumptions about the 
equation of state.  All four objects are found to be very massive, close to or 
exceeding \({2\,M_{\odot}}\).  
This makes them especially important: any theoretical equation of state must be able 
to support at least this mass, and these systems therefore provide a robust lower 
bound on the maximum mass in any mass--radius relation.

In addition, NICER X-ray pulse-profile modelling has yielded simultaneous
mass and radius estimates for two rotation-powered millisecond pulsars:
PSR J0030+0451~\cite{Miller_2019_J0030,Riley_2019_J0030} and
PSR J0740+6620~\cite{Miller_2021_J0740,Salmi_2024_J0740}.
The NICER analyses rely on detailed modelling of the surface hot-spot geometry,
beaming, and atmosphere physics and are therefore more model dependent than
radio timing, but they currently provide some of the most informative
constraints on radii in the mass range $\sim 1.3$--$2.1\,M_{\odot}$.
Independent analyses by different groups, using different waveform models,
give mutually consistent results for both pulsars, which increases confidence
in the overall radius scale.

Figure~\ref{fig:mr_band} shows the mass--radius relation for the $npe\mu$
model together with these observational constraints.  The solid and dashed curves
represent the stable and unstable branches of the TOV solutions, respectively.
Horizontal shaded bands indicate the $1\sigma$ mass ranges of the four massive
pulsars, while the NICER measurements for PSR~J0030+0451 and PSR~J0740+6620 are
plotted as crosses with $1\sigma$ error bars in both mass and radius.

\begin{figure}[ht]
  \centering
  \includegraphics[width=0.85\textwidth]{figures/npemu/mr_band_obs}
  \caption[Mass--radius curve compared with observational constraints]{
    Mass--radius relation for the $npe\mu$ model compared with selected
    observational constraints:
    horizontal bands show the measured masses of the pulsars
    J0348+0432, J1614$-$2230, J2215+5135, and J0952$-$0607
    (central value $\pm 1\sigma$), while crosses with error bars
    represent the NICER mass--radius measurements for
    J0030+0451 and J0740+6620.  Stable configurations are shown with
    solid lines, unstable ones with dashed lines.}
  \label{fig:mr_band}
\end{figure}

The comparison in Fig.~\ref{fig:mr_band} shows a clear mismatch between the
$npe\mu$ model and the observational constraints.  First, the maximum mass
supported by the model sequence remains below the highest-mass pulsars in the
sample, in particular the very massive systems J2215+5135 and J0952$-$0607,
whose central values lie well above $2\,M_{\odot}$.
Thus, in its present form the $npe\mu$ model does not reproduce the full range
of neutron-star masses inferred from observations.

Second, the model mass--radius curve does not pass through the NICER mass--radius
contours.  At the masses of J0030+0451 and J0740+6620, the theoretical sequence
predicts radii that are systematically offset from the central values of the
NICER measurements, and the observed crosses lie outside the narrow region
covered by the stable branch of the $npe\mu$ model.  This indicates that,
even if the maximum mass problem were remedied, the simple nucleonic RMF model
with an FPS crust would still struggle to match the combined mass and radius
information.

These discrepancies are not surprising in view of the simplicity of the
underlying microphysics. The $npe\mu$ model includes only nucleons and leptons
in uniform matter and neglects additional degrees of freedom and many-body
correlations that may become important in the dense cores of massive neutron
stars. The present comparison therefore serves mainly to show
that, while the minimal $npe\mu$ framework provides a useful baseline and is
compatible with the semiparametric EoS constraints of Ng~\textit{et al.},
it is not sufficient to account quantitatively for the current mass--radius
measurements.
