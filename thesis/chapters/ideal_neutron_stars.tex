\chapter{Ideal Neutron Stars}
\label{chap:ideal_neutron_stars}

\section{Introduction}

The concept of an \emph{ideal neutron star} provides a minimal
theoretical model for a relativistic compact star.
In this picture the stellar interior is assumed to consist solely
of neutrons, treated as a cold, non--interacting Fermi gas.
This construction parallels the classic white dwarf model,
where support against gravity is provided by the electron degeneracy
pressure. The key distinction is that neutron stars reach densities
so high that the gravitational field itself is extremely strong,
requiring a treatment within general relativity.
The appropriate framework is therefore given by the TOV equations
introduced in Chapter~\ref{chap:tov}.

To close the TOV system, an EOS relating pressure $P$ and energy density
$\epsilon$ must be specified.
Here we adopt the ideal Fermi gas EOS developed in
Chapter~\ref{chap:fermi}, specialized to neutrons of mass $m_{n}$.
Substituting this EOS into the TOV equations yields a complete system
that can be solved numerically.

The aim of this chapter is therefore to construct and analyze such ideal neutron
star models, with the focus on the numerical implementation and results.

\section{Numerical Setup}

The TOV equations derived in Chapter~\ref{chap:tov} describe the
radial dependence of the pressure $P(r)$ and enclosed mass $m(r)$ in a
static, spherically symmetric star. To integrate these equations we
must specify the boundary conditions. They are
\[
m(0)=0, \qquad P(0)=P_c, \qquad P(R)=0,
\]
where $P_c$ is the central pressure and $R$ is the stellar radius at
which the pressure vanishes. The corresponding gravitational mass is
then $M=m(R)$.

In Chapter~\ref{chap:fermi} we obtained this relation for a degenerate,
zero--temperature Fermi gas. Specializing to neutrons of mass $m_N$,
the EOS can be expressed in parametric form
\[
P = P(x_F), \qquad \epsilon = \epsilon(x_F),
\]
with $x_F=p_F/m_N$ the dimensionless Fermi momentum.
Although this parametric representation is convenient analytically,
the numerical solver expects $\epsilon(P)$. We therefore invert
$P(x_F)$ numerically to obtain $x_F(P)$, and then evaluate
$\epsilon(x_F)$ accordingly. This guarantees a consistent mapping
between $P$ and $\epsilon$ across all density regimes.

For numerical stability it is advantageous to work with dimensionless
variables. We introduce reference scales for radius, mass, and energy
density,
\[
r_0 = \frac{1}{\sqrt{G \epsilon_0}}, \qquad
m_0 = \frac{1}{\sqrt{G^3 \epsilon_0}}, \qquad
\epsilon_0 = m_N^4.
\]
All quantities are then expressed in units of $(r_0, m_0, \epsilon_0)$,
so that
\[
\hat r = r/r_0, \qquad
\hat m = m/m_0, \qquad
\hat P = P/\epsilon_0, \qquad
\hat \epsilon = \epsilon/\epsilon_0.
\]
In terms of these variables the TOV system takes a dimensionless form
suitable for numerical integration. This rescaling avoids very large
or small floating--point numbers and improves the stability of the
Runge--Kutta solver.

In summary, the numerical setup consists of:
\begin{enumerate}
  \item The TOV equations, integrated outward from the center with 
  initial conditions $m(0)=0$ and $P(0)=P_c$, and terminated when 
  the surface condition $P(R)=0$ is reached. The corresponding 
  radius $R$ and gravitational mass $M=m(R)$ then characterize 
  the stellar model.
  \item The ideal neutron gas EOS $\epsilon(P)$, obtained from the
  parametric Fermi gas expressions of Chapter~\ref{chap:fermi}.
  \item Dimensionless variables to reduce numerical instabilities.
\end{enumerate}
Together these ingredients define a well--posed system suitable for
numerical solution.

\section{Numerical Method}

The numerical integration of the TOV equations with the ideal neutron
gas EOS is carried out using a Python program originally developed by
Sletmoen as part of his Master's thesis~\cite{sletmoen2022}.
In this work the code is used without modification to generate stellar
models for the case of ideal neutron matter.

The implementation proceeds as follows:
\begin{itemize}
  \item The coupled ODEs for $P(r)$ and $m(r)$ are integrated outward
  from $r=0$ with initial values $P(0)=P_c$ and $m(0)=0$.
  \item A Runge--Kutta algorithm with adaptive step size is used to
  advance the solution.
  \item The integration terminates once $P(r)$ vanishes, which defines
  the stellar radius $R$. The corresponding $m(R)$ gives the
  gravitational mass $M$.
\end{itemize}
Repeating the calculation for a range of central pressures $P_c$
generates a sequence of stellar configurations. From these solutions
one can extract the global properties of the star --- quantities defined
at the surface that characterize the star as a whole --- namely its mass
$M=m(R)$ and radius $R$ where $P(R)=0$. These global results can then be
combined into the mass--radius relation. In addition, the solver also
provides access to the internal structure, meaning the radial profiles
of pressure $P(r)$ and energy density $\epsilon(r)$ throughout the
stellar interior. Such internal results reveal how matter is distributed
and how pressure balances gravity at each radius, complementing the
global description. In the following section we present both global and
internal results for ideal neutron stars.


\section{Results}

The numerical solutions of the TOV equations with the ideal neutron gas
EOS allow us to construct the mass--radius relation of ideal neutron
stars and to examine the internal pressure profiles.

\subsection*{Mass--radius relation}

Figure~\ref{fig:mr_neutron} shows the mass--radius curves obtained in
different levels of approximation. For comparison, both Newtonian
hydrostatic equilibrium and the fully relativistic TOV equations are
solved, each with two variants of the EOS: the non--relativistic limit
and the general relativistic dispersion relation. This yields four
distinct curves. The Newtonian models predict unrealistically large
masses and radii, while the relativistic treatment produces the expected
turnover in the $M$--$R$ diagram, signaling the existence of a maximum
stable mass. The difference between using the approximate
non--relativistic EOS and the full relativistic form is modest at low
central pressures but becomes significant as $P_c$ increases.

\begin{figure}[ht]
  \centering
  \includegraphics[width=0.7\textwidth]{figures/mass_radius_ideal_neutron_star.pdf}
  \caption[Mass--radius relation for ideal neutron stars]{
  Mass--radius relation for ideal neutron stars. Four models are
  compared: Newtonian hydrostatic equilibrium with non--relativistic and
  general EOS, and relativistic TOV equations with the same two EOS
  inputs. Color indicates the central pressure $P_c$.}
  \label{fig:mr_neutron}
\end{figure}

\subsection*{Pressure profiles}

Figure~\ref{fig:profiles_neutron} displays radial profiles of the
normalized pressure $P(r)/P_c$ for a sequence of central pressures
spanning several orders of magnitude. The overall behavior is smooth,
with $P(r)$ decreasing monotonically from the center to the surface.
For higher $P_c$ the profiles become more centrally peaked, reflecting
the stronger gravitational compression. These results are consistent
with the qualitative expectation that more massive stars are also more
compact.

\begin{figure}[ht]
  \centering
  \includegraphics[width=0.7\textwidth]{figures/pressure_profile_ideal_neutron_star.pdf}
  \caption[Pressure profiles of ideal neutron stars]{
  Radial pressure profiles $P(r)/P_c$ for ideal neutron stars with
  different central pressures $P_c$, shown on a logarithmic color scale.}
  \label{fig:profiles_neutron}
\end{figure}
