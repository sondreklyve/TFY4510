\chapter{Ideal Neutron Stars}
\label{chap:ideal_neutron_stars}

\section{Introduction}

The concept of an \emph{ideal neutron star} provides a minimal
theoretical model for a relativistic compact star.
Here the term ``ideal'' refers to the use of the simplest, non--interacting
Fermi gas equation of state, representing matter at zero temperature.
The composition is taken to be pure neutrons for simplicity,
in analogy with the classic white dwarf model where support against
gravity arises from the electron degeneracy pressure.
The key distinction is that neutron stars reach densities so high
that the gravitational field itself becomes extremely strong,
requiring a treatment within general relativity.
The appropriate framework is therefore given by the TOV equations
introduced in Chapter~\ref{chap:tov}.

The aim of this chapter is to construct and analyze such ideal neutron
star models, with the focus on the numerical implementation and results.

\section{Numerical Setup}

The TOV equations derived in Chapter~\ref{chap:tov} describe the
radial dependence of the pressure $P(r)$ and enclosed mass $m(r)$ in a
static, spherically symmetric star. To integrate these equations we
must specify the boundary conditions. They are
\[
m(0)=0, \qquad P(0)=P_c, \qquad P(R)=0,
\]
where $P_c$ is the central pressure and $R$ is the stellar radius at
which the pressure vanishes. The corresponding gravitational mass is
then $M=m(R)$.

In Chapter~\ref{chap:fermi} we obtained the EOS for a degenerate,
zero--temperature Fermi gas. Specializing to neutrons of mass $m_N$,
the EOS can be expressed in parametric form
\[
P = P(x_F), \qquad \epsilon = \epsilon(x_F),
\]
with $x_F=p_F/m_N$ the dimensionless Fermi momentum.
Although this parametric representation is convenient analytically,
the numerical solver expects $\epsilon(P)$. We therefore invert
$P(x_F)$ numerically to obtain $x_F(P)$, and then evaluate
$\epsilon(x_F)$ accordingly. This guarantees a consistent mapping
between $P$ and $\epsilon$ across all density regimes.

For numerical stability it is advantageous to work with dimensionless
variables. We introduce reference scales for length, mass, and energy
density,
\[
r_0 = \frac{1}{\sqrt{G \epsilon_0}}, \qquad
m_0 = \frac{1}{\sqrt{G^3 \epsilon_0}}, \qquad
\epsilon_0 = m_N^4.
\]
All quantities are then expressed in units of $(r_0, m_0, \epsilon_0)$,
so that
\[
\hat r = r/r_0, \qquad
\hat m = m/m_0, \qquad
\hat P = P/\epsilon_0, \qquad
\hat \epsilon = \epsilon/\epsilon_0.
\]
In terms of these variables the TOV system takes a dimensionless form
suitable for numerical integration. This rescaling avoids very large
or small floating--point numbers and improves the stability of the
Runge--Kutta solver.

In summary, the numerical setup consists of:
\begin{enumerate}
  \item The TOV equations, integrated outward from the center with 
  initial conditions $m(0)=0$ and $P(0)=P_c$, and terminated when 
  the surface condition $P(R)=0$ is reached. The corresponding 
  radius $R$ and gravitational mass $M=m(R)$ then characterize 
  the macroscopic quantities of the star.
  \item The ideal neutron gas EOS $\epsilon(P)$, obtained from the
  parametric Fermi gas expressions of Chapter~\ref{chap:fermi}.
  \item Dimensionless variables to reduce numerical instabilities.
\end{enumerate}
Together these ingredients define a well--posed system suitable for
numerical solution.

\section{Numerical Method}

The numerical integration of the TOV equations with an ideal neutron
gas EOS is carried out using a Python program originally developed by
Sletmoen as part of his Master's thesis~\cite{sletmoen2022}.
In this work the code is used without modification to generate stellar
models for the case of ideal neutron matter.

The implementation proceeds as follows:
\begin{itemize}
  \item The coupled ODEs for $P(r)$ and $m(r)$ are integrated outward
  from $r=0$ with initial values $P(0)=P_c$ and $m(0)=0$.
  \item A fifth--order Runge--Kutta method with adaptive step size is used to
  advance the solution.
  \item The integration terminates once $P(r)$ vanishes, which defines
  the stellar radius $R$. The corresponding $m(R)$ gives the
  gravitational mass $M$.
\end{itemize}
Repeating the calculation for a range of central pressures $P_c$
generates a sequence of stellar configurations. From these solutions
one can extract the macroscopic properties of the star, namely its mass
$M=m(R)$ and radius $R$ where $P(R)=0$. These results can then be
combined into the mass--radius relation. In addition, the solver also
provides access to the radial structure, meaning the profiles
of pressure $P(r)$ and energy density $\epsilon(r)$ throughout the
stellar interior. Such results reveal how matter is distributed
and how pressure balances gravity at each radius, complementing the
macroscopic description. In the following section we present both
macroscopic and structural results for ideal neutron stars.
These results can be directly compared with the analytic behavior found
for the uniform--density star in Chapter~\ref{chap:tov}.

\section{Results}
\subsection*{Mass--radius relation}

Figure~\ref{fig:mr_neutron} shows the resulting mass--radius curves for
ideal neutron stars.  The four models represent combinations of
Newtonian and relativistic hydrostatic equilibrium with either the
non--relativistic or the full relativistic form of the Fermi gas EOS.
Each point corresponds to a distinct central pressure $P_c$.

At low central pressures the curves coincide, reflecting the Newtonian
limit where gravitational fields are weak and the EOS is nearly
non--relativistic. As $P_c$ increases, however, the relativistic models
deviate markedly: the curves flatten and eventually exhibit a turnover,
indicating that beyond a certain maximum mass, no further stable
solutions exist. This behavior is analogous to the instability already 
encountered in the incompressible model, where the
pressure diverges at the Buchdahl bound $2M/R=8/9$.  In an ideal, non--interacting
neutron gas, the same effect manifests itself as the \emph{Oppenheimer--Volkoff
limit}~\cite{oppenheimer1939}: a maximum gravitational mass of order
$M_{\mathrm{max}}\!\approx\!0.7\,M_{\odot}$ for an ideal, non--interacting
neutron fluid.  Beyond this point, increasing the central pressure only
reduces the radius without adding mass, and the configuration becomes
unstable to collapse.

The Newtonian solutions, in contrast, show no turnover.  They predict
monotonically increasing mass with increasing central pressure, an
artifact of neglecting relativistic corrections to both gravity and the
EOS.  The fully relativistic treatment thus captures a key qualitative
feature: the existence of a maximum stable mass set by the balance
between degeneracy pressure and spacetime curvature.

\begin{figure}[ht]
  \centering
  \includegraphics[width=0.8\textwidth]{figures/mass_radius_ideal_neutron_star.pdf}
  \caption[Mass--radius relation for ideal neutron stars]{
  Mass--radius relation for ideal neutron stars. Four models are compared:
  Newtonian and relativistic hydrostatic equilibrium, each with a
  non--relativistic or fully relativistic Fermi gas EOS.  The color scale
  indicates the central pressure $P_c$.  The relativistic models exhibit a
  maximum mass, the Oppenheimer--Volkoff limit, beyond which no static
  equilibrium is possible.}
  \label{fig:mr_neutron}
\end{figure}

\subsection*{Pressure profiles}

Figure~\ref{fig:profiles_neutron} displays the normalized pressure
profiles $P(r)/P_c$ for a range of central pressures.
For small $P_c$, the pressure decreases almost linearly with radius,
resembling the Newtonian incompressible case.
As $P_c$ increases, the curves become increasingly concentrated toward
the center, revealing stronger gravitational compression and greater
compactness.  The outer layers contribute less to the total mass, while
the interior supports an ever larger fraction of the gravitational
weight.  This mirrors the analytic results for the constant--density
star, where high compactness led to steep gradients and eventual
divergence at the Buchdahl bound.

At the highest central pressures, the pressure gradient near the center
becomes extremely steep, signaling the proximity of the
Oppenheimer--Volkoff limit.  Beyond this limit, no equilibrium solution
exists: the star must collapse into a black hole.  Thus, the numerical
pressure profiles confirm the same qualitative physics inferred from the
analytic model—namely, that gravity strengthens faster than the pressure
support as compactness increases.

\begin{figure}[ht]
  \centering
  \includegraphics[width=0.8\textwidth]{figures/pressure_profile_ideal_neutron_star.pdf}
  \caption[Pressure profiles of ideal neutron stars]{
  Radial pressure profiles $P(r)/P_c$ for ideal neutron stars with
  different central pressures $P_c$.  The color scale represents
  $\log_{10}(P_c)$.  Increasing compactness leads to steeper gradients and
  a stronger concentration of pressure toward the center, consistent with
  the approach to the Oppenheimer--Volkoff limit.}
  \label{fig:profiles_neutron}
\end{figure}

\section{Radial stability analysis}
\label{sec:ideal_ns_stability}

\subsection{Physical motivation}

The mass--radius curves in Fig.~\ref{fig:mr_neutron} show that the
relativistic models reach a maximum mass and then turn over.
This turning point marks a change in stability. 
To understand why this happens, we study how small radial
perturbations behave in a star that is initially in hydrostatic
equilibrium.

If the star is slightly compressed, gravity and pressure respond in
opposite directions: gravity pulls matter inward while pressure pushes
it outward. 
If the pressure increase is strong enough to balance the added gravity,
the star will oscillate around its equilibrium shape and remain stable.
If gravity grows faster than the pressure response, the perturbation
will instead amplify with time and lead to collapse.

\subsection{Linearized equations}

A convenient way to study the stability of a relativistic star is to
examine how it responds to small, time--dependent, radial perturbations.
The stability problem was first solved in detail by
Chandrasekhar~\cite{chandrasekhar1964}, who derived the full set of
relativistic perturbation equations for a fluid sphere.
In this work we follow the later formulation of
Tooper~\cite{tooper1965}, in which the equations are rewritten in a form that is
more convenient for numerical integration and directly leads to the
Sturm--Liouville equation for the radial modes.

In this approach, one introduces a small radial displacement
$\xi(r,t)$ of the fluid layers and linearizes the Einstein and fluid
equations to first order in the perturbation amplitude.
The resulting system can be combined into a single
second--order differential equation for a variable $U(r,t)$ that is
proportional to $\xi(r,t)$.
Assuming harmonic time dependence $U(r,t)=U(r)\,e^{i\omega t}$ gives
\begin{equation}
\frac{d}{dr}\!\left[\Pi(r)\frac{dU}{dr}\right]
+ Q(r)\,U(r) + \omega^2 W(r)\,U(r) = 0.
\label{eq:SL_ideal_NS}
\end{equation}
The quantity $\omega^2$ is the squared oscillation frequency.
The coefficients $\Pi(r)$, $Q(r)$, and $W(r)$ depend only on the
background stellar structure, that is, on $P(r)$, $\epsilon(r)$,
$m(r)$, and the metric potential $\alpha(r)$ obtained from the TOV
equations.
They also include the adiabatic index
\[
\Gamma_1(r) = \frac{\epsilon(r)+P(r)}{P(r)}
\frac{dP}{d\epsilon}\bigg|_{r},
\]
which describes the local stiffness of the equation of state.

Equation \eqref{eq:SL_ideal_NS} has the standard form of a \emph{Sturm--Liouville eigenvalue
problem}. This is a general class of second--order linear
differential equations with specific boundary conditions that admit a
discrete set of eigenvalues and eigenfunctions.
Each allowed solution $U_n(r)$ corresponds to an \emph{eigenmode}, and
the associated constant $\omega_n^2$ is its \emph{eigenvalue}.
In the context of stellar oscillations, $\omega_n^2$ represents the
squared frequency of the $n$th radial mode.
If all eigenvalues are positive, the star performs stable oscillations.
When the lowest eigenvalue $\omega_0^2$ becomes zero or negative, the
oscillation changes into a growing mode, and the equilibrium becomes
dynamically unstable~\cite{arfken2013}.

Regularity at the center requires that the eigenfunction vanish at least
as $U(r)\!\sim\!r^{3}$ for $r\!\to\!0$, ensuring that the 
displacement $\xi = U\,e^{-\alpha}/r^{2}$ remains finite.  The cubic
form is the lowest--order regular behavior and is therefore used as the
boundary condition at $r = 0$.  At the surface $r = R$, the pressure
must vanish smoothly, giving the boundary condition $\Delta P(R) = 0$
for the Lagrangian pressure perturbation, which represents the total
change in pressure experienced by a moving fluid element,
$\Delta P = \delta P + \xi\,dP/dr$.  With these two boundary conditions,
the problem is well defined and can be solved for the eigenvalues
$\omega^{2}$.

\subsection{Numerical method}

Equation~\eqref{eq:SL_ideal_NS} forms an eigenvalue problem for the
frequency $\omega^2$. For each stellar model obtained from the TOV
equations, we can calculate the background functions
$P(r)$, $\epsilon(r)$, $m(r)$, and $\alpha(r)$, and from them construct
the coefficients $\Pi(r)$, $Q(r)$, and $W(r)$.
These three functions contain all the information about the background
structure and determine how the star responds to small radial perturbations.
They are given by~\cite{tooper1965}
\[
\Pi = e^{\beta+3\alpha}\frac{\Gamma_1 P}{r^2}, \qquad
Q = e^{\beta+3\alpha}\!\left(
  \frac{dP}{dr}\frac{d\alpha}{dr}
  - \frac{4}{r}\frac{dP}{dr}
  - 8\pi G e^{2\beta} P(\epsilon + P)
\right), \qquad
W = e^{3\beta+\alpha}\frac{\epsilon+P}{r^2}.
\]
where $\alpha(r)$ and $\beta(r)$ are the metric functions,
$\Gamma_1$ is the adiabatic index defined in the previous section, and
$P(r)$ and $\epsilon(r)$ are determined by the equation of state.
The equation is then fully specified and can be solved for the allowed
values of $\omega^2$.

In practice, this is done using a \emph{shooting method}.
We start by choosing a trial value of $\omega^2$ and integrate the
differential equation outward from the center using the regularity
condition $U(r)\propto r^3$ near $r=0$.
The integration proceeds until the surface $r=R$ is reached, where the
Lagrangian pressure perturbation must vanish, $\Delta P(R)=0$.
For most trial values, this boundary condition will not be satisfied:
the solution may diverge or exhibit an incorrect number of nodes,
indicating that the guessed eigenvalue $\omega^2$ is not an allowed one.

To find the correct eigenvalue, $\omega^2$ is adjusted and the
integration is repeated.
The number of nodes (zero crossings) of $U(r)$ provides a useful guide:
the fundamental mode has no internal nodes, the first overtone has one
node, and so on.
By searching for the value of $\omega^2$ that produces the correct
number of nodes and satisfies the surface condition, we obtain the
eigenvalue of that mode.
This process is repeated for several stellar models with different
central pressures $P_c$.

In the implementation used here, the coefficients
$\Pi(r)$, $Q(r)$, and $W(r)$ are evaluated numerically from the TOV
profiles, and the equation is integrated using a finite--difference
scheme with adaptive step size.
A bisection search is used to locate the eigenvalue $\omega^2$ to the
desired accuracy.
The full algorithm is based on the code developed by
Sletmoen~\cite{sletmoen2022}, used to analyze the stability of the ideal neutron star
sequence.

\subsection{Results and stability criterion}

The numerical integration of Eq.~\eqref{eq:SL_ideal_NS} yields the
discrete eigenvalues $\omega_n^2$ and eigenfunctions $U_n(r)$ for each
stellar model in the sequence.
The sign of the lowest eigenvalue determines the dynamical stability:
\[
\omega_0^2 > 0 \;\Rightarrow\; \text{stable}, \qquad
\omega_0^2 = 0 \;\Rightarrow\; \text{marginally stable}, \qquad
\omega_0^2 < 0 \;\Rightarrow\; \text{unstable}.
\]

Figure~\ref{fig:shooting_convergence} shows typical normalized mode
functions $U_n(r)$ obtained with the shooting method.
Each function satisfies the regularity condition $U_n(r)\propto r^3$
near the center and the vanishing Lagrangian pressure perturbation at
the surface.
The number of interior nodes increases with mode index $n$, as expected
from Sturm--Liouville theory: the fundamental mode $U_0(r)$ has no
nodes, $U_1(r)$ one node, and so on.
This confirms that the numerical integration behaves as required and
that the node theorem is fulfilled.
The smooth convergence of the solutions toward the correct boundary
condition at $r=R$ demonstrates the reliability of the shooting
algorithm.

\begin{figure}[ht]
  \centering
  \includegraphics[width=0.8\textwidth]{figures/shoot_shoot_relative.pdf}
  \caption[Shooting method convergence]{
  Shooting method convergence for the first few radial modes. Each line
  corresponds to a trial eigenfunction $U_n(r)$ normalized by
  $\max|U_0|$. The smooth approach to the boundary condition
  $\Delta P(R)=0$ illustrates successful eigenvalue convergence.}
  \label{fig:shooting_convergence}
\end{figure}

The corresponding normalized eigenfunctions are displayed in
Fig.~\ref{fig:nmodes_norm}.
They form a complete set of orthogonal modes describing the radial
oscillations of the star.
The fundamental mode is positive everywhere and represents a coherent
breathing motion of the entire star, while higher overtones exhibit
additional internal nodes associated with local compressions and
rarefactions.

\begin{figure}[ht]
  \centering
  \includegraphics[width=0.8\textwidth]{figures/nmodes_norm.pdf}
  \caption[Normalized eigenfunctions]{
  Normalized eigenfunctions $U_n(r)$ for the lowest radial modes. The
  number of interior nodes increases with $n$, as expected for a
  Sturm--Liouville spectrum.}
  \label{fig:nmodes_norm}
\end{figure}

The eigenfrequencies $\omega_n^2$ vary systematically along the TOV
sequence.
As the central pressure $P_c$ increases, the stellar mass grows until it
reaches a maximum at the turning point of the mass--radius curve.
At this point, the fundamental frequency satisfies $\omega_0^2 = 0$,
indicating marginal stability.
Configurations on the ascending branch ($dM/dP_c > 0$) have
$\omega_0^2 > 0$ and are dynamically stable; beyond the maximum
($dM/dP_c < 0$), $\omega_0^2$ becomes negative, signalling instability
against radial collapse.
This behaviour constitutes the general relativistic stability criterion
for spherical stars.
The equivalence between the turning point condition $dM/dP_c = 0$
and the vanishing of the fundamental frequency $\omega_0^2 = 0$
was first demonstrated by Chandrasekhar~\cite{chandrasekhar1964}
using a variational principle for radial oscillations.
In this formalism, the total energy of the configuration is stationary
at equilibrium, and the sign of the second variation determines
stability. The change of sign in $\omega_0^2$ at the mass maximum
therefore coincides with the point where the equilibrium sequence
passes from stable to unstable configurations.


Physically, the transition arises because general relativity
strengthens the effective gravitational attraction compared to
Newtonian theory.
At low compactness, the pressure response of the Fermi gas remains
sufficient to counteract gravity, and perturbations lead to stable
oscillations.
As the central density increases, relativistic corrections make gravity
grow more rapidly than pressure, reducing the effective restoring force.
Once the mass maximum is reached, any further compression increases the
gravitational binding faster than the pressure can respond, causing
small perturbations to amplify.
The star thus becomes dynamically unstable and collapses toward a more
compact configuration—typically a black hole in realistic equations of
state.

In summary, the numerical results verify the expected behaviour:
\begin{itemize}
  \item The fundamental mode frequency $\omega_0^2$ changes sign exactly
  at the maximum of the mass--radius curve.
  \item The node structure of $U_n(r)$ confirms the correct mode
  ordering.
  \item The shooting solutions converge smoothly and respect all boundary
  conditions.
\end{itemize}
This establishes that, for the ideal neutron star, the
Oppenheimer--Volkoff limit corresponds precisely to the onset of
dynamical instability against radial perturbations.

\subsection*{Discussion}

These results illustrate how the inclusion of relativistic gravity and a
degenerate Fermi gas EOS naturally produces a maximum stable mass for
neutron stars.  The simple ideal neutron model predicts a limiting mass
of only $M_{\mathrm{max}}\!\sim\!0.7\,M_{\odot}$, known as the
Oppenheimer--Volkoff limit~\cite{oppenheimer1939}.  In reality,
observations show several neutron stars with masses above
$2\,M_{\odot}$~\cite{linares2018,strader2019,linares2020,romani2022}.
The difference arises because the ideal model assumes non--interacting
neutrons, while real neutron matter is affected by strong nuclear forces
that make the material stiffer and able to support more mass.  Still,
the ideal neutron star captures the essential qualitative behavior:
general relativity imposes an upper bound on the stable mass, just as
the incompressible model indicated through the Buchdahl limit.  The
Oppenheimer--Volkoff limit therefore remains the simplest physical
expression of this relativistic instability.
