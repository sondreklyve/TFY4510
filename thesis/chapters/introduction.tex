\chapter{Introduction}
\label{chap:introduction}

\begin{quote}\itshape
When I heard the learn’d astronomer,\\
When the proofs, the figures, were arranged in columns before me,\\
When I was shown the charts and the diagrams, to add, divide, and measure them,\\
When I, sitting, heard the astronomer where he lectured with much applause in the lecture-room,\\
How soon unaccountable I became tired and sick,\\
Till rising and gliding out, I wandered off by myself,\\
In the mystical moist night-air, and from time to time,\\
Look’d up in perfect silence at the stars.
\par\raggedleft
Walt Whitman, \emph{When I Heard the Learn’d Astronomer}\cite{Whitman1855}
\end{quote}

Stars have been part of human life for as long as we know. They have been used
to tell time, to mark the seasons, to find the way at sea, and to shape myths
and religions. Today we know that they are distant suns, and they are measured
and modelled with modern instruments and computers, but they are still the
same lights that people have looked up at for thousands of years. For life on
Earth, one star is more important than all the others: the Sun. It sets the
cycle of day and night, provides the energy that drives the climate, and
ultimately supports the ecosystems we depend on.

From a physical point of view, stars like the Sun are large, hot spheres of gas
that shine because of nuclear reactions in their interiors. Their size and
shape are set by a balance between gravity, which pulls matter inward, and
pressure from hot particles and radiation, which pushes outward. For most of
their active lives this balance is stable, and the star emits light and heat at
a roughly constant rate. 

However, this is only one part of a star's history. More broadly, a star passes
through three qualitative stages: an initial phase of formation, a long middle
period of more or less stable burning, and a final stage in which it exhausts
its fuel and dies. Stars are born when cold, diffuse clouds of gas in a galaxy
slowly contract under their own gravity. As a cloud fragment collapses, its
central regions become denser and hotter. When the conditions are extreme
enough, nuclear fusion reactions start in the core: light nuclei combine into
heavier ones, releasing energy. Once the energy production settles into a
stable pattern that balances gravity, the newly formed star enters its
long-lived main sequence phase \cite{HansenKawalerTrimble2004}.

During this main sequence phase the star shines by steadily fusing hydrogen
into helium in its core. Gravity continues to draw matter inward, but the
pressure from the hot plasma, together with the energy released by fusion,
pushes outward. As long as the star can sustain this balance, it remains
stable. Over time, however, the hydrogen in the core is gradually used up. When
the easy nuclear fuel is exhausted, the fusion reactions slow down and can no
longer fully support the star against its own gravity. The core contracts and
heats up, and the outer layers adjust, sometimes expanding into a red giant. In
more massive stars, further nuclear reactions involving heavier nuclei can
occur, but this chain cannot continue indefinitely. At some point, fusing
heavier elements no longer produces energy. The star then reaches the end of
its active life.

What happens next depends mainly on how massive the star is. If the remaining
core is relatively light, it can settle into a compact object called a white
dwarf: a stellar remnant with roughly the mass of the Sun compressed into a
volume similar to that of the Earth. White dwarfs are held up by electron
degeneracy pressure, a quantum mechanical effect that arises because electrons
are fermions and cannot all occupy the same quantum state. When the electrons
are packed very densely, they are forced into higher and higher momentum
states, and this generates a pressure that does not rely on temperature in the
usual way. This mechanism can support a white dwarf only up to a certain mass,
the Chandrasekhar limit, of about \(1.4\,M_{\odot}\) \cite{chandrasekhar1931}.
Stellar cores with masses below this limit end their lives as white dwarfs.

More massive stars do not stop there. If the collapsing core exceeds the
Chandrasekhar limit, electron degeneracy pressure is not sufficient to halt the
collapse. The core continues to contract, the outer layers can be expelled in a
violent explosion known as a supernova, and the matter in the core is driven to
extreme densities. Under such conditions, protons and electrons combine to form
neutrons and neutrinos. The result is a neutron star: an object containing about
one to two times the mass of the Sun squeezed into a sphere with a radius of
only about ten to fifteen kilometres. To put this into perspective, a teaspoon
of neutron-star matter would weigh more than a large mountain on Earth. Neutron
stars are supported by neutron degeneracy pressure and by the strong
interactions between nucleons, but this support also has an upper limit. If the
mass of the collapsing core is larger than about two to three solar masses, even
these mechanisms are not enough, and the collapse continues until a black hole
is formed~\cite{oppenheimer1939,LattimerPrakash2004}.

Neutron stars occupy an intermediate position between white dwarfs and black
holes. Their typical masses lie around one to two times the mass of the Sun,
and theoretical models indicate that no stable neutron star can exist much
above about two to three solar masses; beyond this range the star is expected
to collapse to a black hole~\cite{LattimerPrakash2004}.
They are compact enough that the effects of
Einstein's theory of general relativity are essential for describing their
structure and gravitational field, but they are not so extreme as to be hidden
behind an event horizon. Their densities are so high that matter is pushed far
beyond the conditions that can be created in laboratories on Earth. A neutron
star therefore provides a unique natural laboratory for physics at the
intersection of gravity, nuclear and particle physics. On the one hand, our
theoretical description of these objects is based on extrapolating physical
laws that have been tested in less extreme regimes. On the other hand,
observations of neutron stars allow us to test and refine those laws under
conditions that would otherwise be inaccessible \cite{LattimerPrakash2004}.

The basic idea that such ultra-dense stellar remnants might exist dates back to
the early 1930s, shortly after the experimental discovery of the neutron in
1932 \cite{Chadwick1932}.
It was then realised that extremely dense matter could consist mainly of
neutrons. Neutron stars were proposed as possible remnants of supernova
explosions: stellar cores so dense that electrons and protons are forced to
combine into neutrons \cite{BaadeZwicky1934}.
For several decades this remained a theoretical
speculation. The situation changed in the late 1960s with the discovery of
pulsars: rapidly rotating, highly regular sources of radio pulses~\cite{Hewish1968}.
These were
soon interpreted as spinning neutron stars whose strong magnetic fields sweep
beams of radiation across our line of sight. Since then, neutron stars have
been observed in many different settings, including binary systems where matter
is accreted from a companion star, and in systems where two neutron stars orbit
around each other and eventually merge.

As observational techniques have improved, the range and precision of neutron
star measurements have increased significantly. High-precision radio timing,
developed from the late 1960s onwards, allows very accurate mass measurements
for pulsars in binary systems. X-ray observations of the hot surfaces of
neutron stars, which became possible in the 1970s, give information that can be
used, with some modelling, to infer their radii. More recently, in 2017, the
first gravitational-wave signal from a merging neutron-star binary was detected
by the LIGO and Virgo observatories, providing additional constraints because
the details of the signal depend on how easily the stars deform under tidal
forces~\cite{Abbott2017GW170817}.
All of these observations can be translated into bounds on the
relationship between the mass and radius of neutron stars \cite{LattimerPrakash2004}.

On the theoretical side, the structure of a non-rotating neutron star is
governed by two main ingredients. The first is general relativity, which
provides the equations that describe how gravity behaves in a static,
spherically symmetric star. These equations express the balance between
gravity, pressure, and mass at each radius in such a way that the star can
remain in hydrostatic equilibrium. The second ingredient is an input that tells
us how the pressure inside the star depends on the density: the so-called
equation of state. Physically, an equation of state summarises our knowledge
of what the star is made of and how its constituents interact. Once it is
specified, the relativistic structure equations can be integrated from the
centre outwards to give a complete model of the star and, in particular, a
prediction of how mass and radius are related \cite{LattimerPrakash2004}.

The equation of state of neutron-star matter at very high densities is not
known with certainty. In the outer regions of the star, matter is thought to
be relatively similar to very dense nuclear matter, arranged in a crust of
nuclei and electrons. Deeper inside, the density increases far beyond values
that can be reached in laboratory experiments, and our knowledge becomes much
more uncertain. Different theoretical models make different assumptions about
how matter behaves in this regime, and each choice leads to a different
equation of state and therefore to a different prediction for the relationship
between mass and radius. Comparing these predictions with astrophysical
observations is one of the main ways to learn about the behaviour of matter
under such extreme conditions \cite{LattimerPrakash2004}.

The aim of this project is to explore, in a controlled and stepwise way, how
different theoretical descriptions of dense matter translate into different
neutron-star structures. The focus is on a small number of models that are
simple enough to be treated in detail, but still realistic enough to show how
assumptions about the matter inside the star shape observable quantities such
as mass and radius. 

In the second chapter, we
introduce the framework of general relativity for static, spherically symmetric
stars and derive the Tolman-Oppenheimer-Volkoff (TOV) equations \cite{tolman1939,oppenheimer1939}.
These equations express the balance between
gravity and pressure in a compact star and form the backbone of all later
calculations. In the third chapter, we study ideal Fermi gases at zero
temperature. This provides a simple and transparent description of degenerate
matter and serves as a reference point for more realistic models. In the fourth
chapter, we combine these ingredients to build ideal neutron-star models based
on non-interacting neutrons, solve the TOV equations numerically, and analyse
the resulting mass--radius curves and their radial stability.

Having established this baseline, the fifth chapter turns to a more realistic
description of neutron-star matter in terms of the so-called \(npe\mu\) model,
in which neutrons and protons interact through a relativistic mean-field
theory and are accompanied by electrons and muons to ensure charge neutrality
and equilibrium \cite{SerotWalecka1986,glendenning2000}.
Within this framework, we construct the equation of state
self-consistently and extend it towards lower densities to include a simple
crust description. The resulting \(npe\mu\) equation of state is then used as
input for the TOV equations to compute mass--radius relations. Finally, these
theoretical curves are compared with observationally inferred bands for the
equation of state and for the mass--radius relation, illustrating how the
underlying model assumptions are reflected in the observable properties of
neutron stars.
