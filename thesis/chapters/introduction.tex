\chapter{Introduction}
\label{chap:introduction}

\begin{quote}\itshape
When I heard the learn’d astronomer,\\
When the proofs, the figures, were arranged in columns before me,\\
When I was shown the charts and the diagrams, to add, divide, and measure them,\\
When I, sitting, heard the astronomer where he lectured with much applause in the lecture-room,\\
How soon unaccountable I became tired and sick,\\
Till rising and gliding out, I wandered off by myself,\\
In the mystical moist night-air, and from time to time,\\
Look’d up in perfect silence at the stars.
\par\raggedleft
Walt Whitman, \emph{When I Heard the Learn’d Astronomer}
\end{quote}

On a clear night, far from streets and houses and the glow of city lights, the
sky opens up into a dark dome scattered with starlight. The stars do not just
appear as distant lamps; they give the night a quiet structure. Constellations
trace out shapes, familiar or imagined, and slow, patient patterns in their
motion mark the passing of the seasons. For most of human history, these
lights have been woven into stories and beliefs. They have served as a calendar
for planting and harvest, as a map for travellers and sailors, and as a canvas
for myths about origins and destiny. Long before anyone knew what a star
really is, people understood that the sky above was something both constant and
mysterious.

Among all these lights there is one that dominates our everyday experience,
although we rarely think of it as a star at all: the Sun. Its rising and
setting define the rhythm of our days. Its warmth makes liquid water and a
mild climate possible on Earth. The light it provides is taken up by plants and
turned into food, and through countless steps in the chain of life it becomes
our own energy. When clouds cover the sky for days, we feel its absence; when
it breaks through after a long winter, the world changes character. In a very
direct sense, everything familiar on the surface of our planet depends on this
one ordinary star.

From a physical point of view, stars like the Sun are enormous, hot spheres of
ionized gas that shine because of nuclear reactions in their interiors. Their
size and shape are maintained by a balance between two competing effects:
gravity, which pulls all the matter inward, and pressure from hot, fast-moving
particles and radiation, which pushes outward. For most of their active lives
these effects balance each other in a relatively stable way. This is the stage
we usually picture when we think about stars: long-lived objects steadily
emitting light and heat over billions of years.

However, this is only one part of a star's history. More broadly, a star passes
through three qualitative stages: an initial phase of formation, a long middle
period of more or less stable burning, and a final stage in which it exhausts
its fuel and dies. Stars are born when cold, diffuse clouds of gas in a galaxy
slowly contract under their own gravity. As a cloud fragment collapses, its
central regions become denser and hotter. When the conditions are extreme
enough, nuclear fusion reactions start in the core: light nuclei combine into
heavier ones, releasing energy. Once the energy production settles into a
stable pattern that balances gravity, the newly formed star enters its
long-lived main sequence phase.

During this main sequence phase the star shines by steadily fusing hydrogen
into helium in its core. Gravity continues to draw matter inward, but the
pressure from the hot plasma, together with the energy released by fusion,
pushes outward. As long as the star can sustain this balance, it remains
stable. Over time, however, the hydrogen in the core is gradually used up. When
the easy nuclear fuel is exhausted, the fusion reactions slow down and can no
longer fully support the star against its own gravity. The core contracts and
heats up, and the outer layers adjust, sometimes expanding into a red giant. In
more massive stars, further nuclear reactions involving heavier nuclei can
occur, but this chain cannot continue indefinitely. At some point, fusing
heavier elements no longer produces energy. The star then reaches the end of
its active life.

What happens next depends mainly on how massive the star is. If the remaining
core is relatively light, it can settle into a compact object called a white
dwarf: a stellar remnant with roughly the mass of the Sun compressed into a
volume similar to that of the Earth. White dwarfs are held up by quantum
mechanical effects rather than by ordinary thermal pressure. More massive stars
do not stop there. Their cores continue to collapse, and the outer layers can
be expelled in a violent explosion known as a supernova. The collapsing core is
subjected to extreme densities, and very different end states become possible.

For stars with sufficiently massive cores, the collapse can be so extreme that
even the quantum resistance of electrons is no longer enough to halt it. In
this case the core is compressed to densities comparable to those inside atomic
nuclei. Under such conditions, protons and electrons are driven together to
form neutrons and neutrinos. The result is a neutron star: an object containing
about one to two times the mass of the Sun squeezed into a sphere with a radius
of only about ten to fifteen kilometres. To put this into perspective, a
tablespoon of neutron-star material would weigh about the same as Mount Everest. 
If the mass of the collapsing core is even larger, gravity can overcome
all known forms of pressure and the end product is a black hole.

Neutron stars occupy an intermediate position between white dwarfs and black
holes. They are compact enough that the effects of Einstein's theory of general
relativity are essential for describing their structure and gravitational
field, but they are not so extreme as to be hidden behind an event horizon.
Their densities are so high that matter is pushed far beyond the conditions
that can be created in laboratories on Earth. A neutron star therefore
provides a unique natural laboratory for physics at the intersection of
gravity, nuclear physics, and quantum theory. On the one hand, our theoretical
description of these objects is based on extrapolating physical laws that have
been tested in less extreme regimes. On the other hand, observations of neutron
stars allow us to test and refine those laws under conditions that would
otherwise be inaccessible.

The basic idea that such ultra-dense stellar remnants might exist dates back to
the early 1930s, when it was first realized that extremely dense matter could
consist mainly of neutrons. Shortly afterwards, neutron stars were proposed as
possible remnants of supernova explosions: stellar cores so dense that
electrons and protons are forced to combine into neutrons. For several decades
this remained a theoretical speculation. The situation changed in the late
1960s with the discovery of pulsars: rapidly rotating, highly regular sources
of radio pulses. These were soon interpreted as spinning neutron stars whose
strong magnetic fields sweep beams of radiation across our line of sight. Since
then, neutron stars have been observed in many different settings, including
binary systems where matter is accreted from a companion star, and in systems
where two neutron stars orbit around each other and eventually merge.

As observational techniques have improved, the range and precision of neutron
star measurements have increased significantly. High-precision radio timing
allows very accurate mass measurements for pulsars in binary systems. X-ray
observations of the hot surfaces of neutron stars give information that can be
used, with some modelling, to infer their radii. Gravitational-wave detections
from merging neutron star binaries add further constraints, because the details
of the signal depend on how easily the stars deform under tidal forces. All of
these observations can be translated into bounds on the relationship between
the mass and radius of neutron stars.

On the theoretical side, the structure of a non-rotating neutron star is
governed by two main ingredients. The first is general relativity, which
provides the equations that describe how gravity behaves in a static,
spherically symmetric star. These equations express the balance between
gravity, pressure, and mass at each radius in such a way that the star can
remain in hydrostatic equilibrium. The second ingredient is an input that tells
us how the pressure inside the star depends on the density: the so-called
equation of state. Physically, the equation of state summarises our knowledge
of what the star is made of and how its constituents interact. Once it is
specified, the relativistic structure equations can be integrated from the
centre outwards to give a complete model of the star and, in particular, a
prediction of how mass and radius are related.

The equation of state of neutron-star matter at very high densities is not
known with certainty. In the outer regions of the star, matter is thought to
be relatively similar to very dense nuclear matter, arranged in a crust of
nuclei and electrons. Deeper inside, the density increases far beyond values
that can be reached in laboratory experiments, and our knowledge becomes much
more uncertain. Different theoretical models make different assumptions about
how matter behaves in this regime, and each choice leads to a different
equation of state and therefore to a different prediction for the relationship
between mass and radius. Comparing these predictions with astrophysical
observations is one of the main ways to learn about the behaviour of matter
under such extreme conditions.

The aim of this thesis is to explore, in a controlled and stepwise way, how
different theoretical descriptions of dense matter translate into different
neutron-star structures. The focus is on a small number of models that are
simple enough to be treated in detail, but still realistic enough to show how
assumptions about the matter inside the star shape observable quantities such
as mass and radius. 

In the first main chapter, we
introduce the framework of general relativity for static, spherically symmetric
stars and derive the Tolman-Oppenheimer-Volkov (TOV) equations. 
These equations express the balance between
gravity and pressure in a compact star and form the backbone of all later
calculations. In the second chapter, we study ideal Fermi gases at zero
temperature. This provides a simple and transparent description of degenerate
matter and serves as a reference point for more realistic models. In the third
chapter, we combine these ingredients to build ideal neutron-star models based
on non-interacting neutrons, solve the TOV equations numerically, and analyse
the resulting mass--radius curves and their radial stability.

Having established this baseline, the fourth chapter turns to a more realistic
description of neutron-star matter in terms of the so-called \(npe\mu\) model,
in which neutrons and protons interact through a relativistic mean-field
theory and are accompanied by electrons and muons to ensure charge neutrality
and equilibrium. Within this framework, we construct the equation of state
self-consistently and extend it towards lower densities to include a simple
crust description. The resulting \(npe\mu\) equation of state is then used as
input for the TOV equations to compute mass--radius relations. Finally, these
theoretical curves are compared with observationally inferred bands for the
equation of state and for the mass--radius relation, illustrating how the
underlying model assumptions are reflected in the observable properties of
neutron stars.
